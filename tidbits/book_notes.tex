    \documentclass[11pt,
        usenames, % allows access to some tikz colors
        dvipsnames % more colors: https://en.wikibooks.org/wiki/LaTeX/Colors
    ]{report}
    \usepackage{
        amsmath,
        amssymb,
        fouriernc, % fourier font w/ new century book
        fancyhdr, % page styling
        lastpage, % footer fanciness
        hyperref, % various links
        setspace, % line spacing
        amsthm, % newtheorem and proof environment
        mathtools, % \Aboxed for boxing inside aligns, among others
        float, % Allow [H] figure env alignment
        enumerate, % Allow custom enumerate numbering
        graphicx, % allow includegraphics with more filetypes
        wasysym, % \smiley!
        upgreek, % \upmu for \mum macro
        listings, % writing TrueType fonts and including code prettily
        tikz, % drawing things
        booktabs, % \bottomrule instead of hline apparently
        xcolor, % colored text
        cancel % can cancel things out!
    }
    \usepackage[margin=1in]{geometry} % page geometry
    \usepackage[
        labelfont=bf, % caption names are labeled in bold
        font=scriptsize % smaller font for captions
    ]{caption}
    \usepackage[font=scriptsize]{subcaption} % subfigures

    \newcommand*{\scinot}[2]{#1\times10^{#2}}
    \newcommand*{\dotp}[2]{\left<#1\,\middle|\,#2\right>}
    \newcommand*{\rd}[2]{\frac{\mathrm{d}#1}{\mathrm{d}#2}}
    \newcommand*{\pd}[2]{\frac{\partial#1}{\partial#2}}
    \newcommand*{\rdil}[2]{\mathrm{d}#1 / \mathrm{d}#2}
    \newcommand*{\pdil}[2]{\partial#1 / \partial#2}
    \newcommand*{\rtd}[2]{\frac{\mathrm{d}^2#1}{\mathrm{d}#2^2}}
    \newcommand*{\ptd}[2]{\frac{\partial^2 #1}{\partial#2^2}}
    \newcommand*{\md}[2]{\frac{\mathrm{D}#1}{\mathrm{D}#2}}
    \newcommand*{\pvec}[1]{\vec{#1}^{\,\prime}}
    \newcommand*{\svec}[1]{\vec{#1}\;\!}
    \newcommand*{\bm}[1]{\boldsymbol{\mathbf{#1}}}
    \newcommand*{\uv}[1]{\hat{\bm{#1}}}
    \newcommand*{\ang}[0]{\;\text{\AA}}
    \newcommand*{\mum}[0]{\;\upmu \mathrm{m}}
    \newcommand*{\at}[1]{\left.#1\right|}
    \newcommand*{\bra}[1]{\left<#1\right|}
    \newcommand*{\ket}[1]{\left|#1\right>}
    \newcommand*{\abs}[1]{\left|#1\right|}
    \newcommand*{\ev}[1]{\langle#1\rangle}
    \newcommand*{\p}[1]{\left(#1\right)}
    \newcommand*{\s}[1]{\left[#1\right]}
    \newcommand*{\z}[1]{\left\{#1\right\}}

    \newtheorem{theorem}{Theorem}[section]

    \let\Re\undefined
    \let\Im\undefined
    \DeclareMathOperator{\Res}{Res}
    \DeclareMathOperator{\Re}{Re}
    \DeclareMathOperator{\Im}{Im}
    \DeclareMathOperator{\Log}{Log}
    \DeclareMathOperator{\Arg}{Arg}
    \DeclareMathOperator{\Tr}{Tr}
    \DeclareMathOperator{\E}{E}
    \DeclareMathOperator{\Var}{Var}
    \DeclareMathOperator*{\argmin}{argmin}
    \DeclareMathOperator*{\argmax}{argmax}
    \DeclareMathOperator{\sgn}{sgn}
    \DeclareMathOperator{\diag}{diag\;}

    \colorlet{Corr}{red}

    % \everymath{\displaystyle} % biggify limits of inline sums and integrals
    \tikzstyle{circ} % usage: \node[circ, placement] (label) {text};
        = [draw, circle, fill=white, node distance=3cm, minimum height=2em]
    \definecolor{commentgreen}{rgb}{0,0.6,0}
    \lstset{
        basicstyle=\ttfamily\footnotesize,
        frame=single,
        numbers=left,
        showstringspaces=false,
        keywordstyle=\color{blue},
        stringstyle=\color{purple},
        commentstyle=\color{commentgreen},
        morecomment=[l][\color{magenta}]{\#}
    }

\begin{document}

\def\Snospace~{\S{}} % hack to remove the space left after autorefs
\renewcommand*{\sectionautorefname}{\Snospace}
\renewcommand*{\appendixautorefname}{\Snospace}
\renewcommand*{\figureautorefname}{Fig.}
\renewcommand*{\equationautorefname}{Eq.}
\renewcommand*{\tableautorefname}{Tab.}

\pagestyle{fancy}
\rfoot{Yubo Su}
\rhead{}
\cfoot{\thepage/\pageref{LastPage}}

\title{Miscellaneous Book Notes}
\author{Yubo Su}
\date{\today}

\maketitle

\chapter{Stein \& Shakarchi: Princeton Lectures in Analysis}

\section{Book 1: Fourier Analysis}

\begin{itemize}
    \item \emph{Lipschitz continuity} means continuity but also a bounded
        derivative.

    \item We define the vector space $\ell^2(\mathbb{Z})$ to be the set of all
        two-sided infinite sequences of complex numbers satisfying
        $\sum\limits_{n \in \mathbb{Z}}\abs{a_n}^2 < \infty$, i.e.\ the space of
        Fourier coefficients. This is an infinite-dimensional Hilbert space
        (inner product space such that the inner product is positive definite
        and complete, so every Cauchy sequence in the norm converges to a limit
        in the vector space).

    \item Note that the partial sums of the Fourier series of a function $f$ are
        convolututions with the \emph{Dirichlet kernel}, i.e.\ we have
        \begin{align}
            S_N(f)(x) &= \frac{1}{2\pi}\int\limits_{-\pi}^\pi
                f(y)\p{\sum\limits_{n = -N}^N e^{in(x - y)}}\;\mathrm{d}y,\\
                &= \p{f * D_N}(x),
        \end{align}
        where
        \begin{equation}
            D_N(x) = \sum\limits_{n = -N}^N e^{inx}.
        \end{equation}

    \item In general, we can consider a family of kernels $\z{K_n}_{n =
        1}^\infty$. Then families of \emph{good kernels} satisfy:
        \begin{itemize}
            \item For $n \geq 1$, $\int\limits_{-\pi}^\pi K_n(x)\;\mathrm{d}x
                = 2\pi$.
            \item There exists finite $M$ for which $\int\limits_{-\pi}^\pi
                \abs{K_n(x)}\;\mathrm{d}x \leq M$ for all $n \geq 1$.
            \item $\int\limits_{\delta \leq \abs{x} \leq
                \pi}\abs{K_n(x)}\;\mathrm{d}x \to 0$ as $n \to \infty$.
        \end{itemize}
        If $f$ is integrable, and $K_n$ are good kernels, then
        \begin{equation}
            \lim_{n \to \infty} \p{f * K_n}(x) = f(x),
        \end{equation}
        whenever $f$ is continuous at $x$. Moreover, if $f$ is continuous
        everywhere, the above limit is uniform. Sometimes, this is why good
        kernels are called an \emph{approximation to the identity}.

        In particular, the Dirichlet kernel is \emph{not} a good kernel, as the
        integral of the absolute value diverges $\propto \log N$.

    \item We know that a Fourier series can fail to converge at individual
        points, i.e.\ the limit
        \begin{equation}
            \lim_{N \to \infty}S_N(f) = f,
        \end{equation}
        where the $S_N$ are the sums of the first $N$ terms, does not converge.
        We resolve this with \emph{Ces\`aro} and \emph{Abel summability}.

        Suppose $s_n = \sum\limits_{k = 0}^n c_k$. Normally, we say $s_n$
        converges to $s$ if $\lim_{n \to \infty} s_n = s$, and is the most
        natural type of ``summability''. However, if this fails to converge, we
        can define the $N$th Ces\`aro mean or Ces\`aro sum by
        \begin{equation}
            \sigma_N = \frac{1}{N}\sum\limits_{n = 0}^{N - 1}s_n.
        \end{equation}
        If $\sigma_N$ converges to a limit as $N$ tends to infinity, we say that
        the original series $\sum\limits c_n$ is \emph{Ces\`aro summable} to
        $\sigma$. The archetypal Ces\`aro sum is the sum of alternating $\pm 1$,
        which Ces\`aro sums to $1/2$.

    \item Earlier, we saw that Dirichlet kernels are not good kernels, but their
        averages are well behaved. We see this by taking the $N$th Ces\`aro mean
        of the Fourier series
        \begin{align}
            \sigma_N(f)(x) &= \frac{1}{N}\sum\limits_{n = 0}^{n - 1}S_n(f)(x)
                    ,\\
                &= (f * F_n)(x),\\
            F_N(x) &= \frac{1}{N}\sum\limits_{n = 0}^{n - 1}
                D_{n}(x),\\
                &= \frac{1}{N}\frac{\sin^2(Nx/2)}{\sin^2(x/2)}.
        \end{align}
        This is the \emph{Fej\'er Kernel}, and is a good kernel. Thus, if $f$ is
        integrable, then the Fourier series of $f$ is Ces\`aro summable to $f$
        at every point of continuity of $f$, and is uniformly summable if $f$ is
        everywhere continous.

    \item Abel summability is an even more powerful notion of Ces\`aro
        summability. Given a series $c_k$, it is \emph{Abel summable} to $s$ if
        for every $0 \leq r < 1$, the series
        \begin{equation}
            A(r) = \sum\limits_{k = 0}^\infty c_kr^k
        \end{equation}
        converges, and
        \begin{equation}
            \lim_{r \to 1}A(r) = s.
        \end{equation}
        These $A(r)$ are the \emph{Abel means} of the series. Abel summation
        shows that $1 - 2 + 3 - 4 \dots = 1/4$, since
        \begin{equation}
            A(r) = \sum\limits_{k = 0}^\infty \p{-1}^k(k + 1)r^k
                = \frac{1}{(1+r)^2}.
        \end{equation}

    \item Similarly to how Ces\`aro summation gave the Fej\'er Kernel, Abel
        summation gives the \emph{Poisson kernel}:
        \begin{align}
            A_r(f)(\theta) &= \sum\limits_{n = -\infty}^\infty r^{\abs{n}}
                    a_ne^{in\theta},\\
                &= (f * P_r)(\theta),\\
            P_r(\theta) &= \sum\limits_{n = -\infty}^\infty
                r^{\abs{n}}e^{in\theta}.
        \end{align}
        Again, the Poisson kernel is a good kernel for $0 \leq r < 1$.

    \item Recall that the Fourier series converges in the mean-square sense:
        \begin{equation}
            \lim_{N \to \infty} \frac{1}{2\pi}\int\limits_{0}^{2\pi}
                \abs{f(\theta) - S_N(f)(\theta)}^2\;\mathrm{d}\theta
                    = 0,
        \end{equation}
        and moreover the coefficients of the $N$th partial sum are the unique
        best approximation of the first $N$ harmonics.

        Note that the terms of a converging series must tend to $0$, so the
        Fourier coefficients must go to zero as well. This is the
        \emph{Reimann-Lebesgue Lemma}:
        \begin{equation}
            \lim_{N \to \infty}
                \int\limits_0^{2\pi}f(\theta)\sin(N\theta)\;\mathrm{d}\theta
                = 0.
        \end{equation}

    \item Consider $f$ Lipschitz continuous at $\theta_0$ ($\abs{f(\theta) -
        f(\theta_0)} \leq M\abs{\theta - \theta_0}$ for some $M \geq 0$ and all
        $\theta$) and differentiable. Then the Fourier series converges at
        $\theta_0$ as $N \to \infty$.

        Construct
        \begin{equation}
            F(t) =
            \begin{cases}
                (f(\theta_0 - t) - f(\theta_0)) / t & t \neq 0,\\
                -f'(\theta_0) & t = 0.
            \end{cases}
        \end{equation}
        It is easy then to show that
        \begin{align}
            S_N(f)(\theta_0) - f(\theta_0) &= \frac{1}{2\pi}
                \int\limits_{-\pi}^\pi f(\theta_0 - t) D_n(t)\;\mathrm{d}t
                - f(\theta_0),\\
            &= \frac{1}{2\pi}
                \int\limits_{-\pi}^\pi \p{f(\theta_0 - t)
                - f(\theta_0)} D_n(t)\;\mathrm{d}t,\\
            &= \frac{1}{2\pi}\int\limits_{-\pi}^\pi
                F(t)tD_n(t)\;\mathrm{d}t,\\
            tD_n(t) &= \frac{t}{\sin(t/2)}\sin\p{\p{N + \frac{1}{2}}t},
        \end{align}
        where $D_n(t)$ is the Dirichlet kernel. Then the Reimann-Lebesgue lemma
        implies the second to last line vanishes, as the integrand is
        Reimann-integrable. We should be surprised by this, since this implies
        pointwise convergence depends only on the behavior of $f$ near
        $\theta_0$, even though the coefficients are obtained by integrating
        over all $\theta$.

        Note that above we required the function be differentiable. Otherwise,
        functions can be very carefully constructed to fail to pointwise
        converge.

    \item Fourier analysis can be used to prove \emph{Weyl's equidistribution
        theorem}. First, note that, for any real $\gamma \neq 0$, the sequence
        of numbers $\ev{n\gamma}$, where $\ev{X}$ denotes the fractional part,
        is either repeating for rational $\gamma$ or never repeating for
        irrational $\gamma$. Weyl's equidistribution theorem goes further and
        says that the $\ev{n\gamma}$ are equidistributed (and thus dense) on $0
        \leq x < 1$.

        The proof is simple. Consider any interval $(a, b)$ on the unit
        interval, and extend its membership/indicator function over $\mathbb{R}$
        by periodicity, defined $\chi_{(a,b)}(x)$. The number of times
        $\ev{n\gamma}$ is in the interval is just the sum of
        $\chi_{(a,b)}(n\gamma)$, and equidistribution becomes the statement
        \begin{equation}
            \lim_{N \to \infty}
                \frac{1}{N}\sum\limits_{n = 1}^N \chi_{(a,b)}(n\gamma)
                = \int\limits_0^1 \chi_{(a,b)}(x)\;\mathrm{d}x.
        \end{equation}
        This can be shown by verifying linearity, then for each of the individual
        Fourier harmonics. Clever!

        This theorem generalizes: a sequence $\z{\xi_n}$ over the unit interval
        is equidistributed iff
        \begin{equation}
            \lim_{N \to \infty}
                \frac{1}{N}\sum\limits_{n = 1}^N e^{2\pi i k \xi_n} = 0.
        \end{equation}
        This is \emph{Weyl's criterion}.
\end{itemize}

Moving onto Fourier Transforms (continuous):
\begin{itemize}
    \item The space over which FTs can be taken is $\mathcal{M}(\mathbb{R})$,
        the set of functions of \emph{moderate decrease}, i.e.\ that fall off
        at least as fast as $1/x^2$. This forms a vector space under addition
        and scalar multiplication. $1/x^2$ is necessary so that the integral
        over $\s{-N, N}$ as $N \to \infty$ falls off like $1/N$ and converges.

        However, $\mathcal{M}$ is insufficient to provide guarantees on the
        integral of the transform. Instead, consider the \emph{Schwartz space},
        denoted $\mathcal{S}(\mathbb{R})$, which falls off faster than any
        power of $x$ (e.g.\ Gaussian); this implies its derivatives do as well.
        The Gaussian is a family of good kernels on the real line as their width
        goes to zero. The FT generally does not require
        $\mathcal{S}(\mathbb{R})$, and just requires both FT and $f$ to be in
        $\mathcal{M}$.

    \item The \emph{Poisson summation formula} gives a way to construct a
        \emph{periodization} of a function $f \in \mathcal{S}$. It says
        \begin{equation}
            \sum\limits_{n = -\infty}^\infty f(x + n)
                = \sum\limits_{n = -\infty}^\infty \hat{f}(n) e^{2\pi inx}.
        \end{equation}
        This just follows from the definition of the FT, then the periodization
        is
        \begin{equation}
            F_1(x) = \sum\limits_{n = -\infty}^\infty f(x + n).
        \end{equation}

    \item Some special functions that are related to this are: the \emph{theta
        function}
        \begin{equation}
            \theta(s) = \sum\limits_{n = -\infty}^\infty e^{-\pi n^2s}.
        \end{equation}
        Note that $s^{-1/2}\theta(1/s) = \theta(s)$ for $s > 0$, which follows
        from the Poisson summation formula for $\exp(-\pi sx^2)$. This is
        connected to the zeta function
        \begin{equation}
            \zeta(s) = \sum\limits_{n=1}^\infty \frac{1}{n^s},
        \end{equation}
        and the $\Gamma$ function via
        \begin{equation}
            \pi^{-s/2}\Gamma\p{s/2}\zeta(s)
                = \frac{1}{2}\int\limits_{0}^\infty t^{s/2 - 1}
                    \p{\theta(t) - 1}\;\mathrm{d}t.
        \end{equation}

    \item The \emph{Radon transform} is a useful concept in imaging, and
        concerns the following: a beam of incident intensity $I_0$ passes
        through a medium with variable attenuation coefficient $\rho$, then the
        Radon transform is
        \begin{equation}
            X(\rho)(L) = \int\limits_L \rho(s) \;\mathrm{d}s.
        \end{equation}
        We then want to ask whether it is possible to invert $X(\rho)$, which
        contains the attenuation information for all lines $L$. Counting
        dimensions, we see that in 2D, the space of lines is of dimension $2$,
        as is the medium, so the inverse may exist, and it turns out it does,
        but is tricky to deal with so we discuss 3D.

        In 3D, the space of lines is of dimension $4$, while the medium is only
        of dimension $3$. Instead, the more natural generalization is to
        integrate over \emph{planes}. So the Radon transform maps from a plane
        (characterized by a tangent vector, so 3D) and a time to a scalar, and
        is given by
        \begin{equation}
            \mathcal{R}(f)(t, \gamma) = \int\limits_{P_{t, y}}
                f\;\mathrm{d}A.
        \end{equation}
        It turns out that the result is (and can be given by FTs)
        \begin{equation}
            \nabla^2\p{\mathcal{R}^* \mathcal{R}(f)}
                = -8\pi^2 f.
        \end{equation}

    \item By applying Fourier analysis to finite spaces, we can prove
        \emph{Dirichlet's theorem} on primes in arithmetic progression: if $q,
        l$ are positive integers with no common factor, then $l + kq$ for $k \in
        \mathbb{Z}$ contains infinitely many prime numbers. I won't actually
        ever use this, so see Ch 8 of this book.
\end{itemize}

\section{Book 3: Measure Theory, Integration \& Hilbert Spaces}

\begin{itemize}
    \item There are a few problems that arise from the traditional notions of
        integrability/differentiability/continuity. For instance, the Fourier
        transform maps the space of Reimann-integrable functions $\mathcal{R}$ to
        the space of Fourier coefficients, denoted $\ell^2(\mathbb{Z})$.
        However, $\ell^2(\mathbb{Z})$ is \emph{complete}, while $\mathcal{R}$ is
        not. The question is then: how do we complete $\mathcal{R}$, and how do
        we integrate these completed functions $f$?
\end{itemize}

\end{document}

