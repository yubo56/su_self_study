    \documentclass[10pt]{article}
    \usepackage{fancyhdr, amsmath, amsthm, amssymb, mathtools, lastpage,
    hyperref, enumerate, graphicx, setspace, wasysym, upgreek, listings}
    \usepackage[margin=0.5in, top=0.8in,bottom=0.8in]{geometry}
    \newcommand{\scinot}[2]{#1\times10^{#2}}
    \newcommand{\bra}[1]{\left<#1\right|}
    \newcommand{\ket}[1]{\left|#1\right>}
    \newcommand{\dotp}[2]{\left<#1\,\middle|\,#2\right>}
    \newcommand{\rd}[2]{\frac{\mathrm{d}#1}{\mathrm{d}#2}}
    \newcommand{\pd}[2]{\frac{\partial#1}{\partial#2}}
    \newcommand{\rtd}[2]{\frac{\mathrm{d}^2#1}{\mathrm{d}#2^2}}
    \newcommand{\ptd}[2]{\frac{\partial^2 #1}{\partial#2^2}}
    \newcommand{\norm}[1]{\left|\left|#1\right|\right|}
    \newcommand{\abs}[1]{\left|#1\right|}
    \newcommand{\pvec}[1]{\vec{#1}^{\,\prime}}
    \newcommand{\tensor}[1]{\overleftrightarrow{#1}}
    \let\Re\undefined
    \let\Im\undefined
    \newcommand{\ang}[0]{\text{\AA}}
    \newcommand{\mum}[0]{\upmu \mathrm{m}}
    \DeclareMathOperator{\Re}{Re}
    \DeclareMathOperator{\Im}{Im}
    \DeclareMathOperator{\Log}{Log}
    \DeclareMathOperator{\Arg}{Arg}
    \DeclareMathOperator{\Tr}{Tr}
    \DeclareMathOperator{\E}{E}
    \DeclareMathOperator{\Var}{Var}
    \DeclareMathOperator*{\argmin}{argmin}
    \DeclareMathOperator*{\argmax}{argmax}
    \DeclareMathOperator{\sgn}{sgn}
    \newcommand{\expvalue}[1]{\left<#1\right>}
    \usepackage[labelfont=bf, font=scriptsize]{caption}\usepackage{tikz}
    \usepackage[font=scriptsize]{subcaption}
    \everymath{\displaystyle}
    \lstset{basicstyle=\ttfamily\footnotesize,frame=single,numbers=left}

\tikzstyle{circ} = [draw, circle, fill=white, node distance=3cm, minimum
height=2em]

\begin{document}

\pagestyle{fancy}
\rhead{Yubo Su --- Tidbits}
\cfoot{\thepage/\pageref{LastPage}}

Welcome back to my random tidbits file! When I come up with interesting
problems, I will put them here.

\section{Probability Distributions and Weight Loss}

I was keeping track of my own weight when I realized that my scale was
sufficiently inconsistent that my weight loss was dominated by the statistical
noise. So then I was curious what the best way of mitigating this is, mean or
median of multiple measurements. One would suspect it's the mean, or one would
know simply by having taken any real statistics class, but I'm curious.

\subsection{Mean-based averaging}

This one is easy. Assume we have $n$ iid variables $X_i$ with mean $\mu$ and
variance $\sigma^2$, then the random variable corresponding to their average
$\expvalue{X_i}$ has mean $\mu$ and variance $\frac{\sigma^2}{n}$, so standard
deviation $\frac{\sigma}{\sqrt{n}}$. Thus, we have an unbiased estimator of the
true mean and a variance that falls off like $\sim n^{-1/2}$.


\subsection{Median-based averaging}

This one is a bit more fun. Let's start with $n=3$, then defining $f(x)$ the
probability density function and $F_X(x) = f_X(X \leq x)$ the cumulative
distribution function, the probability density of the median $f_\eta(y)$ is
given
\begin{align}
    f_\eta(y) = 6f_X(y)F_X(y)\left( 1 - F_X(y) \right)\label{1-eta}
\end{align}
the probability we choose one value greater than $y$ the median and one less,
multiplied by $6$ for orderings. This seems to be a bit difficult to verify to
be normalized in the general case, or that
\begin{align}
    \int\limits_{-\infty}^{\infty}f_\eta(y)\;\mathrm{d}y &=
    \int\limits_{-\infty}^{\infty}\left[
        6f_X(y)\int\limits_{-\infty}^{y}f_X(\xi)\;\mathrm{d}\xi
        \int\limits_{y}^{\infty}f_X(\zeta)\;\mathrm{d}\zeta
    \right]\mathrm{d}y = 1
\end{align}

Let's just verify this in the uniform distribution case, and leave the general
case as an exercise to brighter colleagues. We consider the normalized uniform
distribution $f_X(x) = 1, x \in [0,1]$, or $F_X(x) = x, x \in [0,1]$. We confirm
that the expression for $f_\eta$ is normalized:
\begin{align}
    \int\limits_{0}^{1}6y(1-y)\;\mathrm{d}y = 1
\end{align}

We then wish to examine whether $f_\eta(y)$ is an unbiased estimator of $\mu$.
Again, we begin with examining a sub-case, where $f_X(x)$ is symmetric about its
mean $\mu$. This yields that $F_X(\mu) = 0.5$ and is odd about
$\mu$\footnote{This is a slight abuse of terminology: we mean that $F_X(x - \mu)
- 0.5 = -(F_X(-(x-\mu)) - 0.5)$.} and so that $F_X(y)\left( 1 - F_X(y) \right)$
is also even/symmetric about $\mu$. Finally, this implies that $f_\eta(y)$ as
defined in \autoref{1-eta} is also symmetric about $\mu$ and we are done.

However, this analysis breaks down in the asymmetric case. We see that
$F_X(y)\left( 1 - F_X(y) \right)$ is \emph{always} symmetric about the
median $\eta$ of $f_X$, since $F_X(\eta) = 0.5$. In general, the mean and median
of a probability distribution are not equal, so there is no guarantee that
$\expvalue{f_\eta(y)} = \expvalue{f(y)}$, and indeed we can verify for some
contrived probability distribution such as
\begin{align}
    f_X(x) &=
    \begin{cases}
        2 & 0 \leq x \leq 0.25\\
        1 & 0.5 \leq x \leq 1\\
        0 & \text{else}
    \end{cases}
\end{align}
that $\expvalue{f_X(x)} = 0.4375$ while
\begin{align}
    \expvalue{f_\eta(y)} &= \int\limits_{0}^{0.25}24y^2(1-2y)\;\mathrm{d}y +
    \int\limits_{0.5}^{1}6y^2(1-y)\;\mathrm{d}y\\
    &\approx 0.4218
\end{align}

\subsection{Open Questions}

\begin{itemize}
    \item If we have discretized measurements, what are the statistics of
        mode-based averaging?
    \item Did I actually normalize the median-based averaging correctly, for a
        general probability distribution?
\end{itemize}

\end{document}

