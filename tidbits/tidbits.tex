    \documentclass[10pt]{article}
    \usepackage{fancyhdr, amsmath, amsthm, amssymb, mathtools, lastpage, hyperref, enumerate, graphicx, setspace, wasysym, upgreek, listings}
    \usepackage[margin=0.5in, top=0.8in,bottom=0.8in]{geometry}
    \newcommand{\scinot}[2]{#1\times10^{#2}}
    \newcommand{\bra}[1]{\left<#1\right|}
    \newcommand{\ket}[1]{\left|#1\right>}
    \newcommand{\dotp}[2]{\left<#1\,\middle|\,#2\right>}
    \newcommand{\rd}[2]{\frac{\mathrm{d}#1}{\mathrm{d}#2}}
    \newcommand{\pd}[2]{\frac{\partial#1}{\partial#2}}
    \newcommand{\rtd}[2]{\frac{\mathrm{d}^2#1}{\mathrm{d}#2^2}}
    \newcommand{\ptd}[2]{\frac{\partial^2 #1}{\partial#2^2}}
    \newcommand{\norm}[1]{\left|\left|#1\right|\right|}
    \newcommand{\abs}[1]{\left|#1\right|}
    \newcommand{\pvec}[1]{\vec{#1}^{\,\prime}}
    \newcommand{\tensor}[1]{\overleftrightarrow{#1}}
    \let\Re\undefined
    \let\Im\undefined
    \newcommand{\ang}[0]{\text{\AA}}
    \newcommand{\mum}[0]{\upmu \mathrm{m}}
    \DeclareMathOperator{\Re}{Re}
    \DeclareMathOperator{\Im}{Im}
    \DeclareMathOperator{\Log}{Log}
    \DeclareMathOperator{\Arg}{Arg}
    \DeclareMathOperator{\Tr}{Tr}
    \DeclareMathOperator{\E}{E}
    \DeclareMathOperator{\Var}{Var}
    \DeclareMathOperator*{\argmin}{argmin}
    \DeclareMathOperator*{\argmax}{argmax}
    \DeclareMathOperator{\sgn}{sgn}
    \newcommand{\expvalue}[1]{\left<#1\right>}
    \usepackage[labelfont=bf, font=scriptsize]{caption}\usepackage{tikz}
    \usepackage[font=scriptsize]{subcaption}
    \everymath{\displaystyle}
    \lstset{basicstyle=\ttfamily\footnotesize,frame=single,numbers=left}

\tikzstyle{circ} = [draw, circle, fill=white, node distance=3cm, minimum height=2em]

\begin{document}

\pagestyle{fancy}
\rhead{Yubo Su --- Tidbits}
\cfoot{\thepage/\pageref{LastPage}}

Welcome back to my random tidbits file! When I come up with interesting problems, I will put them here.

\section{Probability Distributions}

I was keeping track of my own weight when I realized that my scale was sufficiently inconsistent that my weight loss was dominated by the statistical noise. So then I was curious what the best way of mitigating this is, mean or median of multiple measurements. One would suspect it's the mean, or one would know simply by having taken any real statistics class, but I'm curious.

\subsection{Mean-based averaging}

This one is easy. Assume we have $n$ iid variables $X_i$ with mean $\mu$ and variance $\sigma^2$, then the random variable corresponding to their average $\expvalue{X_i}$ has mean $\mu$ and variance $\frac{\sigma^2}{n}$, so standard deviation $\frac{\sigma}{\sqrt{n}}$. Thus, we have an unbiased estimator of the true mean and a variance that falls off like $\sim n^{-1/2}$.


\subsection{Median-based averaging}

This one is a bit more fun. Let's start with $n=3$, then defining $P(X)$ the probability density function and $F(x) = P(X \leq x)$ the cumulative distribution function, the probability density of the median $P_\mu(y)$ (forgive the reuse of notation, typing this up before work) is given
\begin{align}
    P_\mu(y) = 2F(y)\left( 1 - F(y) \right)\label{1.mu}
\end{align}
the probability we choose one value greater than $y$ the median and one less, multiplied by $2$ for orderings. We verify that this is normalized in the case of a normally distributied $P(x)$ because the integral of \autoref{1.mu} is simply $2$ in the first term

\subsection{Mode-based averaging?}

% TODO

\end{document}

