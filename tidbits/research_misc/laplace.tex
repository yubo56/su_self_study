    \documentclass[11pt,
        usenames, % allows access to some tikz colors
        dvipsnames % more colors: https://en.wikibooks.org/wiki/LaTeX/Colors
    ]{article}
    \usepackage{
        amsmath,
        amssymb,
        fouriernc, % fourier font w/ new century book
        fancyhdr, % page styling
        lastpage, % footer fanciness
        hyperref, % various links
        setspace, % line spacing
        amsthm, % newtheorem and proof environment
        mathtools, % \Aboxed for boxing inside aligns, among others
        float, % Allow [H] figure env alignment
        enumerate, % Allow custom enumerate numbering
        graphicx, % allow includegraphics with more filetypes
        wasysym, % \smiley!
        upgreek, % \upmu for \mum macro
        listings, % writing TrueType fonts and including code prettily
        tikz, % drawing things
        booktabs, % \bottomrule instead of hline apparently
        xcolor, % colored text
        cancel % can cancel things out!
    }
    \usepackage[margin=1in]{geometry} % page geometry
    \usepackage[
        labelfont=bf, % caption names are labeled in bold
        font=scriptsize % smaller font for captions
    ]{caption}
    \usepackage[font=scriptsize]{subcaption} % subfigures

    \newcommand*{\scinot}[2]{#1\times10^{#2}}
    \newcommand*{\dotp}[2]{\left<#1\,\middle|\,#2\right>}
    \newcommand*{\rd}[2]{\frac{\mathrm{d}#1}{\mathrm{d}#2}}
    \newcommand*{\pd}[2]{\frac{\partial#1}{\partial#2}}
    \newcommand*{\rdil}[2]{\mathrm{d}#1 / \mathrm{d}#2}
    \newcommand*{\pdil}[2]{\partial#1 / \partial#2}
    \newcommand*{\rtd}[2]{\frac{\mathrm{d}^2#1}{\mathrm{d}#2^2}}
    \newcommand*{\ptd}[2]{\frac{\partial^2 #1}{\partial#2^2}}
    \newcommand*{\md}[2]{\frac{\mathrm{D}#1}{\mathrm{D}#2}}
    \newcommand*{\pvec}[1]{\vec{#1}^{\,\prime}}
    \newcommand*{\svec}[1]{\vec{#1}\;\!}
    \newcommand*{\bm}[1]{\boldsymbol{\mathbf{#1}}}
    \newcommand*{\uv}[1]{\hat{\bm{#1}}}
    \newcommand*{\ang}[0]{\;\text{\AA}}
    \newcommand*{\mum}[0]{\;\upmu \mathrm{m}}
    \newcommand*{\at}[1]{\left.#1\right|}
    \newcommand*{\bra}[1]{\left<#1\right|}
    \newcommand*{\ket}[1]{\left|#1\right>}
    \newcommand*{\abs}[1]{\left|#1\right|}
    \newcommand*{\ev}[1]{\left\langle#1\right\rangle}
    \newcommand*{\p}[1]{\left(#1\right)}
    \newcommand*{\s}[1]{\left[#1\right]}
    \newcommand*{\z}[1]{\left\{#1\right\}}

    \newtheorem{theorem}{Theorem}[section]

    \let\Re\undefined
    \let\Im\undefined
    \DeclareMathOperator{\Res}{Res}
    \DeclareMathOperator{\Re}{Re}
    \DeclareMathOperator{\Im}{Im}
    \DeclareMathOperator{\Log}{Log}
    \DeclareMathOperator{\Arg}{Arg}
    \DeclareMathOperator{\Tr}{Tr}
    \DeclareMathOperator{\E}{E}
    \DeclareMathOperator{\Var}{Var}
    \DeclareMathOperator*{\argmin}{argmin}
    \DeclareMathOperator*{\argmax}{argmax}
    \DeclareMathOperator{\sgn}{sgn}
    \DeclareMathOperator{\diag}{diag\;}

    \colorlet{Corr}{red}

    % \everymath{\displaystyle} % biggify limits of inline sums and integrals
    \tikzstyle{circ} % usage: \node[circ, placement] (label) {text};
        = [draw, circle, fill=white, node distance=3cm, minimum height=2em]
    \definecolor{commentgreen}{rgb}{0,0.6,0}
    \lstset{
        basicstyle=\ttfamily\footnotesize,
        frame=single,
        numbers=left,
        showstringspaces=false,
        keywordstyle=\color{blue},
        stringstyle=\color{purple},
        commentstyle=\color{commentgreen},
        morecomment=[l][\color{magenta}]{\#}
    }

\begin{document}

\section{Laplace Plane Dynamics}

\subsection{Maximum Separatrix Area---Simple}

Consider a planet with orbit normal $\uv{l}_{\rm p}$ that experiences precession
driven by stellar oblateness $\uv{l}_{\rm s}$ and an outer perturber
$\uv{l}_{\rm o}$. We assume that the planet's orbit is circular. The vector
form of the precessional dynamics are (e.g.\ Tremaine+2009, Eq~19):
\begin{align}
    \rd{\uv{l}_{\rm p}}{t}
        &= \omega_{\rm sp} \p{\uv{l}_{\rm p} \cdot \uv{l}_{\rm s}}
            \p{\uv{l}_{\rm p} \times \uv{l}_{\rm s}}
            + \omega_{\rm op} \p{\uv{l}_{\rm p} \cdot \uv{l}_{\rm o}}
            \p{\uv{l}_{\rm p} \times \uv{l}_{\rm o}}.
\end{align}

We first make an important symmetry argument: in the limits of $\omega_{\rm sp}
\ll \omega_{\rm op}$ or $\omega_{\rm sp} \gg \omega_{\rm op}$, the evolution of
$\uv{l}_{\rm p}$ consist of uniform precession about $\uv{l}_{\rm o}$ and
$\uv{l}_{\rm p}$ respectively, and thus the separatrix area must go to zero in
these limits. In fact, the phase portrait must the same under the following
transformation: swap the two frequencies $(\omega_{\rm sp}, \omega_{\rm op})$
and the two vectors $(\uv{l}_{\rm o}, \uv{l}_{\rm s})$. Swapping the precession
frequencies is equivalent to taking $a / r_{\rm M} \mapsto r_{\rm M} / a$ (and
rescaling time), since $\omega_{\rm op} / \omega_{\rm sp} = (a / r_{\rm M})^5$.
Thus, we arrive at an important conclusion: \emph{the phase portraits are
equivalent, up to a rotation of reference frame, for any two $r_{\rm M,1}$ and
$r_{\rm M, 2}$ satisfying $a / r_{\rm M, 1} = r_{\rm M, 2} / a$}. This implies
that the separatrix area is symmetric about $r_{\rm M} = a$ as well.

It's not clear that the separatrix area must be monotonic between $r_{\rm M} \in
\s{0, a}$, but intuitively \textbf{this seems like it should be the case} (?),
since there are no special values of $\omega_{\rm sp} / \omega_{\rm op}$ in the
equation of motion. If so, then the maximum separatrix area is obtained for
$r_{\rm M} = a$. The curve for the separatrix in this case is significantly
easier to obtain, though it still seems difficult to integrate explicitly (maybe
there's a clever idea?).

To compute the separatrix area for $a = r_{\rm M}$, we note that the
low-obliquity Laplace equilibrium P1 is located exactly halfway between
$\uv{l}_{\rm o}$ and $\uv{l}_{\rm s}$. Thus, we choose the reference frame such
that $\uv{z}\propto \uv{l}_{\rm o} + \uv{l}_{\rm s}$, and we choose $\uv{y}$ to
point towards P2 (which is always $\pi/2$ away from P1). Then, defining
\begin{equation}
    \cos \epsilon \equiv \uv{l}_{\rm o} \cdot \uv{l}_{\rm s},
\end{equation}
we can write
\begin{align}
    \uv{l}_{\rm o} &= \cos \frac{\epsilon}{2} \uv{z} + \sin
        \frac{\epsilon}{2}\uv{x},\\
    \uv{l}_{\rm s} &= \cos \frac{\epsilon}{2} \uv{z} - \sin
        \frac{\epsilon}{2}\uv{x}.
\end{align}
Finally, upon inspection, $\uv{x}$ is also an equilibrium point, which must be
P3. \textbf{In summary, in this reference frame, P1 lies along $\uv{z}$, P2 lies
along $\uv{y}$, and P3 lies along $\uv{x}$}.

To get the level curve corresponding to the separatrix, we evaluate the
Hamiltonian (factoring out the prefactor $\omega_{\rm sp} = \omega_{\rm op}$)
and adopt a spherical coordinate system:
\begin{align}
    H &\propto -\frac{1}{2} \s{
        \p{\uv{l}_{\rm p} \cdot \uv{l}_{\rm s}}^2
        + \p{\uv{l}_{\rm p} \cdot \uv{l}_{\rm o}}^2},\\
    \tilde{H}\p{\theta, \phi} &= -\s{
        \sin^2\frac{\epsilon}{2}\sin^2\theta\cos^2\phi
        + \cos^2\frac{\epsilon}{2} \cos^2\theta},
\end{align}
where we have adopted spherical coordinates $(\theta, \phi)$ to describe the
orientation of $\uv{l}_{\rm p}$, and $\theta = \pi/2, \phi = 0$ corresponds to
$\uv{x}$. We first evaluate $H$ (dropping the tilde) at P3:
\begin{equation}
    H_3 = -\sin^2\frac{\epsilon}{2},
\end{equation}
then the separatrix is given by
\begin{align}
    H\p{\theta_{\rm sep}(\phi), \phi} &= H_3,\\
    \sin^2\frac{\epsilon}{2}\p{1 - \cos^2\theta_{\rm sep}}\cos^2\phi
        + \cos^2\frac{\epsilon}{2} \cos^2\theta_{\rm sep}
        &= \sin^2\frac{\epsilon}{2},\\
    \cos^2\theta_{\rm sep} &= \frac{\sin^2\frac{\epsilon}{2}\sin^2\phi}{
            \cos^2\frac{\epsilon}{2} -
            \sin^2\frac{\epsilon}{2}\cos^2\phi}\nonumber\\
        &= \frac{\sin^2\phi}{\cot^2\frac{\epsilon}{2} - \cos^2\phi},\\
    A_{\rm sep} &= 4\int\limits_0^\pi \cos_+\theta_{\rm sep}\;\mathrm{d}\phi.
\end{align}
Here, $\cos_+\theta_{\rm sep}$ indicates that we take the positive root; one
factor of two arises because the vertical extent of the separatrix is
$\cos_+\theta_{\rm sep} - \cos_- \theta_{\rm sep}$, and a second factor of two
arises because we are only integrating $\phi \in [0, \pi]$. Under this
convention, the maximum possible phase space area is $4\pi$. We display the
value of $A_{\rm sep}$ in Fig.~\ref{fig:laplace}.
\begin{figure}
    \centering
    \includegraphics[width=0.5\columnwidth]{laplace.png}
    \caption{Fractional phase space area enclosed by the maximal separatrix as a
    function of $\epsilon$. Reminder: this is the area surrounding Laplace
    equilibrium P2 when $r_{\rm M} = a$, which is also the maximum extent of the
    separatrix.}\label{fig:laplace}
\end{figure}

Note: the integral for $A_{\rm sep}$ is analytic:
\begin{align}
    A_{\rm sep} &= 4\int\limits_0^\pi
            \frac{\sin\phi}{\sqrt{\cot^2\frac{\epsilon}{2} - \cos^2\phi}}
            \;\mathrm{d}\phi\nonumber\\
        &= 4\int\limits_{-1}^1
            \frac{1}{\sqrt{\cot^2\frac{\epsilon}{2} - \cos^2\phi}}
            \;\mathrm{d}\cos \phi\nonumber\\
        &= 4\s{\tan^{-1}\p{\frac{u}{
            \sqrt{\cot^2\frac{\epsilon}{2} - u^2}}}}_{u=-1}^{u=1}\nonumber\\
        &= 8\s{\tan^{-1}\sqrt{\frac{\sin^2(\epsilon / 2)}{\cos \epsilon}}}.
\end{align}

\subsection{Maximum Separatrix Area---Melaine}

Melaine says that the separatrix is given by the solutions to the equation
($I_Q$ is the satellite inclination to the planet equator, and $\delta Q$ is the
corresponding phase angle; we call these $\theta,\phi$ above)
\begin{align}
    \tan I_{\rm Q,\pm}
        &= \frac{\cos \delta Q \sin (2\epsilon) \pm \sin \delta Q
            \sin \epsilon\sqrt{2\p{u - 1 + \sqrt{1 + u^2 + 2u\cos\p{2\epsilon}}}}}{
            u + 1 - 2\cos^2 \delta Q \sin^2\epsilon - \sqrt{1 + u^2 +
            2u\cos(2\epsilon)}}.
\end{align}
Here $u = r_{\rm M}^5/a^5$.
However, this expression is singular for $\cos \delta_Q = w_{\pm}$, where
\begin{equation}
    w_{\pm} = \pm \sqrt{\frac{1 + u - \sqrt{1 + u^2 +
        2u\cos(2\epsilon)}}{2\sin^2\epsilon}}.
\end{equation}
This is where the denominator vanishes. Note that this must be a
removable/coordinate singularity: physically, there is no special value of
$\delta_Q$. The separatrix area is then just given by integrating $\cos I_Q$ but
taking the correctly-signed roots, which Melaine works out to be
\begin{align}
    \frac{A}{2} ={}&
            \int\limits_0^\pi \cos I_{\rm Q, +} - \cos I_{\rm Q,
            -}\;\mathrm{d}\delta_Q,\\
        ={}&
        \int\limits_0^{\arccos w_+}
            \p{\frac{-1}{\sqrt{1 + x_+^2}}
                - \frac{-1}{1 + x_-^2}}\;\mathrm{d}\delta_Q\nonumber\\
        &+ \int\limits_{\arccos w_+}^{\arccos w_-}
            \p{\frac{1}{\sqrt{1 + x_+^2}}
                - \frac{-1}{1 + x_-^2}}\;\mathrm{d}\delta_Q\nonumber\\
        &+ \int\limits_{\arccos w_-}^{\pi}
            \p{\frac{1}{\sqrt{1 + x_+^2}}
                - \frac{1}{1 + x_-^2}}\;\mathrm{d}\delta_Q.
\end{align}

We find that the two expressions agree, see Fig.~\ref{fig:laplace}. Is it
obvious that they should? Setting $u = 1$, we find that
\begin{align}
    \tan I_{\rm Q, \pm} &=
        \frac{\cos \delta Q \sin (2\epsilon) \pm \sin \delta Q
            \sin \epsilon\sqrt{4\cos \epsilon}}{
            2 - 2\cos^2 \delta Q \sin^2\epsilon - 2\cos \epsilon}\nonumber\\
        &=
        \frac{\cos \delta Q \sin (2\epsilon) \pm \sin \delta Q
            \sin \epsilon\sqrt{4\cos \epsilon}}{
            2 - 2\cos^2 \delta Q \sin^2\epsilon - 2\cos \epsilon}.
\end{align}
Not obvious.

\end{document}

