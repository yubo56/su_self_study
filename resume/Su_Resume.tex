%%%%%%%%%%%%%%%%%%%%%%%%%%%%%%%%%%%%%%%
% Deedy - One Page Two Column Resume
% LaTeX Template
% Version 1.1 (30/4/2014)
%
% Original author:
% Debarghya Das (http://debarghyadas.com)
%
% New author:
% Yubo Su (ys835@cornell.edu)
%
% Original repository:
% https://github.com/deedydas/Deedy-Resume
%
% IMPORTANT: THIS TEMPLATE NEEDS TO BE COMPILED WITH XeLaTeX
%
% This template uses several fonts not included with Windows/Linux by
% default. If you get compilation errors saying a font is missing, find the line
% on which the font is used and either change it to a font included with your
% operating system or comment the line out to use the default font.
%%%%%%%%%%%%%%%%%%%%%%%%%%%%%%%%%%%%%%

\newif\ifisCS
% true by default, makefile will set to false to compile physics resume
\isCStrue
\documentclass[]{yubo-resume-openfont}

\begin{document}

\newcommand{\tab}{\hspace{10pt}}
\renewcommand{\emph}[1]{
    {\fontspec[Path = fonts/lato/]{Lato-LigIta}\selectfont #1}
}

%%%%%%%%%%%%%%%%%%%%%%%%%%%%%%%%%%%%%%
%
%     LAST UPDATED DATE
%
%%%%%%%%%%%%%%%%%%%%%%%%%%%%%%%%%%%%%%
\lastupdated

%%%%%%%%%%%%%%%%%%%%%%%%%%%%%%%%%%%%%%
%
%     TITLE NAME
%
%%%%%%%%%%%%%%%%%%%%%%%%%%%%%%%%%%%%%%

\namesection{Yubo}{Su}{\href{mailto:ys835@cornell.edu}{ys835@cornell.edu}
    $\bullet$ 770.527.2575\\
    \urlstyle{same}{http://www.linkedin.com/in/yubosu} $\bullet$
    \urlstyle{same}{https://github.com/yubo56}
}

%%%%%%%%%%%%%%%%%%%%%%%%%%%%%%%%%%%%%%
%
%     COLUMN ONE
%
%%%%%%%%%%%%%%%%%%%%%%%%%%%%%%%%%%%%%%

\begin{minipage}[t]{0.33\textwidth}

%%%%%%%%%%%%%%%%%%%%%%%%%%%%%%%%%%%%%%
%     EDUCATION
%%%%%%%%%%%%%%%%%%%%%%%%%%%%%%%%%%%%%%

\section{Education}
    \subsection[Cornell]{Cornell University}
    \descript{Ph.D.~Astrophysics}
    \location{Aug 2017--Present \\
    Ithaca, NY}
    \sectionsep
    \subsection[CIT]{California Institute\\
        of Technology}
    \descript{B.S.~in Physics, \\
        \tab Computer Science}
    \location{Oct 2012--Jun 2016 \\
        Pasadena, CA | GPA:\@ 3.74}
\sectionsep

%%%%%%%%%%%%%%%%%%%%%%%%%%%%%%%%%%%%%%
%     SKILLS
%%%%%%%%%%%%%%%%%%%%%%%%%%%%%%%%%%%%%%

\def\ProgrammingSkills{
    \subsection{Programming}
    Javascript (Node.js) $\bullet$ Python $\bullet$ C/C++\\
    Java $\bullet$ Shell $\bullet$ CUDA $\bullet$ Assembly\\
}
\def\Skillset{
    \subsection{Skillset}
    Numerical Simulation\\
    Systems Infrastructure \&\\
        \tab Optimization\\
    Data Management \& Security
}

\section{Skills}
    \ifisCS
        \ProgrammingSkills
        \sectionsep
        \Skillset
    \else
        \Skillset
        \sectionsep
        \ProgrammingSkills
    \fi
    \sectionsep
    \subsection{Tools}
    Matlab $\bullet$ Mathematica $\bullet$ \LaTeX\\
    MongoDB $\bullet$ PostgreSQL \\
    AWS (EC2, S3, etc.) $\bullet$ Docker\\
    Ansible $\bullet$ Jenkins $\bullet$ Protractor\\
    Git $\bullet$ Linux\\
    \sectionsep
    \subsection{Languages}
    English $\bullet$ Chinese $\bullet$ French
    \sectionsep

%%%%%%%%%%%%%%%%%%%%%%%%%%%%%%%%%%%%%%
%     LINKS
%%%%%%%%%%%%%%%%%%%%%%%%%%%%%%%%%%%%%%

% (ys) not using

%%%%%%%%%%%%%%%%%%%%%%%%%%%%%%%%%%%%%%
%     COURSEWORK
%%%%%%%%%%%%%%%%%%%%%%%%%%%%%%%%%%%%%%

\def\PhWork{
    \subsection{Physics}
        Astrophysical Processes\\
        Advanced Plasma Physics\\
        Computational Physics\\
        Introduction to Particle Physics\\
        Introduction to Solid State Physics\\
}
\def\CSWork{
    \subsection{Computer Science}
        Machine Learning\\
        GPU Programming\\
        Networks and Economics\\
        Relational Databases
}

\section{Coursework}
    \ifisCS
        \PhWork
        \sectionsep
        \CSWork
    \else
        \CSWork
        \sectionsep
        \PhWork
    \fi
    \sectionsep

    \subsection{Teaching}
        Differential Equations\\
        Complex Analysis\\
        C++ Language Workshop
    \sectionsep

%%%%%%%%%%%%%%%%%%%%%%%%%%%%%%%%%%%%%%
%
%     COLUMN TWO
%
%%%%%%%%%%%%%%%%%%%%%%%%%%%%%%%%%%%%%%

\end{minipage}
\hfill
\begin{minipage}[t]{0.66\textwidth}

%%%%%%%%%%%%%%%%%%%%%%%%%%%%%%%%%%%%%%
%     RESEARCH
%%%%%%%%%%%%%%%%%%%%%%%%%%%%%%%%%%%%%%

\def\ResearchSec{
    \section{Research}

    \subsection{Cornell University}
    \descript{Graduate Research Assistant}
    \location{Aug 2017--Present | Pasadena, CA}
    \ifisCS
    \else
        \vspace{\topsep} % Hacky fix for awkward extra vertical space
    \fi
    \begin{tightemize}
        \item Working with \textbf{\href
               {http://astro.cornell.edu/members/dong-lai.html}
               {Prof.\ Dong Lai}
            } to explore numerically energy and angular momentum redistribution
            by nonlinear wave breaking of internal tidal excitations in white
            dwarfs.
        \item Working with Sr.\ Research Associate Henrik Spoon on a webpage to
            disseminate diagnostics for \emph{The Infrared Database of
            Extragalactic Observables from Spitzer}, to go live at \url{
                http://ideos.astro.cornell.edu
            }.
        \item \emph{High performance computing, numerical fluid dynamics,
            theoretical astrophysics}.
    \end{tightemize}
    \sectionsep

    \subsection{California Institute of Technology}
    \descript{Undergraduate Research Assistant}
    \location{Jan 2015--Jun 2016 | Pasadana, CA}
    \begin{tightemize}
        \item Worked with \textbf{\href
                {http://www.astro.caltech.edu/~golwala/}
                {Prof.\ Sunil Golwala}
            } to quantify detectability of kinetic Sunyaev-Zel'dovich Effect with
            future sub-millimeter telescopes.
        \item Used Monte Carlo simulation to estimate nonlinear kSZ detection
            uncertainties due to imperfect source subtraction.
        \item Code at
            \url{https://github.com/yubo56/Bolocam_Source_Subtraction}.
        \item \emph{Signal Processing, IDL, Linux}.
    \end{tightemize}
    \sectionsep

    \subsection{NASA Jet Propulsion Laboratory}
    \descript{Undergraduate Research Assistant}
    \location{Jun 2014--Dec 2014 | Pasadena, CA}
    \begin{tightemize}
        \item Worked with \textbf{\href
                {https://science.jpl.nasa.gov/people/Liewer/}
                {Dr.\ Paulett Liewer}
            } to generate synthetic white light images for solar phenomena
            simulating Solar Probe Plus (exp.~2020) view parameters.
        \item AGU 2014---
            {\small
                \url{https://agu.confex.com/agu/fm14/webprogram/Paper18882.html}
            }
        \item \emph{Raytracing, IDL, C}.
    \end{tightemize}
    \sectionsep
}

%%%%%%%%%%%%%%%%%%%%%%%%%%%%%%%%%%%%%%
%     EXPERIENCE
%%%%%%%%%%%%%%%%%%%%%%%%%%%%%%%%%%%%%%

\def\ExperienceSec{
    \section{Experience}

    \runsubsection{Blend Labs}
    \descript{| Software Engineer}
    \ifisCS
        \location{July 2016--Present | San Francisco, CA}
        \vspace{\topsep} % Hacky fix for awkward extra vertical space
    \else
        \location{July 2016--Aug 2017 | San Francisco, CA}
    \fi
    \begin{tightemize}
        \item Developed AWS S3 file management microservice. Implemented
            per-file encryption, set up load testing suite and stabilized all
            microservice deploys.
        \item Profiled and optimized test suites and app deploy by parallelizing
            tests, improving build caching and decreasing app size. Average
            speed up of 3x.
        \item Developed internal SDK to simplify encoding user transition
            business logic.
        \item Stabilized unit and end-to-end tests, reducing failures by 3x to
            99\%+ stability.
        \item \emph{Node.js, Angular, Mongo, Python, Docker, Shell, Ansible,
            AWS}.
    \end{tightemize}
}

\ifisCS
    \ExperienceSec
    \ResearchSec
\else
    \ResearchSec
    \ExperienceSec
\fi

%%%%%%%%%%%%%%%%%%%%%%%%%%%%%%%%%%%%%%
%     AWARDS
%%%%%%%%%%%%%%%%%%%%%%%%%%%%%%%%%%%%%%

\section{Awards}
\subsection{California Institute of Technology}
\begin{tabular}{p{20pt}p{80pt}p{7.5cm}}
    2016 & Best TA---Teaching Feedback & Among all Caltech Undergraduate and
    Graduate TAs. 22/24 students who responded gave perfect reviews in all
    categories.\\
    2016 & Outstanding Teaching Award & Nominated by students among teachers and
    TAs, selected by student body.\\
    2015 & NSF GRFP Honorable Mention & Proposed to study core-collapse
    supernovae gravitational waves using machine learning techinques.
\end{tabular}
\sectionsep

\end{minipage}
\end{document}
