     \documentclass[xcolor=dvipsnames, 9pt]{beamer}
     \usepackage{fancyhdr, amsmath, amsthm, amssymb, mathtools, lastpage,
     hyperref, enumerate, graphicx, setspace, wasysym, upgreek, listings, times}
     \usetheme{Madrid}
     \usefonttheme{professionalfonts}
     \newcommand{\scinot}[2]{#1\times10^{#2}}
     \newcommand{\bra}[1]{\left<#1\right|}
     \newcommand{\ket}[1]{\left|#1\right>}
     \newcommand{\dotp}[2]{\left<#1\,\middle|\,#2\right>}
     \newcommand{\rd}[2]{\frac{\mathrm{d}#1}{\mathrm{d}#2}}
     \newcommand{\pd}[2]{\frac{\partial#1}{\partial#2}}
     \newcommand{\rtd}[2]{\frac{\mathrm{d}^2#1}{\mathrm{d}#2^2}}
     \newcommand{\ptd}[2]{\frac{\partial^2 #1}{\partial#2^2}}
     \newcommand{\norm}[1]{\left|\left|#1\right|\right|}
     \newcommand{\abs}[1]{\left|#1\right|}
     \newcommand{\pvec}[1]{\vec{#1}^{\,\prime}}
     \newcommand{\svec}[1]{\vec{#1}\;\!}
     \newcommand{\bm}[1]{\boldsymbol{\mathbf{#1}}}
     \let\Re\undefined
     \let\Im\undefined
     \newcommand{\ang}[0]{\text{\AA}}
     \newcommand{\mum}[0]{\upmu \mathrm{m}}
     \DeclareMathOperator{\Res}{Res}
     \DeclareMathOperator{\Re}{Re}
     \DeclareMathOperator{\Im}{Im}
     \DeclareMathOperator{\Log}{Log}
     \DeclareMathOperator{\Arg}{Arg}
     \DeclareMathOperator{\Tr}{Tr}
     \DeclareMathOperator{\E}{E}
     \DeclareMathOperator{\Var}{Var}
     \DeclareMathOperator*{\argmin}{argmin}
     \DeclareMathOperator*{\argmax}{argmax}
     \DeclareMathOperator{\sgn}{sgn}
     \DeclareMathOperator{\diag}{diag}
     \newcommand{\expvalue}[1]{\left<#1\right>}
     \usepackage[labelfont=bf, font=scriptsize]{caption}\usepackage{tikz}
     \usepackage[font=scriptsize]{subcaption}
     \everymath{\displaystyle}
     \lstset{basicstyle=\ttfamily\footnotesize,frame=single,numbers=left}
\tikzstyle{circ} = [draw, circle, fill=white, node distance=3cm, minimum
height=2em]

\title[MongoDB World 2017]{Blend in Chicago: MongoDB World 2017}
% \subtitle[]{Subtitle}
\author[Y. Su]{Yubo Su}
\institute[Blend]{Blend}
\date{\today}
\logo{%
    \makebox[0.95\paperwidth]{%
        \includegraphics[width=0.15\textwidth]{media/mongo.png}
        \hfill
        \includegraphics[width=0.15\textwidth]{media/blend.png}
    }
}
\setbeamertemplate{caption}{\insertcaption\par}

\begin{document}

\frame{\titlepage}

\begin{frame}
    \frametitle{Morning Keynotes}
    \framesubtitle{06/20/17 0900--1030}

    \begin{itemize}
        \item Tom Schenk, Chief data officer, Chicago. \emph{WindyGrid}.
            \begin{itemize}
                \item Track colocated data, 911 calls to Tweets to
                    weather.
                \item Flexible schema: \texttt{\{what, when, where\}}
                \item Predictive analytics (example, where to send food
                    inspector) using visualization of multiple causal layers.
            \end{itemize}
        \item Dev Ittycheria, CEO MongoDB
            \begin{itemize}
                \item 2007 is watershed year, AWS, iPhone, Android, and
                    many others.
                \item Argue b/c storage costs dropped below a critical
                    point.
                \item MongoDB also in 2007: document model, distributed
                    systems + aggregation.
            \end{itemize}
    \end{itemize}
\end{frame}

\begin{frame}
    \frametitle{Morning Keynotes}
    \framesubtitle{06/20/17 0900--1030}
    \begin{itemize}
        \item Eliot Horowitz, CTO, MongoDB
            \begin{itemize}
                \item 3.6 ships November, already on Github.
                \item MongoDB Charts (3.6)
                    \begin{itemize}
                        \item Business Intelligence: BI Connector is SQL
                            interface.
                        \item Coercing data to table is difficult:
                            polymorphic schemas, arrays.
                        \item Solution: \emph{MongoDB Charts}! Data
                            visualization tool, handles above.
                    \end{itemize}

                \item 3.6 document model features:
                    \begin{itemize}
                        \item \texttt{\$lookup} takes sub-pipelines!
                        \item \texttt{\$update} can operate on arrays
                            natively! Takes a filter over array entries,
                            can iterate over nested.
                        \item JSON Schemas.
                    \end{itemize}

                \item 3.6 distributed systems:
                    \begin{itemize}
                        \item Native retryable writes
                        \item \emph{Change Streams} can get a stream of changes
                            to a db.
                    \end{itemize}
            \end{itemize}
    \end{itemize}
\end{frame}

\begin{frame}
    \frametitle{Morning Keynotes}
    \framesubtitle{06/20/17 0900--1030}
    \begin{itemize}
        \item Eliot Horowitz, CTO, MongoDB (continued)
            \begin{itemize}
                \item Mongo Atlas
                    \begin{itemize}
                        \item ``Should be irrespensible to run MongoDB in cloud
                            w/o Atlas''
                        \item Built in security, one-click spin up, built in
                            scaling elasticity.
                        \item Data browser + performance viewer in UI
                            (utilization stats, examine queries as stream,
                            explore data),
                        \item Live migration service (not very live in demo,
                            requires downtime for mirror to catch up and change
                            source of truth).
                        \item Now with MS Azure + Google Cloud support too (+
                            AWS). \\[9pt]
                        \item Performance Adviser.
                        \item CRUD support in data browser.
                        \item Charts!
                        \item LDAP Auth.
                        \item Cross-region, cross-cloud!
                    \end{itemize}
                \item MongoDB Stich (Beta as of today in Atlas, 06/20/17)
                    \begin{itemize}
                        \item ``Backend as a service''
                        \item REST API for MongoDB
                        \item Configuration-based auth/security
                        \item Service composition to govern how services talk to
                            each other.
                    \end{itemize}
            \end{itemize}
    \end{itemize}
\end{frame}

\begin{frame}
    \frametitle{Squeezing the Most out of Your Document Model}
    \framesubtitle{%
        06/20/17 1050--1130:
        Norberto Leite, Lead Curriculum Engineer, MongoDB
    }
    \begin{columns}
        \begin{column}{0.5\textwidth}
            \begin{itemize}
                \item Nested schema, spectrum of highly normalized or denormed storage.
                    \begin{itemize}
                        \item Normalized requires foreign keys, requires looking
                            into many collections.
                        \item Denorm is simpler query, complex schema.
                    \end{itemize}
                \item Consider three possible behaviors:
                    \begin{itemize}
                        \item Get player: Denorm outperforms.
                        \item Add new field to doc: either add new collection
                            or modify every doc, the same.
                        \item Change existing field: If a highly shared field,
                            normalized is very fast.
                    \end{itemize}
            \end{itemize}
        \end{column}
        \begin{column}{0.5\textwidth}
            \begin{itemize}
                \item Optimizing highly normalized:
                    \begin{itemize}
                        \item Can optimize with aggregate, but more importantly
                            \texttt{db.createView()}.
                            \begin{itemize}
                                \item Views are basically stored aggregates.
                                \item Better \texttt{\$project} support.
                            \end{itemize}
                        \item Also consider, if reading much more than writing,
                            should store calculated fields!
                    \end{itemize}
                \item Optimizing denormed:
                    \begin{itemize}
                        \item Should normalize fields that are infrequently
                            updated.
                    \end{itemize}
                \item \texttt{tl;dr} normalized have fast write, slow reads.
                    Should embed everything that is infrequently updated.
            \end{itemize}
        \end{column}
    \end{columns}
\end{frame}

\begin{frame}
    \frametitle{Advanced Schema Design Patterns}
    \framesubtitle{%
        06/20/17 1140--1220:
        Daniel Coupal, Senior Curriculum Engineer, MongoDB
    }
    \begin{columns}
        \begin{column}{0.5\textwidth}
            \begin{itemize}
                \item Axiom: data models maximize performance + scalability
                    despite latency, costs, hardware.
                \item Common issues \#1, too many optional fields:
                    \begin{itemize}
                        \item Use attribute array, \texttt{[\{key: keyName, value\}]}.
                        \item Accommodates optional fields.
                    \end{itemize}
                \item Common issues \#2, working set does not fit in RAM.%
                    \begin{itemize}
                        \item Can subset, truncate data
                        \item Probably also useful for showing users too.
                    \end{itemize}
                \item Common issues \#3, data consistency.
                    \begin{itemize}
                        \item Accept instantaneous inconsistency, duplicate at
                            regular intervals \frownie.
                    \end{itemize}
            \end{itemize}
        \end{column}
        \begin{column}{0.5\textwidth}
            \begin{itemize}
                \item Common issues \#4, repeated computations
                    \begin{itemize}
                        \item Reads generally outnumber writes, apply
                            computation on write.
                    \end{itemize}
                \item Common issues \#5, expensive tracking
                    \begin{itemize}
                        \item e.g.\ expensive to increment on every page view
                        \item Solution: random number in range $[1, N]$,
                            increment by $N$.
                    \end{itemize}
                \item Common issues \#6, large data easily overflow
                    \begin{itemize}
                        \item Bucket, store buckets into a separate
                            collection.
                    \end{itemize}
            \end{itemize}
        \end{column}
    \end{columns}
\end{frame}

\begin{frame}
    \frametitle{Powering Microservices with Docker, Kubernetes, Kafka and MongoDB}
    \framesubtitle{%
        06/20/17 1350--1430:
        Andrew Morgan, Product Marketing, MongoDB
    }
    \begin{itemize}
        \item Microservices vs.\ monolith, preferable b/c web scale,
            faster iteration, compartmentalized.
        \item One common rule of thumb is that one developer can own the
            whole thing, a couple hundred lines, but not everybody
        \item Hard metal vs.\ Docker (Kubernetes) vs.\ Atlas.
        \item Kafka can run general events while Mongo streams (the new feature)
            only handles database updates.
    \end{itemize}
\end{frame}

\end{document}
