    \documentclass[11pt,
        usenames, % allows access to some tikz colors
        dvipsnames % more colors: https://en.wikibooks.org/wiki/LaTeX/Colors
    ]{article}
    \usepackage{
        amsmath,
        amssymb,
        fouriernc, % fourier font w/ new century book
        fancyhdr, % page styling
        lastpage, % footer fanciness
        hyperref, % various links
        setspace, % line spacing
        amsthm, % newtheorem and proof environment
        mathtools, % \Aboxed for boxing inside aligns, among others
        float, % Allow [H] figure env alignment
        enumerate, % Allow custom enumerate numbering
        graphicx, % allow includegraphics with more filetypes
        wasysym, % \smiley!
        upgreek, % \upmu for \mum macro
        listings, % writing TrueType fonts and including code prettily
        tikz, % drawing things
        booktabs, % \bottomrule instead of hline apparently
        cancel % can cancel things out!
    }
    \usepackage[margin=1in]{geometry} % page geometry
    \usepackage[
        labelfont=bf, % caption names are labeled in bold
        font=scriptsize % smaller font for captions
    ]{caption}
    \usepackage[font=scriptsize]{subcaption} % subfigures

    \newcommand*{\scinot}[2]{#1\times10^{#2}}
    \newcommand*{\dotp}[2]{\left<#1\,\middle|\,#2\right>}
    \newcommand*{\rd}[2]{\frac{\mathrm{d}#1}{\mathrm{d}#2}}
    \newcommand*{\pd}[2]{\frac{\partial#1}{\partial#2}}
    \newcommand*{\rtd}[2]{\frac{\mathrm{d}^2#1}{\mathrm{d}#2^2}}
    \newcommand*{\ptd}[2]{\frac{\partial^2 #1}{\partial#2^2}}
    \newcommand*{\md}[2]{\frac{\mathrm{D}#1}{\mathrm{D}#2}}
    \newcommand*{\pvec}[1]{\vec{#1}^{\,\prime}}
    \newcommand*{\svec}[1]{\vec{#1}\;\!}
    \newcommand*{\bm}[1]{\boldsymbol{\mathbf{#1}}}
    \newcommand*{\ang}[0]{\;\text{\AA}}
    \newcommand*{\mum}[0]{\;\upmu \mathrm{m}}
    \newcommand*{\at}[1]{\left.#1\right|}

    \newtheorem{theorem}{Theorem}[section]

    \let\Re\undefined
    \let\Im\undefined
    \DeclareMathOperator{\Res}{Res}
    \DeclareMathOperator{\Re}{Re}
    \DeclareMathOperator{\Im}{Im}
    \DeclareMathOperator{\Log}{Log}
    \DeclareMathOperator{\Arg}{Arg}
    \DeclareMathOperator{\Tr}{Tr}
    \DeclareMathOperator{\E}{E}
    \DeclareMathOperator{\Var}{Var}
    \DeclareMathOperator*{\argmin}{argmin}
    \DeclareMathOperator*{\argmax}{argmax}
    \DeclareMathOperator{\sgn}{sgn}
    \DeclareMathOperator{\diag}{diag\;}

    \DeclarePairedDelimiter\bra{\langle}{\rvert}
    \DeclarePairedDelimiter\ket{\lvert}{\rangle}
    \DeclarePairedDelimiter\abs{\lvert}{\rvert}
    \DeclarePairedDelimiter\ev{\langle}{\rangle}
    \DeclarePairedDelimiter\p{\lparen}{\rparen}
    \DeclarePairedDelimiter\s{\lbrack}{\rbrack}
    \DeclarePairedDelimiter\z{\lbrace}{\rbrace}

    % \everymath{\displaystyle} % biggify limits of inline sums and integrals
    \tikzstyle{circ} % usage: \node[circ, placement] (label) {text};
        = [draw, circle, fill=white, node distance=3cm, minimum height=2em]
    \definecolor{commentgreen}{rgb}{0,0.6,0}
    \lstset{
        basicstyle=\ttfamily\footnotesize,
        frame=single,
        numbers=left,
        showstringspaces=false,
        keywordstyle=\color{blue},
        stringstyle=\color{purple},
        commentstyle=\color{commentgreen},
        morecomment=[l][\color{magenta}]{\#}
    }

\begin{document}

\def\Snospace~{\S{}} % hack to remove the space left after autorefs
\renewcommand*{\sectionautorefname}{\Snospace}
\renewcommand*{\appendixautorefname}{\Snospace}
\renewcommand*{\figureautorefname}{Fig.}
\renewcommand*{\equationautorefname}{Eq.}
\renewcommand*{\tableautorefname}{Tab.}

\onehalfspacing

\pagestyle{fancy}
\rfoot{Yubo Su}
\rhead{}
\cfoot{\thepage/\pageref{LastPage}}

\title{Q Exam Notes}
\author{Yubo Su}

\maketitle

\section{PHYS 7680: Computational Physics}

Not much to review here, we know most of it or it's too low level

\begin{itemize}
    \item CFL conditions, von Neumann Analysis.
    \item Spectral methods (exponential conv, low numerical viscosity).
    \item What is numerical viscosity, how does it arise?
    \item Convergence of various numerical schemes, higher order than RK4.
    \item Runge Phenomenon vs Gibbs Phenomenon, non-sinusoidal expansions.
\end{itemize}

\section{ASTRO 6578: Planet Formation}

\begin{itemize}
    \item Protostellar disks: Coolers stars have more complex lines, young stars
        don't support planets. OBAFGKM (Oh be a fine gal, kiss me) stars.
    \item Lol: Peak power generation of the core $\ll$ human power generation
        (recall Eddington luminosity as well from Ph101).
    \item Disk formation must be driven by eddy viscosity in order for the only
        timescale $t = \frac{L^2}{\nu}$ to be less than a Hubble time.
    \item Shakura-Sunyaev discs $\nu = \alpha c_s H$ viscosity parameterization.
    \item In a disc w/ gas and dust, gas orbits sub-Keplerian if polytrope $P
        \propto r^{-n}$ profile.

    \item Planet formation: Isochron dating.
    \item Gravitational focusing $\propto R^2/R^4$ for small/large objects,
        favors runaway accretion.
    \item Pebble accretion for gas giants, ice line means smaller inner planets.
        Dynamical friction speeds up smaller bodies and counteracts
        gravitational focusing, preventing runaway accretion and inducing
        oligarchal growth but no big planets. Pebble accretion solves because
        pebbles ($\sim 10\;\mathrm{cm}$) feel gas drag, further magnifying
        gravitational focusing.
    \item Minimum Mass Solar Nebula constructed by replenishing all planets up
        to solar abundances (makes up for photoevaporation, though not ejecta
        from early Sun).

    \item Planetary migration: Grand Tack model uses Jupiter/Saturn resonances
        to form small Mars. Also produces \emph{Late Heavy Bombardment}.
    \item Lindblad resonances spaced unevenly on either side of planet, Type I
        migration until planet opens a permanent gap immune to disc viscosity
        replenishing (Saturn-sized).

    \item Giant planet interiors: molecular hydrogen envelope, metallic hydrogen
        mantle and diffuse core in Jupiter.
\end{itemize}

\section{AEP 6060: Introduction to Plasma Physics}

\begin{itemize}
    \item Electron plasma frequency $\omega_{pe} =
        \sqrt{\frac{n_0e^2}{\varepsilon_0m_e}}$ (derivation: oscillating box of
        charges).
    \item Debye length is bottom end for plasmas.
    \item $\rd{f}{t} = 0$ Vlasov equation/kinetic theory, produces fluid
        equations.
    \item The world is Maxwellian $f(\vec{v}) \propto e^{-\p*{\vec{v} -
        \vec{v}_0}^2}$ or $f(v) \propto v^2e^{-(v - v_0)^2}$.
    \item Alfv\'en speed $v_A = B/\sqrt{\mu_0\rho}$. Two-fluid yields more waves
        than MHD unless introduce thermal/pressure terms.

    \item Instabilities: streaming instability, Landau damping.
    \begin{itemize}
        \item Plane wave ansatz gets dispersion relation of form $1 + \chi_e =
            0$ where $\chi_e$ the electron susceptibility consists of an
            integral over a singular value.

        \item Careful IVP analysis + Landau prescription deforming contour above
            the singularity in $\chi_e$ shows that the correct way to interpret
            the dispersion relation is to deform the integral (equivalent).
    \end{itemize}

    \item Infinite cold plasma waves (instead of permitting pressure/temp terms,
        use source term in Amp\`ere's Law, dielectric tensor).

    \item Geometric optics for waves, magnetic reconnection.
\end{itemize}

\section{ASTRO 6531: Astrophysical Fluids}

\begin{itemize}
    \item Boundi radius (1D accretion), Bondi-Lyttleton Accretion (moving BH
        accretion) for \emph{barotropic flow} $P \propto \rho^\gamma$.

    \item Waves in self-gravitating fluid have $\omega^2 = a^2k^2 - 4\pi G
        \rho_0$. Extra term is self gravitation, \emph{Jean's Mass}.

    \item Surface gravity waves (deep vs shallow). Coarse treatment of
        steepening, just ``increased height makes faster velocity.''

    \item Atmospheric waves, $N^2 > 0$ is \emph{Schwarzchild criterion}.
        Nonradial spherical atmospheric waves = my research.

    \item Sommerfield quantization condition for modes in irregular cavities.

    \item Kelvin-Helmholtz Instability: two fluid streaming. Extension: shear
        flow instability, Richardson criterion.

    \item Rotating flows, Rossby waves, can source Chandrasekar-Friedman-Schutz
        Instability (major GW source for quadrupole moment).

    \item Viscous flows, high/low Re drag, Stokes flow. Boundary layers, Ekman
        layer in rotating flow.

    \item Kolmogorov turbulence, similarity.

    \item Discs: Shakura-Sunyaev discs, radial-epicyclic frequencty. Toomre $Q$
        stability, Lindblad resonance, corotation resonance.

    \item Shocks, Rankine-Huginot conditions continuity. Blast waves,
        Sedov-Taylor = radial shocks and self-similarity.

    \item MHD, Alfv\'en speed, magnetorotational instability.
\end{itemize}

\section{PHYS 6562: Statistal Mechanics}

\begin{itemize}
    \item The diffusion equation, probability current.

    \item Microcanonical Ensemble: ensemble of all states at fixed energy $E$.

    \item Entropy is uncertainty, log number of accessible states. Is a
        thermodynamic potential like chemical potential (derivatives of thermo
        potentials = observables), $T^{-1} = \pd{S}{E}, \frac{\mu}{T} =
        -\pd{S}{N}$.

    \item Bunch of free energy stuff. Bosons (condensation)/Fermions (Fermi
        sea).

    \item Markov chains (detailed balance).

    \item Ising model, phase changes (discontinuities in order parameters).

    \item Correlation functions show power laws in continuous phase transitions.

    \item Abrupt transitions require often some small activation energy past the
        critical value.

    \item Scale invariance, renormalization.
\end{itemize}

\section{ASTRO 6516: Galaxies}

\begin{itemize}
    \item Milky Way: $10\;\mathrm{kpc} \times 0.5\;\mathrm{kpc}$,
        $\sim 10^{11}L_{\odot}$. Rotation curve flattens $220\;\mathrm{km/s}$
        (implies mass is not dominated by SMBH at center). Galaxy spacing $\sim
        \;\mathrm{Mpc}$.

    \item \emph{Cold} regions = dominated by Keplerian motion, \emph{warm}
        dominated by many-body gravitational interactions.

    \item Plummer law for galactic density distribution $\rho(r) =
        \frac{\rho_0}{\p*{1 + x^2}^{5/2}}$, $x = r/r_c$, simply looks like $\Phi
        = \frac{GM}{r}\frac{1}{\sqrt{1 + x^2}}$ smoothing.

    \item Quasars are AGNs, very bright compared to rest of galaxy. Standard
        model of AGNs is that viewing angle and dustiness produce varying kinds
        of emission (broader IR lines if obscured e.g.\ DSFGs).

    \item Universe age $t = 0.3\;\mathrm{s}\p*{T/\;\mathrm{MeV}}^{-2}$. Early
        universe is radiation dominated, then hydrogen formation/recombination
        when photons drop below photodissociation (photon decoupling releases
        photons as new atoms decay, CMB). Then eventually Population III stars,
        galaxies, large scale structure.
        \begin{itemize}
            \item Recombination $z = 1100$.
            \item Dark ages (no stars).
            \item Reionization $z = 20$--$6$ as stars and galaxies form,
                ionizing radiation.

            \item Only now can clusters form, lasts until present day.
        \end{itemize}

    \item Lyman-$\alpha$ forest: high redshift light passes through many
        different redshift absorbing media. Largely happens before reionization
        since much more hydrogen.

    \item Lyman-$\alpha$ emitters are very old galaxies that have mostly neutral
        hydrogen. Lyman-break galaxies have lower redshifts $z \sim 3, 4$ whose
        redshift can be pinpointed by the $912\ang$ dropoff in radiation.

    \item Metallicity tracks how many stars have already formed in the galaxy.

    \item $21\;\mathrm{cm}$ line is important since one of only transitions
        possible in neutral Hydrogen.

    \item Collisionless Boltzmann Equation is the Vlasov equation, with
        Maxwellian energy distribution allows solving density distributions in
        galaxies.
\end{itemize}

\section{MATH 6270: Applied Dynamical Systems}

\begin{itemize}
    \item Homoclinic tangles: stable and unstable manifolds intersect
        transversely!
    \item Melnikov's Method predicts when a system pertubed from a homoclinic
        orbit exhibits chaos (whether the stable + unstable manifolds cleanly
        untangle or homoclinic tangle/chaos). Can also use to predict
        topological changes in behavior.

    \item Symbolic dynamics to describe objects in the invariant set of a map.

    \item Center manifold theory! Dynamics near a center manifold. Fast-decaying
        variable is slaved to motion along the center manifold. Can predict
        e.g.\ XY model topological nature of phase transition.

    \item Normal forms: find spanning subset of a degenerate linearization about
        a fixed point.

    \item ``Find a global bifurcation by blowing up a local analysis on a
        rescaled system.'' Melnikov's Method again.

    \item Kuramoto oscillators synchronization.

    \item Fermi-Pasta-Ulam-Tsingou solitons, Kortweg de Vries.

    \item Henon-Heiles arises from studying star motion around galactic center;
        for sufficiently high level curves, Hamiltonian invariant tori break
        down into chaos. Can study via a map, though I missed this part of
        lecture.
\end{itemize}

\section{ASTRO 6530: Astrophysical Processes}

\begin{itemize}
    \item Radiative transfer equation! $\rd{I_\nu}{\tau_\nu} = -I_\nu + S_\nu$.

    \item Einstein coefficients, stimulated emission key (detailed balance).

    \item In isotropic radiation, expanding in small multipoles gives closure
        relation, plus two-stream approximation relates $I_\nu$ to $J_\nu$.

    \item Polarization: Stokes parameters!

    \item Thomson scattering is non-relativistic limit of Compton.

    \item Bremsstrahlung is when a charged particle moves through a plasma,
        bunch of scatterings means acceleration and radiation. Mostly spectrally
        flat in the R-J regime.

    \item Coherent vs incoherent radiation (groups of particles generally
        incoherent $\sim N$ unless population inversion, cascade $\sim N^2$).

    \item Cyclotron radiation when electron in magnetic field, L-W potential
        gives radiation field $\sim r^{-1}$,

    \item Synchrotron radiation: particle blips $\Delta t \propto \gamma^{-3}$,
        power laws over all frequencies.

    \item Propagation through a plasma, AEP 6060.
\end{itemize}

\end{document}

