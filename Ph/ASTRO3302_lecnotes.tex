    \documentclass[11pt,
        usenames, % allows access to some tikz colors
        dvipsnames % more colors: https://en.wikibooks.org/wiki/LaTeX/Colors
    ]{article}
    \usepackage{
        amsmath,
        amssymb,
        fouriernc, % fourier font w/ new century book
        fancyhdr, % page styling
        lastpage, % footer fanciness
        hyperref, % various links
        setspace, % line spacing
        amsthm, % newtheorem and proof environment
        mathtools, % \Aboxed for boxing inside aligns, among others
        float, % Allow [H] figure env alignment
        enumerate, % Allow custom enumerate numbering
        graphicx, % allow includegraphics with more filetypes
        wasysym, % \smiley!
        upgreek, % \upmu for \mum macro
        listings, % writing TrueType fonts and including code prettily
        tikz, % drawing things
        booktabs, % \bottomrule instead of hline apparently
        xcolor, % colored text
        cancel % can cancel things out!
    }
    \usepackage[margin=1in]{geometry} % page geometry
    \usepackage[
        labelfont=bf, % caption names are labeled in bold
        font=scriptsize % smaller font for captions
    ]{caption}
    \usepackage[font=scriptsize]{subcaption} % subfigures

    \newcommand*{\scinot}[2]{#1\times10^{#2}}
    \newcommand*{\dotp}[2]{\left<#1\,\middle|\,#2\right>}
    \newcommand*{\rd}[2]{\frac{\mathrm{d}#1}{\mathrm{d}#2}}
    \newcommand*{\pd}[2]{\frac{\partial#1}{\partial#2}}
    \newcommand*{\rdil}[2]{\mathrm{d}#1 / \mathrm{d}#2}
    \newcommand*{\pdil}[2]{\partial#1 / \partial#2}
    \newcommand*{\rtd}[2]{\frac{\mathrm{d}^2#1}{\mathrm{d}#2^2}}
    \newcommand*{\ptd}[2]{\frac{\partial^2 #1}{\partial#2^2}}
    \newcommand*{\md}[2]{\frac{\mathrm{D}#1}{\mathrm{D}#2}}
    \newcommand*{\pvec}[1]{\vec{#1}^{\,\prime}}
    \newcommand*{\svec}[1]{\vec{#1}\;\!}
    \newcommand*{\bm}[1]{\boldsymbol{\mathbf{#1}}}
    \newcommand*{\uv}[1]{\hat{\bm{#1}}}
    \newcommand*{\ang}[0]{\;\text{\AA}}
    \newcommand*{\mum}[0]{\;\upmu \mathrm{m}}
    \newcommand*{\at}[1]{\left.#1\right|}
    \newcommand*{\bra}[1]{\left<#1\right|}
    \newcommand*{\ket}[1]{\left|#1\right>}
    \newcommand*{\abs}[1]{\left|#1\right|}
    \newcommand*{\ev}[1]{\left\langle#1\right\rangle}
    \newcommand*{\p}[1]{\left(#1\right)}
    \newcommand*{\s}[1]{\left[#1\right]}
    \newcommand*{\z}[1]{\left\{#1\right\}}

    \newtheorem{theorem}{Theorem}[section]

    \let\Re\undefined
    \let\Im\undefined
    \DeclareMathOperator{\Res}{Res}
    \DeclareMathOperator{\Re}{Re}
    \DeclareMathOperator{\Im}{Im}
    \DeclareMathOperator{\Log}{Log}
    \DeclareMathOperator{\Arg}{Arg}
    \DeclareMathOperator{\Tr}{Tr}
    \DeclareMathOperator{\E}{E}
    \DeclareMathOperator{\Var}{Var}
    \DeclareMathOperator*{\argmin}{argmin}
    \DeclareMathOperator*{\argmax}{argmax}
    \DeclareMathOperator{\sgn}{sgn}
    \DeclareMathOperator{\diag}{diag\;}

    \colorlet{Corr}{red}

    % \everymath{\displaystyle} % biggify limits of inline sums and integrals
    \tikzstyle{circ} % usage: \node[circ, placement] (label) {text};
        = [draw, circle, fill=white, node distance=3cm, minimum height=2em]
    \definecolor{commentgreen}{rgb}{0,0.6,0}
    \lstset{
        basicstyle=\ttfamily\footnotesize,
        frame=single,
        numbers=left,
        showstringspaces=false,
        keywordstyle=\color{blue},
        stringstyle=\color{purple},
        commentstyle=\color{commentgreen},
        morecomment=[l][\color{magenta}]{\#}
    }

\begin{document}

\section{Mar 14, 2022---Stellar Structure on the Main Sequence}

\begin{itemize}
    \item Have learned that stars are powered by nuclear burning, but this only
        occurs in the core. What sets the structure outside of the core? Need to
        understand the energy transfer processes, including: conduction,
        convection, radiation. Last class, said conduction is unimportant.
        Today, we will talk about radiation, and possibly a little bit of
        convection. A way to rephrase our goal today is to answer: \emph{why do
        stars shine?} Ira's favorite question.

        First, stars must be mostly opaque: when we stare at the Sun, we cannot
        see the core. So let's understand what happens to radiation being
        emitted at the core, since it obviously doesn't make it to the surface.
        Understanding this will let us understand how energy gets transferred
        via radiation from the core to the surface.

    \item How does a photon interact with matter? Two ways: scattering,
        absorption.

        Scattering, Thomson scattering. If photon ``hits'' electron, then will
        change direction. But electrons are tiny, roughly point particles; how
        does it hit? EM scattering,
        \begin{equation}
            \sigma_{\rm T} \sim \scinot{6.65}{-25}\;\mathrm{cm^2}.
        \end{equation}

        Absorption/emission [of photons]: free-free emission/absorption, $e + p
        \to e + p + \gamma$ or vice versa. Bound-free/bound-bound, change energy
        levels of atom.

        Often, define an \emph{opacity}, which is just the cross-section/gram.
        For instance, for Thomson:
        \begin{align}
            \kappa_{\rm T} &= \frac{n_{\rm e}\sigma_{\rm T}}{\rho},\\
                &= \frac{\sigma_{\rm T}}{m_{\rm p}}\frac{Z}{A},\\
                &= 0.4\;\mathrm{cm^2/g}\frac{Z}{A}.
        \end{align}
        Note: independent of density and temperature! And for those curious,
        free-free and bound-free \emph{do} depend on density and temperature,
        $\kappa_{\rm ff} \propto \rho T^{-7/2}$. Thus, for higher temperatures,
        Thomson scattering is what sets opacity, and for lower temperatures,
        absorption/emission!

    \item Okay, so the picture is that photons get jostled as they radiate
        outwards from the core; how often do they interact? \emph{Mean free
        path}; we should expect $\ell \ll R_{\odot}$, else we would see the
        photons from the core (and the star would not be opaque).

        In general, think of a circle moving through a medium with number
        density $n$; how far to move before cylinder contains $\sim 1$ particle.
        $\ell \sigma_{\rm T} = 1 / n$, or $\ell = 1 / \p{\rho \kappa}$. Near the
        solar core, $\ell \sim 0.017\;\mathrm{cm}(Z/A)^2$, dominated by
        scattering. In the solar atmosphere, $\ell \sim 0.5\;\mathrm{cm}$
        (dominated by $H^-$ bound-free, $\kappa_{\rm H-}\propto \rho^{1/2}T^9$,
        it turns out, for $T \in [4000, 8000]\;\mathrm{K}$ if H- exists).

    \item Now, we know that photons travel on average $0.5\;\mathrm{cm}$ before
        encountering something in the solar atmosphere. Can we now show that the
        Sun is opaque? Photons obey a \emph{random walk} (evolve
        \emph{diffusively}). Lots of examples and analogies, \emph{drunkard's
        walk}.

        To get some intuition, let's consider the proporties of this behavior:
        you flip a coin, and H/T $\to$ L/R step by $\ell$. Then:
        \begin{itemize}
            \item What is the average displacement? $0$
            \item However, if you take a million steps, and end up exactly at
                $0$, you would be very surprised, right?

            \item Calculate mean and \emph{variance} (related to stdev).
                \begin{align}
                    \ev{\sum\limits x_i} &= 0,\\
                    \ev{(\sum\limits x_i)^2} &= N\ell^2.
                \end{align}
                Thus, central limit theorem, mean $0$ but stdev $\ell \sqrt{N}$.
        \end{itemize}
        In the Sun's case, we need $\ell \sqrt{N} = R$, setting some scale on
        the number of times a photon will bounce. Then, the time for a photon to
        exist the Sun is:
        \begin{align}
            t_{\rm diff} = N\frac{\ell}{c} = \frac{R^2}{\ell c} \sim
                10^4\;\mathrm{yr}.
        \end{align}
        This implies that if the Sun stopped fusion right now, it would take
        $10^4 \;\mathrm{yr}$ for there to be substantially fewer photons
        reaching the surface, and the Sun would take $\sim 10^4\;\mathrm{yr}$ to
        collapse!

        Let's also perform a quick estimate: if the core is cooling over
        timescales of $10^4\;\mathrm{yr}$, then what is the luminosity?
        \begin{equation}
            L \sim \frac{\p{\sigma T_{\rm c}^4}\p{4\pi R_{\rm c}^3/3}}{
                R^2 / \ell c} \sim \scinot{2}{33}\;\mathrm{erg/s}.
        \end{equation}
        In reality, $L_{\odot} \sim \scinot{4}{33}\;\mathrm{erg/s}$. This must
        be balanced by $\dot{E}_{\rm nuc} = L$.

        Interestingly, how do we know that the Sun hasn't already stopped
        fusion? \emph{Neutrinos} are emitted at the center of the Sun, and the
        Sun is transparent to these ($a \sim 10^{-44}\;\mathrm{cm^2}$,
        or $\ell \sim 10^{20}\;\mathrm{cm}$), and we can measure them.

    \item Now, we see that the structure of the stellar atmosphere is set by
        $\dot{E}_{\rm nuc}$, but $\dot{E}_{\rm nuc}$ is set by $T_{\rm c}$ which
        is set by the stellar structure. Thus, these two are both linked, so we
        can solve for them to understand the general relation:
        \begin{align}
            L &\sim \frac{a T_{\rm c}^4 (4\pi R_{\rm c}^3/3)}{R^2/\ell c}
                \propto \frac{T_{\rm c}^4 R_{\rm c}^3}{R^2 \rho \kappa}.
        \end{align}
        Recall though that $kT_{\rm c} \sim \frac{GMm_{\rm p}}{R}$ is fixed and
        set by nuclear burning, so $M \propto R$ and $L \propto M^3/\kappa$.

        For $\kappa \propto \rho^0$ (Thomson), we obtain $L \propto M^3$. For
        $\kappa \propto \rho \sim M^{-2}$ (free-free), we obtain $L \propto
        M^5$. In practice, $L \propto M^{3.5}$ is often found, but we'll use $L
        \propto M^4$.

        We can also calculate the scaling of stellar lifetimes,
        \begin{align}
            t_{\rm lifetime} &\sim \frac{0.007 M_{\rm c}c^2}{L},\\
                &\propto \frac{1}{M^3}.\\
            t_{\odot} &= \frac{0.007\p{0.1M_{\odot}}c^2}{L_{\odot}}
                = 10\;\mathrm{Gyr}.
        \end{align}

    \item Thus, we can answer how stars shine: black body radiation at the
        surface, due to [slow] radiative transfer out of the nuclear-burning
        core.

    \item Convection? TODO

    \item Today, we've covered: photon-matter interactions, how this leads to
        energy transfer out of the core (random walk), and how this leads to the
        stellar structure.

        In summary, stellar structure is set by:
        \begin{itemize}
            \item Hydrostatic equilibrium: force on each each element must
                vanish, balancing gravity and pressure: $\rdil{P}{r} =
                -Gm(r)/r^2 \times \rho(r)$.

                In particular, the central pressure $P_{\rm c} \sim GM^2/R^4$.
                Since, for an ideal gas $P_{\rm gas} = nkT$, and $P_{\rm rad} =
                aT^4/3$. For stars $\sim M_{\odot}$, they are pressure-dominated
                and $kT_{\rm c} \sim GMm_{\rm p} / R$

            \item Energy transfer out from the core: diffusively $t_{\rm diff}
                \sim R^2 / \ell c$, and possibly convectively if too large.

            \item Energy source is nuclear burning.
        \end{itemize}
        Combining all of these, we get the LMR relation!

        More precisely, we have $P$, $M$, $L$, $T$ dynamical variables, with
        $\kappa$, $\rho$, $Z$ prescribed variables, and they are all coupled to
        determine the structure of a star!
\end{itemize}

\section{Mar 16, 2022---Beyond the Main Sequence}

Late stellar evolution
\begin{itemize}
    \item At some point, cores become mostly Helium, what next? Off the
        \emph{main sequence}. This is necessary: after Big Bang, roughly $75\%$
        hydrogen, $25\%$ helium, trace amounts rest. We need heavier elements!
        Must come from stars, but how?

    \item As cores run out of H, they lose pressure support, contract, releasing
        energy and $T_{\rm c}$ increases. Causes \emph{hydrogen shell burning}.

        Unlike core burning, shell burning is \emph{unstable}. In core burning,
        any increase in burning rate decreases pressure which cools the core. In
        shell burning, burning increases core mass which increases pressure
        which increases burning rate! Red giants, expand by $\sim 100\times$.
        This terminates when $T_{\rm c} \sim 10^8\;\mathrm{K}$, at which point
        Helium fusion begins.

    \item \textbf{The Triple Alpha Reaction: 1953, Salpeter \& Hoyle}However,
        how does He fusion work? He-4 is so stable that any Be-8 formed rapidly
        decays back into He-4! Big puzzle.

        Resolution: at equilibrium, some small amount of Be-8 with short
        lifetime. However, C-12 is very stable; would be nice if Be-8 + He-4
        into C-12, but in typical stellar conditions, such collisions are far
        too rare (typical atomic cross sections).

        Hoyle suggests: maybe can pump up cross section via some sort of
        \emph{resonance}. To understand this, recall that photons can be
        absorbed efficiently when $E_\gamma + E_1 = E_2$, for ground state $1$
        and excited state $2$. Similarly for atoms. However, evaluating the
        energies for ground-state He-4, Be-8, C-12, we find no such resonance.
        Hoyle says that there must exist an excited C-12$^*$!

        In particular, He-4 + Be-8 $-$ C-12 = $7.4\;\mathrm{MeV}$, and kinetic
        energy is about $0.2\;\mathrm{MeV}$, so Hoyle claimed there must be an
        excited state with about $7.6\;\mathrm{MeV}$ above ground. Later
        experimentally verified.

    \item Once C-12 is made, O-16 is easy to make, gives rise to the
        \emph{cosmic abundances} of H, He, C, O, Ne-20,\dots all the way through
        Fe-56. Thus, continued nuclear burning, continued core contraction and
        shell burning leads to envelope exxpansion and mass ejection. How far a
        star goes depends on its mass, but once it stops, the star must
        collapse. Always happens, finite fuel.

    \item What are their end states? They are:
        \begin{itemize}
            \item WD\@: $M \sim 0.7 M_{\odot}$, $R \sim 10^9 \;\mathrm{cm} \sim
                R_{\odot} / 100 \sim R_{\oplus} \sim
                \scinot{6}{8}\;\mathrm{cm}$. For $M_{\rm MS} \lesssim
                8M_{\odot}$ (e.g.\ the Sun).

            \item NS\@: $M \sim 1.4M_{\odot}$, $R \sim 10\;\mathrm{km} \sim
                10^6\;\mathrm{cm}$ for $M_{\rm MS} \lesssim 40M_{\odot}$.

            \item BH\@: $M \gtrsim 3M_{\odot}$, $R = 2GM/c^2 \sim 3\;\mathrm{km}
                \p{M / M_{\odot}}$, for $M_{\rm MS} \gtrsim 40 M_{\odot}$.
        \end{itemize}
        We will start by understanding these objects, then we will understand
        how MS stars get here.
\end{itemize}

WDs:
\begin{itemize}
    \item Sounds like a crazy object: mass of the Sun in the radius of the
        Earth (density $10^6 \;\mathrm{g/cm^3}$, so the weight of a car in the
        size of a sugar cube)? However, observed! Sirus B, companion to Sirius A
        ($2M_{\odot}$). Opik in 1916 analyzed the spectrum, found surface
        temperature $T_{\rm s} \sim 6000\;\mathrm{K}$, but $L \lesssim 0.0003
        L_{\odot}$ (distance via paralla probably?). This suggests that its
        volume must be small, and $\bar{\rho} \sim \scinot{2}{5}\rho_{\odot}$!
        Opik concludes ``this impossible result indicates that our assumptions
        are wrong.''

    \item QM was developed in 1920s, \emph{after} Opik. Uncertainty principle,
        particles cannot both be stationary and unmoving, thus zero-point
        pressure even at zero temp. WD is supported by this zero-point pressure,
        \emph{degeneracy pressure}.

    \item Recall hydrostatic equilibrium $P/R \sim g\rho$, so $P_{\rm c} \sim
        GM^2 / R^4 \sim GM \rho / R$. If $P_{\rm c}$ is provided thermally
        (i.e.\ $P = nkT$), then
        \begin{equation}
            kT_{\rm c} \sim \frac{GMm_{\rm p}}{R}
        \end{equation}
        For $R = R_{\odot}$, $T_{\rm c} \sim 10^7\;\mathrm{K}$, but for $R \sim
        R_{\odot} / 100$, $T_{\rm c} \sim 10^9\;\mathrm{K}$. This is unlikely:
        steep temperature gradient to the surface, and would result in nuclear
        burning (but unstable! And would be explosive due to large $\rho$).

    \item Thus, need something else. Let's ignore temperatuer for now, then
        uncertainty
        \begin{align}
            \Delta p &\sim \hbar / \Delta x,\\
                &\sim \hbar n_{\rm e}^{1/3}
                    \sim \hbar \p{\frac{ZM}{R^3m_{\rm p}}}^{1/3},
        \end{align}
        so the mean kinetic energy per electron is
        \begin{equation}
            \epsilon_{\rm k} \equiv \frac{p_e^2}{2m_e}
                \sim \frac{\hbar^2}{m_{\rm e}} n_{\rm e}^{2/3}.
        \end{equation}
        The energy density is then $u = n_e \epsilon_{\rm k}$, which is also
        similar to the pressure $P \sim 2u/3 \sim u$. Thus, we finally find that
        the pressure should be equated as
        \begin{align}
            \frac{GM^2}{R^4} &\sim \frac{\hbar^2}{m_{\rm e}}
                \p{\frac{ZM}{R^3 Am_{\rm p}}}^{5/3},\\
            R &\sim \frac{\hbar^2}{Gm_{\rm e}}
                \frac{(Z/A)^{5/3}}{m_{\rm p}^{5/3}}
                \frac{1}{M^{1/3}} \propto M^{-1/3}.
        \end{align}
        Often denote $Y = Z / A$. A precise calculation gives a prefactor of
        $4.8$, and we find that
        \begin{equation}
            R \simeq \frac{R_{\odot}}{74}
                \p{\frac{M_{\odot}}{M}}^{1/3}
                \p{\frac{Z/A}{0.5}}^{5/3}.
        \end{equation}
        Note that:
        \begin{itemize}
            \item Smaller $R$ gives smaller $L$
            \item Larger $M$ gives smaller $R$, which \emph{increases} density.
            \item $v_{\rm e} \sim p_{\rm e} / m_{\rm e} \sim n_{\rm e}^{1/3}$.
                Thus, as density increases, so too does the electron momentum /
                velocity.
        \end{itemize}
        Of course, $v_{\rm e} \lesssim c$ is required, so this sets a maxximum
        mass. This is the \emph{Chandrasekar mass}, $\sim 1.4M_{\odot}$ for $Z /
        A = 0.5$.

    \item To get the Chandrasekar mass rigorously, consider \emph{relativistic}
        electrons.
\end{itemize}

\end{document}

