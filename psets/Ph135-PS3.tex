    \documentclass[12pt]{article}
    \usepackage{fancyhdr, amsmath, amsthm, amssymb, mathtools, lastpage,
    hyperref, enumerate, graphicx, setspace, wasysym, upgreek, listings, times}
    \usepackage{geometry}
    \newcommand{\scinot}[2]{#1\times10^{#2}}
    \newcommand{\bra}[1]{\left<#1\right|}
    \newcommand{\ket}[1]{\left|#1\right>}
    \newcommand{\dotp}[2]{\left<#1\,\middle|\,#2\right>}
    \newcommand{\rd}[2]{\frac{\mathrm{d}#1}{\mathrm{d}#2}}
    \newcommand{\pd}[2]{\frac{\partial#1}{\partial#2}}
    \newcommand{\rtd}[2]{\frac{\mathrm{d}^2#1}{\mathrm{d}#2^2}}
    \newcommand{\ptd}[2]{\frac{\partial^2 #1}{\partial#2^2}}
    \newcommand{\norm}[1]{\left|\left|#1\right|\right|}
    \newcommand{\abs}[1]{\left|#1\right|}
    \newcommand{\pvec}[1]{\vec{#1}^{\,\prime}}
    \newcommand{\tensor}[1]{\overleftrightarrow{#1}}
    \let\Re\undefined
    \let\Im\undefined
    \newcommand{\ang}[0]{\text{\AA}}
    \newcommand{\mum}[0]{\upmu \mathrm{m}}
    \DeclareMathOperator{\Re}{Re}
    \DeclareMathOperator{\Im}{Im}
    \DeclareMathOperator{\Log}{Log}
    \DeclareMathOperator{\Arg}{Arg}
    \DeclareMathOperator{\Tr}{Tr}
    \DeclareMathOperator{\E}{E}
    \DeclareMathOperator{\Var}{Var}
    \DeclareMathOperator*{\argmin}{argmin}
    \DeclareMathOperator*{\argmax}{argmax}
    \DeclareMathOperator{\sgn}{sgn}
    \newcommand{\expvalue}[1]{\left<#1\right>}
    \usepackage[labelfont=bf, font=scriptsize]{caption}\usepackage{tikz}
    \usepackage[font=scriptsize]{subcaption}
    \everymath{\displaystyle}
    \lstset{basicstyle=\ttfamily\footnotesize,frame=single,numbers=left}

\tikzstyle{circ} = [draw, circle, fill=white, node distance=3cm, minimum
height=2em]

\begin{document}

\onehalfspacing

\pagestyle{fancy}
\rhead{Yubo Su}
\cfoot{\thepage/\pageref{LastPage}}

\section{4a}

Our objective is to compute the following integral
\begin{align}
    \varepsilon &= \int \frac{\Theta(\vec{k}) - \Theta(\vec{k} + \vec{q})}
    {E(\vec{k} + \vec{q}) - E(\vec{k})}\;\mathrm{d}\vec{k}
\end{align}
for $\Theta$ the zero temperature Fermi-Dirac distribution and
$E(\vec{v}) = \frac{\hbar^2 v^2}{2m}$.

First, we separate the integral into two parts. Define $\pvec{k} = \vec{k} +
\vec{q}$, then
\begin{align}
    \varepsilon &= \int \frac{\Theta(\vec{k})}{E\left( \vec{k} + \vec{q} \right) -
    E\left( \vec{k} \right)}\;\mathrm{d}\vec{k} -
    \int \frac{\Theta(\pvec{k})}{E\left( \pvec{k} \right) - E\left( \pvec{k} -
    \vec{q} \right)}\;\mathrm{d}\pvec{k}
\end{align}
where both $\vec{k}, \pvec{k}$ are integrated in $k_F$-spheres about their
respective origins. Thus, we can drop the prime and write
\begin{align}
    \varepsilon &= \int \frac{\Theta(\vec{k})}{E\left( \vec{k} + \vec{q} \right) -
    E\left( \vec{k} \right)}\;\mathrm{d}\vec{k} -
    \int \frac{\Theta(\vec{k})}{E\left( \vec{k} \right) - E\left( \vec{k} -
    \vec{q} \right)}\;\mathrm{d}\vec{k}
\end{align}

Let's focus on the first integral, since the results for the second integral
follow. Recall that $\abs{ \vec{a} + \vec{b} }^2 = a^2 + b^2 + 2\vec{a}\cdot
\vec{b}$.  This allows us to write
\begin{align}
    \frac{\hbar^2}{2m}I_1 &= \int
        \frac{1}{2\vec{q} \cdot \vec{k} + q^2}\;\mathrm{d}\vec{k}\label{four}\\
    &= 2\pi \int\limits_{0}^{k_F}\int\limits_{0}^{\pi}
        \frac{1}{2qk\cos\theta + q^2}\;k^2\sin\theta
        \mathrm{d}\theta \mathrm{d}k\\
    &= 2\pi \int\limits_{0}^{k_F}\int\limits_{0}^{\pi}
        \frac{(k/2q)\sin\theta}{\cos\theta + q/2k}\;
        \mathrm{d}\theta \mathrm{d}k\\
    &= 2\pi \int\limits_{0}^{k_F}\frac{k}{2q}
        \left[ -\log\left| \cos \theta + \frac{q}{2k} \right| \right]_0^\pi
        \mathrm{d}\theta \mathrm{d}k\\
    &= 2\pi \int\limits_{0}^{k_F}\frac{k}{2q}
        \log \abs{\frac{1 + \frac{q}{2k}}{-1 + \frac{q}{2k}}}
        \mathrm{d}\theta \mathrm{d}k\\
    &= 2\pi \int\limits_{0}^{k_F}\frac{k}{2q}
        \log \abs{\frac{q/2 + k}{q/2 - k}}
        \mathrm{d}\theta \mathrm{d}k\label{nine}
\end{align}

Now, we can look up the following integral (derived later):
\begin{align}
    \int k \log \frac{C+k}{C-k}\;\mathrm{d}k &= \frac{1}{2} \left(
        \left( k^2 - C^2 \right)\log \frac{C+k}{C-k}
    \right) + kC\label{equation}
\end{align}

For us, $C = q/2$, and so we instantly obtain
\begin{align}
    I_1 &= \frac{\pi m}{\hbar^2 q} \left(
        \frac{\left( k_F^2 - \frac{q^2}{4} \right)}{2}
        \log \abs{\frac{q/2+k_F}{q/2-k_F}} + \frac{k_F q}{2}
    \right)
\end{align}
where we leverage the fact that $k=0$ forces both the $kC$ and the logarithm to
vanish in~\eqref{equation}.

For $I_2$, we note that in~\eqref{four}, the $q^2$ term changes sign.
Propagating this all the way through, we see that in~\eqref{nine}, the
$q/2$ changes sign. Pulling out a negative sign from the top and bottom, we find
that $I_2$ has the reciprocal of the log found in $I_1$ (i.e.\ if we were to
pursue the same algebra as above, in~\eqref{nine} we would obtain
$\frac{q/2-k}{q/2+k}$ as the argument of the logarithm). Thus, $I_2 = -I_1$\footnote
    {This should have been obvious in hindsight. Since we are integrating a
    spherical region of $\vec{k}$ about the origin, for every $\vec{k}$ we also
    include the contribution of $-\vec{k}$, and so the integrand of $I_2$ is
    equivalently $\left( E(\vec{k}) - E(\vec{k} + \vec{q}) \right)^{-1}$ which
    even more obviously yields $-I_1$.}
, and
so we find that
\begin{align}
    \varepsilon &= \frac{2\pi m}{\hbar^2 q} \left(
        \frac{\left( k_F^2 - \frac{q^2}{4} \right)}{2}
        \log \abs{\frac{q/2+k_F}{q/2-k_F}} + \frac{k_F q}{2}
    \right)\\
    &= \frac{2\pi m k_F}{\hbar^2} \left(
        \frac{\left( k_F^2 - \frac{q^2}{4} \right)}{2k_Fq}
        \log \abs{\frac{q/2+k_F}{q/2-k_F}} + \frac{1}{2}
    \right)\\
    &= \frac{2\pi m k_F}{\hbar^2} \left(
        \frac{\left( 4k_F^2 - q^2 \right)}{8k_Fq}
        \log \abs{\frac{q+2k_F}{q-2k_F}} + \frac{1}{2}
    \right)
\end{align}
which agrees with the desired answer up to a sign in the absolute value.

\textbf{Lemma:} We argue for the integral formula used above. We simply compute
for one sign
\begin{align}
    \int k \log(C + k)\;\mathrm{d}k ={}&
        k \left[ (C + k) \log (C + k) - k \right] \nonumber\\
    &- \int (C + k) \log (C + k) - k\;\mathrm{d}k \\
    ={}& k(C+k)\log(C+k) - k^2 - C\left[ (C + k) \log (C + k) - k \right]
        \nonumber\\
    &+ \frac{k^2}{2} - \int k\log(C+k)\;\mathrm{d}k\\
    2 \int k\log(C+k)\;\mathrm{d}k ={}&
        (k-C)(C+k)\log(C+k) - \frac{k^2}{2} - Ck\\
    \int k\log(C+k)\;\mathrm{d}k ={}&
        \frac{1}{2}\left[(k^2 - C^2)\log(C+k) - \frac{k^2}{2} - Ck\right]
        \label{final}
\end{align}

The second sign is obtained by flipping the sign of $k$, and the desired
integral is the difference between the two. The log terms in~\eqref{final} then
combine, the $\frac{k^2}{2}$ term cancels since it's the same in both signs, and
$Ck$ doubles since it incurs a sign flip. This reproduces~\eqref{equation}.

\end{document}

