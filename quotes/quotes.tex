    \documentclass[12pt]{article}
    \usepackage{fancyhdr, amsmath, amsthm, amssymb, mathtools, lastpage,
    hyperref, enumerate, graphicx, setspace, wasysym, upgreek, listings, times}
    % chancery
    \usepackage[margin=1in]{geometry}
    \newcommand{\scinot}[2]{#1\times10^{#2}}
    \newcommand{\bra}[1]{\left<#1\right|}
    \newcommand{\ket}[1]{\left|#1\right>}
    \newcommand{\dotp}[2]{\left<#1\,\middle|\,#2\right>}
    \newcommand{\rd}[2]{\frac{\mathrm{d}#1}{\mathrm{d}#2}}
    \newcommand{\pd}[2]{\frac{\partial#1}{\partial#2}}
    \newcommand{\rtd}[2]{\frac{\mathrm{d}^2#1}{\mathrm{d}#2^2}}
    \newcommand{\ptd}[2]{\frac{\partial^2 #1}{\partial#2^2}}
    \newcommand{\norm}[1]{\left|\left|#1\right|\right|}
    \newcommand{\abs}[1]{\left|#1\right|}
    \newcommand{\pvec}[1]{\vec{#1}^{\,\prime}}
    \newcommand{\svec}[1]{\vec{#1}\;\!}
    \newcommand{\bm}[1]{\boldsymbol{\mathbf{#1}}}
    \let\Re\undefined
    \let\Im\undefined
    \newcommand{\ang}[0]{\text{\AA}}
    \newcommand{\mum}[0]{\upmu \mathrm{m}}
    \DeclareMathOperator{\Re}{Re}
    \DeclareMathOperator{\Im}{Im}
    \DeclareMathOperator{\Log}{Log}
    \DeclareMathOperator{\Arg}{Arg}
    \DeclareMathOperator{\Tr}{Tr}
    \DeclareMathOperator{\E}{E}
    \DeclareMathOperator{\Var}{Var}
    \DeclareMathOperator*{\argmin}{argmin}
    \DeclareMathOperator*{\argmax}{argmax}
    \DeclareMathOperator{\sgn}{sgn}
    \DeclareMathOperator{\diag}{diag}
    \newcommand{\expvalue}[1]{\left<#1\right>}
    \usepackage[labelfont=bf, font=scriptsize]{caption}\usepackage{tikz}
    \usepackage[font=scriptsize]{subcaption}
    \everymath{\displaystyle}
    \lstset{basicstyle=\ttfamily\footnotesize,frame=single,numbers=left}

\tikzstyle{circ} = [draw, circle, fill=white, node distance=3cm, minimum
height=2em]

\begin{document}

\onehalfspacing

\pagestyle{fancy}
\rhead{Yubo Su}
\cfoot{\thepage/\pageref{LastPage}}

\title{List of Yubo's Favorite Quotes}
\author{Yubo Su}
\date{\today}

\maketitle

\tableofcontents

\clearpage

\section{Emerson}

\subsection{\emph{The American Scholar}}

\begin{itemize}
    \item \emph{``In this distribution of functions, the scholar is the delegated
        intellect. In the right state he is Man Thinking. In the degenerate
        state, he tends to become a mere thinker, or still worse, a parrot''}

        Emerson proposes that Man prospers because he divides labor, so one's
        role as a scholar is to be the thinking component of this aggregate Man,
        rather than just a parrot. Thus, we should think on behalf of humanity
        and not merely learn what others have thought.

    \item \emph{``Colleges can only highly serve when they aim not to drill, but to
        create; when they gather from far every ray of various genius to their
        hospitable halls, and by the concentrated fires set the hearts of their
        youth on flame.''}

        The context of the speech is some talk Emerson is giving members of some
        society at Harvard, so he is arguing for the role of higher education
        institutes. In line with his previous claim, he wishes colleges not to
        teach classics as revered classics but as concentrated musings of a
        colleague, to kindle the curiosity rather than calling it truth
        incarnate.

    \item \emph{``Action is with the scholal subordinate, but it is essential. Without
        it thought can never ripen into truth. [\dots] Only so much do I know,
        as I have lived.''}

        \emph{``Life is our dictionary. Colleges and books only copy the
        language which the field and the work-yard made.''}

        Part of a larger part where Emerson argues that scholars should not be
        recluses, and their being seen as such is because they have
        misunderstood a fundamental component of being a scholar, namely actualy
        experiencing things. Life is our dictionary, we can only understand and
        express that which we have experienced.

    \item \emph{``[Man Thinking] is one who raises himself from private
        considerations and breaths and lives on public and illustrious thoughts.
        He is the world's eye. He is the world's heart.''}

        Part of the conclusion to Emerson's talk where he argues that scholars
        give up much recognition and pleasures to pursue their work. It comes
        off a bit as self-pity to me, and I do not buy the argument. Scholars
        should study not because they want to be martyrs or the greater man but
        because they love their work and no else.
\end{itemize}

\subsection{\emph{Nature}}

\begin{itemize}
    \item \emph{``He who knows what sweets and virtues are in the ground, the
        waters, the plants, the heavens, and how to come at these enchantments,
        is the rich and royal man''}

        I am not reading \emph{Nature} super carefully since it's one huge rant
        on nature being amazing, but the above sentence is a rather poetic way
        of capturing the spirit of the paper.
\end{itemize}

\end{document}

