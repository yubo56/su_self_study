    \documentclass[10pt]{article}
    \usepackage{fancyhdr, amsmath, amsthm, amssymb, mathtools, lastpage,
    hyperref, enumerate, graphicx, setspace, wasysym, upgreek, listings}
    % chancery
    \usepackage[margin=1in]{geometry}
    \newcommand{\scinot}[2]{#1\times10^{#2}}
    \newcommand{\bra}[1]{\left<#1\right|}
    \newcommand{\ket}[1]{\left|#1\right>}
    \newcommand{\dotp}[2]{\left<#1\,\middle|\,#2\right>}
    \newcommand{\rd}[2]{\frac{\mathrm{d}#1}{\mathrm{d}#2}}
    \newcommand{\pd}[2]{\frac{\partial#1}{\partial#2}}
    \newcommand{\rtd}[2]{\frac{\mathrm{d}^2#1}{\mathrm{d}#2^2}}
    \newcommand{\ptd}[2]{\frac{\partial^2 #1}{\partial#2^2}}
    \newcommand{\norm}[1]{\left|\left|#1\right|\right|}
    \newcommand{\abs}[1]{\left|#1\right|}
    \newcommand{\pvec}[1]{\vec{#1}^{\,\prime}}
    \newcommand{\svec}[1]{\vec{#1}\;\!}
    \newcommand{\bm}[1]{\boldsymbol{\mathbf{#1}}}
    \let\Re\undefined
    \let\Im\undefined
    \newcommand{\ang}[0]{\text{\AA}}
    \newcommand{\mum}[0]{\upmu \mathrm{m}}
    \DeclareMathOperator{\Res}{Res}
    \DeclareMathOperator{\Re}{Re}
    \DeclareMathOperator{\Im}{Im}
    \DeclareMathOperator{\Log}{Log}
    \DeclareMathOperator{\Arg}{Arg}
    \DeclareMathOperator{\Tr}{Tr}
    \DeclareMathOperator{\E}{E}
    \DeclareMathOperator{\Var}{Var}
    \DeclareMathOperator*{\argmin}{argmin}
    \DeclareMathOperator*{\argmax}{argmax}
    \DeclareMathOperator{\sgn}{sgn}
    \DeclareMathOperator{\diag}{diag}
    \newcommand{\expvalue}[1]{\left<#1\right>}
    \usepackage[labelfont=bf, font=scriptsize]{caption}\usepackage{tikz}
    \usepackage[font=scriptsize]{subcaption}
    \everymath{\displaystyle}
    \lstset{basicstyle=\ttfamily\footnotesize,frame=single,numbers=left}

\tikzstyle{circ} = [draw, circle, fill=white, node distance=3cm, minimum
height=2em]

\begin{document}

\onehalfspacing

\pagestyle{fancy}
\rhead{Yubo Su}
\cfoot{\thepage/\pageref{LastPage}}

\tableofcontents

\newpage

\section{Introduction}

\begin{itemize}
    \item Learn a language by answering the following questions:
        \begin{itemize}
            \item What is the typing model? Static dynamic, strong weak?
            \item What is the programming model? OOP, functionall, procedural,
                hybrid of which?
            \item How will you interact? Compiled, interpreted, VMs?
            \item Design constructs/core data structures? Pattern matching,
                collections, unification?
            \item Core features that make it unique?
        \end{itemize}

    \item The languages:
        \begin{itemize}
            \item Ruby---OOP representative.
            \item Io---concurrency constructs w/ simplicity, uniformity and
                minimality of syntax.
            \item Prolog---Parent to Erlang? Old. Nothing else mentioned.
            \item Scala---Functional + OOP to Java.
            \item Erlang---Functional w/ concurrency, distribution + fault
                tolerance \emph{right}. BAse of CouchDB.\@
            \item Clojure---On JVM, same concurrency as versioned dbs. Lisp
                dialect
            \item Haskell---Pure functional, archetypal typing model.
        \end{itemize}

    \item Glossary (to be all on the same page):
        \begin{description}
            \item[Interpreted] Executed by an interpreter rather than a
                compiler.
            \item[Strongly Typed] Errors when types collide.
            \item[Dynamically Typed] Types bound at runtime rather than compile
                time. Generally means types inside functions are only checked on
                execution.
            \item[Statically Typed] Infer polymorphism by relation to other
                types rather than just ``is this function defined'', checked at
                compile time.
            \item[Duck Typing] If an object has a function then that function is
                invokable without type checking for the parent.
            \item[Object Oriented] Encapsulation (data + behavior together),
                inheiritance and polymorphism.
            \item[Prototype Language] Every object is a clone of another, a
                style of OOP.\@
            \item[Declarative Language] ``Throw facts and inferences at the
                language and it will reason for you.''
            \item[Purely Functional] A function invoked with the same inputs
                always produces the same output. \emph{Alias: stateless}.
            \item[Higher-order Functions] Functions that map functions to
                functions.
            \item[Threads vs.\ Processes] Both have their own execution path but
                threads share resources in exchange for being more lightweight.
            \item[Atoms] Symbols that represent singular values, like entries in
                an enumerated type.
        \end{description}
\end{itemize}

\section{Ruby}

\begin{itemize}
    \item Optimized w/ syntactic sugar, programmer efficiency.
    \item Interpreted, OOP, dynamically typed, strongly typed, duck
        typed scripting language.
    \item Every piece of code returns, even if only \lstinline{nil}.
        \begin{itemize}
            \item Functions return the value of the last expression.
        \end{itemize}
    \item Purely OOP, e.g.\ \lstinline{4.class = Fixnum} and has methods viewable
        by \lstinline{4.methods}.
    \item \lstinline{if}, \lstinline{unless}, \lstinline{while}, \lstinline{until} can be
        used either inline or in block form.
    \item \lstinline{nil}, \lstinline{false} are only falsey values, \lstinline{0} is
        true!
    \item Each object natively understands equality.
    \item \emph{Symbols} are prefixed with \lstinline{:identifier}. Identical
        symbols point to the same physical object, unlike identical objects, can
        tell by checking their \lstinline{:identifier.object\_id}.
    \item Arrays are Ruby's primary ordered collection (Ruby 1.9 has ordered
        hashes).
        \begin{itemize}
            \item Out of bounds yields \lstinline{nil}.
            \item Negative counts backwards.
            \item \lstinline{arr[0..1]} returns a slice, since \lstinline{0..1} is a
                \lstinline{Range}.
            \item \lstinline{[]} is a function on \lstinline{Array}.
            \item No need to be homogeneous types.
            \item Implement queue, LL, stack, se etc.\@
        \end{itemize}
    \item Hashes are labeled collections, key-value pairs.
    \item \emph{Code blocks}
        \begin{itemize}
            \item Code blocks are unnamed functions, between braces or
                \lstinline{do/end}, former when single line, latter when multiple
                lines.
            \item Can be passed as function argument, prototype says
                \lstinline{\&block} and can invoke with \lstinline{block.call}.
            \item \lstinline{yield} calls whatever block is passed to the function.
            \item Can be used for delaying execution and conditional execution
                as well.
        \end{itemize}
    \item OOP
        \begin{itemize}
            \item \lstinline{initialize} constructor
            \item Class names are camel cased, instance variables and method
                names are snake cased, constants all caps.
            \item Instance variables are prepended with a single \lstinline{\@},
                class variables with two \lstinline{\@\@}.
            \item \lstinline{modules} to solve multiple inheritance, collection of
                functions and constants, \lstinline{include}ed by \lstinline{class}es.
            \item \lstinline{modules} can call functions it does not define but
                expect \lstinline{include}-ees to define, duck typing! Implicit
                ``abstract functions'' from Java.
        \end{itemize}
    \item \emph{Metaprogramming} is writing programs that write programs.
    \item \emph{Open Classes} allow us to modify existing classes in-line, even
        built-ins like \lstinline{NilClass}.
        \begin{itemize}
            \item A fun use case is to override the
                \lstinline{self.method\_missing} function, which is called whenever
                an attribute is not found. Then, a class called \lstinline{Roman}
                can have attributes like \lstinline{Roman.XII} and use
                \lstinline{method\_missing} to compute the value! Wow! \smiley.
        \end{itemize}
    \item \lstinline{Modules} are extremely adept at metaprogramming, since a
        modulee's \lstinline{included} method is called whenever it is included, so
        it can metaprogram on inclusion.
    \item Core strengths
        \begin{itemize}
            \item Duck typed with OOP is out-of-the-box polymorphism.
            \item Fast for scripting, well-supported for various extensions.
            \item Rails!! Fast time to market.
        \end{itemize}
    \item Weaknesses
        \begin{itemize}
            \item Performance: getting much faster, but still slow.
                Metaprogramming makes any compilation nigh impossible. Also
                against the core design philosophy of programmer's experience vs
                performance.
            \item Concurrency is hard with OOP.\@
            \item No type safety.
        \end{itemize}
\end{itemize}

\section{Io}

\begin{itemize}
    \item Prototype language like Lua and Javascript, no distinction between
        objects and classes, developed in 2002.
    \item Everything is a message that returnss another receiver. Program by
        chaining messages, e.g.\ \lstinline{"Hello World" print}. Message passing
        is a strong concurrency model.
    \item Objects and classes are the same, create new objects by cloning
        existing ones e.g.\ \lstinline{Vehicle} \lstinline{:= Object clone}.
        \begin{itemize}
            \item Inheritance is equivalent to sending the \lstinline{clone}
                message to a parent prototype.
        \end{itemize}
    \item Objects have ``slots'', and a collection of slots is like a hash.
        Objects are basically collections of slots. Can
        \lstinline{Object slotNames} to get list of slots.
    \item When a slot is not found on an object, it is forwarded up to parent
        prototypes or until not found.
    \item Lowercase clones do not override parent's \lstinline{type} slot.
    \item \emph{Methods} are objects with \lstinline{type} \lstinline{Block}. Can be
        attached to object slots, are invoked when the slot is invoked.
    \item \lstinline{Lobby} is an object with a slot for each name in the global
        namespace.
    \item Lists \lstinline{list()} are the prototype for all ordered collections,
        and Maps \lstinline{map()} are the prototype for all key value stores.
    \item \lstinline{true, false, nil} are \emph{singletons}, i.e.\ their
        \lstinline{clone} returns themselves rather than a clone of them! Lots of
        cool tricks by overriding core functionality like this.
    \item Can see list of operators directly with precedence by
        \lstinline{OperatorTable} and create new operators. Use case: short JSON
        $\to$ \lstinline{Map} parser.
    \item Message reflection is possible with the \lstinline{call} operator inside
        method bodies, e.g.\ \lstinline{call message arguments}.
        \begin{itemize}
            \item The reason message reflection works is because the full
                message context (sender, target, message) are all pushed onto
                the execution stack.
            \item In Io, messages passed as arguments to a method are only
                pushed onto the stack and \emph{not evaluated}.
            \item This means that a receiver can call \lstinline{call sender *} and
                hit an arbitrary sender slot.
        \end{itemize}
    \item Can override \lstinline{forward} message slot same way as
        \lstinline{method\_missing} before.
    \item Concurrency
        \begin{itemize}
            \item \emph{Coroutines} are functions w/ multiple entry/exits. Firing a
                message with \lstinline{\@} returns a future, with two
                \lstinline{\@\@} returns \lstinline{nil} and kicks off a new thread.
            \item \lstinline{yield} yields control inside a coroutine.
            \item \emph{Actors} place incoming messages on a queue and dequeue
                with coroutines. An object becomes an actor when sent an
                asynchronous (\lstinline{\@, \@\@}) message.
            \item \emph{Futures} return immediately, but when accessed block
                until the asynchronous result is returned.
        \end{itemize}
    \item Strengths
        \begin{itemize}
            \item Tiny footprint, heavily used for embedded systems.
            \item Compact syntax, fast rampup.
            \item Flexibility because all slots and operators are exposed.
        \end{itemize}
    \item Weaknesses
        \begin{itemize}
            \item Minimal syntactic sugar.
            \item Slow single-threaded execution speed.
        \end{itemize}
\end{itemize}

Illustrative example of reflection, to print slots of ancestors of any object
that clones \lstinline{Object}:
\begin{lstlisting}
Object ancestors := method(
    prototype := self proto
    if(prototype != Object,
        writeln("Slots of ", prototype type)
        prototype slotNames forEach(name, writeln(slotName))
        writeln
        prototype ancestors
    )
)
\end{lstlisting}

\section{Prolog}

\begin{itemize}
    \item Declarative, from 1972.
    \item First letter capitalization says whether an identifier is an
        \emph{atom}, fixed value like a Ruby symbol and lower cased, or
        \emph{variable} whose values can change and are upper cased or start
        with underscore.
    \item Three building blocks, facts, rules and queries. Facts and rules go
        into a \emph{knowledge base}, the Prolog compiler serializes facts and
        rules to be query efficient.
        \begin{description}
            \item[Facts] \lstinline{likes(person, object)} declares a fact.
            \item[Rules] \lstinline{friend(X,Y):-\+(X=Y), likes(X,Z),likes(Y,Z)}
                is an example of a rule.
                \begin{itemize}
                    \item \lstinline{:-} is a \emph{subgoal}. Rules can declare
                        multiple subgoals, only one of which needs be fulfilled.
                    \item \lstinline{\+} logical negation, $X \neq Y$.
                    \item \lstinline{friend(X,Y)} means $X \neq Y$ but both $X,
                        Y$ like $Z$!
                \end{itemize}
            \item[Queries] Run queries after loading a knowledge base, e.g.\
                \lstinline{likes(person, java)}. Same syntax!
        \end{description}
    \item Can specify \emph{variables} in query to query for \emph{all possible
        values} rather than just \lstinline{yes/no} above, e.g.\
        \lstinline{likes(Who, java)}.
        \begin{itemize}
            \item Seems like responses are pagined, with \lstinline{;} to
                advance and \lstinline{a} to get all of them.
        \end{itemize}
    \item Core ideas of logic engine:
        \begin{itemize}
            \item Unification assigns two structures to be identical,
                \lstinline{X = animal}.
            \item Prolog takes a query and unifies any variables in the query
                with any variables in the corresponding rule definition. The
                right hand side of the rule then becomes a list of
                \emph{goals} which are DFS'd.
        \end{itemize}
    \item Recursive rules are powerful but should tail recurse, recursive
        subgoal should be last.
    \item Prolog also has lists \lstinline{[]} and tuples \lstinline{()} which
        unify element by element. Lists can be deconstructed with
        \lstinline{[Head|Tail]}.
    \item Can unify with \lstinline{_} wildcard, discards.
    \item Use \lstinline{Variable is Expression} to unify \lstinline{Variable}
        against \lstinline{Expression}, e.g.
        \lstinline{Count is TailCount + 1}.
    \item Amazing to code up solutions to classic puzzles, many minds were
        boggled reading through the book exmaples.
\end{itemize}

\section{Scala}

\begin{itemize}
    \item Unifies OOP and functional. Statically typed, not purely functional.
        Compiled, the interpreter is actually a compiler + executor all in one.
        XML is a first-class programming construct!
    \item Key differences from Java:
        \begin{description}
            \item[Type Inference] Statically typed but can infer types.
            \item[Immutable Variables] Taking \lstinline{final} to the next
                level.
            \item[Functional Programming] lol.
            \item[Higher-level Abstractions] e.g.\ actors in concurrency.
        \end{description}
    \item Type inference e.g.\ \lstinline{4 + "abc" = "4abc"}, though
        \lstinline{val a: Int = 1} can declare explicitly.
    \item \lstinline{val} is immutable, \lstinline{var} is not to declare.
    \item Supports first-class \lstinline{Ranges} (\lstinline{0 until 10}) and
        \lstinline{Tuples} (fixed length, can do multivalue assignments).
    \item OOP features look mostly like Java with some small syntactic sugar
        (seems the \lstinline{varName: Type} annotation is the more popular one
        now, as Scala uses it too).
    \item \lstinline{object} defines singleton classes, vs \lstinline{class} for
        instance templates. An \lstinline{object}'s functions are like
        \lstinline{static} functions in vanilla Java, and since you can name an
        \lstinline{object} and \lstinline{class} the same name, the
        \emph{companion object} approach lets you put \lstinline{static} Java
        functions in the \lstinline{object} and the instance functions in the
        \lstinline{class}.
    \item \lstinline{traits} are like Java \lstinline{Interface}s plus an
        implementation. \lstinline{implements} becomes \lstinline{with} in Scala.
    \item Moving onto more functional features, can define naked functions too.
    \item Also have \lstinline{List}s, can mix types (to JVM, list of
        \lstinline{Any}s). List access is done with \lstinline{()}.since it is a
        function. Also, native \lstinline{Map}s.
    \item \lstinline{Sets} can be unioned, differenced and intersected with
    \lstinline{++,--,\item}.
    \item Equality works by value for lists and sets!
    \item All three of these collections can be iterated in various more
        functional ways e.g. \ \lstinline{foreach, fold}.
    \item Can assign XML to variables, query, iterate, pattern match! Supports
        XPath syntax for querying. Apparently useful since Java is XML heavy.
    \item Pattern matching both looks a lot like switch case but also hearkens
        to various pattern matching we've seen in Prolog and will see in other
        functional programming languages!
    \item Concurrency done with actors and message passing, send message using
        \lstinline{!} e.g.\ \lstinline{actor ! message}. \lstinline{Actor} is a
        parent class to \lstinline{extends}.
        \begin{itemize}
            \item \lstinline{loop} to loop while waiting for message
            \item \lstinline$react { case ... }$ for pattern-matched
                message-passing handling, or \lstinline{receive} equivalently.
        \end{itemize}
\end{itemize}

Pattern matching + XML example
% ugh lacheck is too dumb for lstlisting . . .
\begin{verbatim}
val movies = <movies>
    <movie>a</movie>
    <film>b</film>
</movies>
(movies \ "_").foreach { movie =>
    movie match {
        case <movie>{movieName></movie> => println(movieName)
        case <film>{filmName></film> => println(filmName + " (film)")
        case _ => println(" error")
    }
}
\end{verbatim}

\section{Erlang}

Oh boy \emph{Few languages have the mystique of Erlang, the concurrency language
that makes hard things easy and easy things hard}.

\begin{itemize}
    \item Near real-time, fault-tolerant distributed applications, hot
        swappable. Functional, language-level message passing. Compiled but has
        interpreter. Dynamically typed.
        \begin{itemize}
            \item Much syntax comes from Prolog, was implemented over Prolog at
                first!
        \end{itemize}
    \item Erlang goes for processes instead of threads, tries to keep them
        lightweight. Eliminates need for locks.
    \item Actors are processes, pattern match off a queue of messages to decide
        how to process.
    \item Erlang handles reliability by ``let it crash'' mantra, then spinning
        processes back up. Lightweight processes important!
    \item Periods terminate lines wow \smiley.
    \item \lstinline{String}s are just \lstinline{List}s, i.e.\ an array of
        ASCII codes prints as a string in \lstinline{erl}.
    \item Basic type coercion but no string $\leftrightarrow$ ints.
    \item Lowercase symbols are atoms, not even assignable and only used in
        pattern matching, uppercase are variables, assignable but immutable..
    \item Tuples \lstinline|{}| and lists \lstinline{[]} exist, former is fixed
        length is all.
    \item Maps are generally implemented as list/tuple of tuples, and to access
        we pattern match!
        \begin{itemize}
            \item Convention is to use the first entry in list as atom
                corresponding to ``type'' of object, so it's easy to guarantee
                pattern matching matches the right object type/attributes.
        \end{itemize}
    \item \emph{Bit matching} lets you pack data into effectively structs w/
        predefined bit lengths for each attribute, e.g.
        \lstinline{<<A:3, B:5, C:5, D:5>>} is a struct w/ 16 total bits, and
        can unpack via pattern matching too.
    \item \lstinline{module(name)}s \lstinline{export([function/numArgs])} in a
        file, where we use shorthand \lstinline{f/1} to say function
        \lstinline{f} takes one arg. Namespaced with
        \lstinline{module:function}.
        \begin{itemize}
            \item \lstinline{mirror(Obj) -> Obj.} is valid syntax in Erlang.
                Note differs from before since we don't specify type of
                \lstinline{Obj} but the return type of \lstinline{mirror} is
                obvious by its argument type. Dynamic typing!
        \end{itemize}
    \item Erlang also tail recurses, but does not overflow on $2000!$ and is
        instantaneous apparently according to book example!
    \item Tend to use \lstinline{case} statements to parse message passing,
        common \lstinline{_ ->} wildcard match.
        \begin{itemize}
            \item Semicolon after each case except for last, so no
                \lstinline{comma-dangle} eslint rule.
            \item Commas separate lines within each case
        \end{itemize}
    \item \lstinline{if} statements use \emph{guards}, so kinda pattern match on
        condition. Can have more than two guards in an if statement.
        \begin{itemize}
            \item Multiple guards can match, ordering is important.
        \end{itemize}
    \item \lstinline{fun(args)} defines an anonymous function of
        \lstinline{args}, can assign to a Variable.
        \begin{itemize}
            \item Usual applications, \lstinline{lists:foldl},
                \lstinline{lists:map}, \lstinline{lists:filter} and more under
                the \lstinline{lists} module
        \end{itemize}
    \item Build lists using \lstinline{[|]} head/tail notation on the RHS of an
        assignment.
    \item List comprehensions with \lstinline{[op(X) || <clauses>]} where
        \lstinline{clauses} can be generators for multiple variables and/or
        constraints on these variables.
    \item Three basic operations, \lstinline{!} sending message,
        \lstinline{spawn}ing a process and \lstinline{receive}ing a message.
        \begin{itemize}
            \item Similar to \lstinline{loop} syntax of Ruby, the canonical way
                to ``infinite loop'' in Erlang is to have a function
                \lstinline{loop()} call itself via tail recursion.
            \item We then \lstinline{spawn(loop/0)} a process that runs a
                function, in this case the infinite loop \lstinline{loop()}
                above. Returns a \lstinline{Pid}.
            \item Then we \lstinline{! message} to the \lstinline{Pid}. When no
                explicit return is specified by an actor (or if the actor is
                dead), returns the message itself.
            \item To achieve synchronous message passing, the receiver can take
                a \lstinline{Pid} and message pass the return back, and the
                sender can send \lstinline{self()} and then \lstinline{receive}
                the returned message and \lstinline{->} return it.
        \end{itemize}
    \item Erlang does provide checked exceptions, but instead \emph{let it
        fail} and handle its failure by linking processes together and handling
        error cases.
        \begin{itemize}
            \item \lstinline{process_flag(trap_exit, true)} registers the
                current process as \emph{the} process one traps exits.
            \item Receives \lstinline${'EXIT', From, Reason }$ provided by
                registering under \lstinline{process_flag}.
            \item Can also receive another message that takes a
                \lstinline{Process} and \lstinline{link(Process)}es it to
                receive exit messages. Can use \lstinline{spawn_link} to spawn
                and link immediately.
            \item Then, if in the \lstinline{EXIT} message, the exit trapper
                \lstinline{spawn_link}s another process, then we have built in
                error handling!
        \end{itemize}
    \item Strengths
        \begin{itemize}
            \item Dynamically typed yet the most reliable core libraries.
            \item Lightweight processes, isolation but easy to let die and spin
                up.
            \item Open Telecom Platform (OTP) enterprise libraries are vast,
                owing to Erlang's original roots in a telecom company.
            \item Let it crash very easy to reason about concurrency.
        \end{itemize}
    \item Weaknesses
        \begin{itemize}
            \item Syntax: everybody hates Erlang syntax \frownie.
        \end{itemize}
\end{itemize}

\section{Clojure}

\begin{itemize}
    \item Lisp on the JVM\@. Dynamically, weakly typed. Lisp is:
        \begin{itemize}
            \item Language of lists: prefix notation is just a list w/ operator
                first.
            \item \emph{Data as code} data structures express code.
        \end{itemize}
    \item Concurrency done via agents. Rather than being fully immutable,
        transactional memory.
    \item Atoms/symbols in clojure are preceeded with \lstinline{:}.
    \item Lists, maps, sets and vectors.
        \begin{itemize}
            \item Lists: \lstinline{()}. \lstinline{first}, \lstinline{last},
                \lstinline{rest}. \lstinline{cons} appends.
            \item Vectors/arrays: \lstinline{[]}. \lstinline{nth} so can random
                access.
            \item Sets: \lstinline$#{}$. Can \lstinline{sorted-set}. Are
                also functions (!) that test for membership!
            \item Maps: \lstinline${}$. Every other entry is a key, others are
                values. Traditionally commas every two elements since they're
                just whitespace. Also functions, do lookup. Can do both
                \lstinline{(:key map)}, \lstinline{(map :key)}.
        \end{itemize}
    \item Use \lstinline{def} to bind a value to  a variable.
    \item \lstinline{defn} to bind a function,
        \lstinline{(defn <docstr> func [params] body)}, optional docstring
        accessible via \lstinline{(doc func)}.
    \item Can destructure e.g.\ the following are equivalent:
        \lstinline{(defn f [l] (last l))} and
        \lstinline{(defn f [[_ l]] l)}
    \item Can declare anonymous function with just
        \lstinline{(fn [params] body)}. If e.g.\ mapping, can even syntactic
        sugar to \lstinline{(map #(* 2 \%) nums)} where \lstinline{\%} refers to
        each item mapped over.
    \item Clojure does not tail recurse automagically. Instead, use
        \lstinline{recur} to re-enter a \lstinline{loop} to effect tail
        recursion.
    \item \emph{Sequences} are an abstraction library over lists, vectors, sets,
        even I/O.
        \begin{itemize}
            \item Syntactic sugar for iterating through a list:
                \lstinline{(for [x colors] ...)} is like ``for x in colors.''
            \item Lazy sequences! \lstinline{take} to take a finite number of an
                infinite sequence.
        \end{itemize}
    \item OO constructions
        \begin{itemize}
            \item \lstinline{defprotocol} defines a Java interface, functions
                that records implementing the \lstinline{protocol} must implement.
            \item \lstinline{defrecord} lets you specify what
                \lstinline{protocol} you're implementing.
        \end{itemize}
    \item \lstinline{macro}s
        \begin{itemize}
            \item The problem is that w/ many functional languages, we evaluate
                args then push them onto the call stack. Possible case we do not
                want to do this is the \lstinline{unless} keyword from before;
                do not evaluate function body.
            \item Solution is to \lstinline{defmacro}, treat even executable
                code as a list and do not execute, push directly onto call
                stack instead. Powerful b/c data is all code!
        \end{itemize}
    \item Concurrency is solved via \emph{software transactional memory}.
        \begin{itemize}
            \item Most operations so far have been immutable, get
                \emph{references} via \lstinline{ref} to a variable, then can
                \lstinline{alter} it inside a \lstinline{dosync} transaction,
                dereference with \lstinline{@ref} or full form
                \lstinline{deref}.
            \item Can also \lstinline{ref-set} instead of operating on the
                existing value.
            \item Can use \lstinline{atom}s to allow change changing w/o
                transaction.
            \item \lstinline{agent}s are also wrapped pieces of data that get
                mutated asynchronously via \lstinline{send}. To get the value of
                a reference after an asynchronous message has been processed,
                \lstinline{await}.
            \item Can also use \lstinline{future} to block until a value is
                available.
        \end{itemize}
    \item Can implement multimethods depending on object metadata (which can be
        attached to symbols and collections).
    \item Clojure is a great lisp b/c fewer parens, ecosystem b/c Java, better
        concurrency, lazy evaluation. It is pretty inaccessible though, being a
        Lisp.
\end{itemize}

\section{Haskell}

I already know a bit of Haskell so I might be a bit terse here, plus I need to
finish this book now.

\begin{itemize}
    \item Absolutely pure functional, born out of functional programming
        language conference in 1990 and revised in 1998. Strong, static typing,
        widely considered the most effective type system.
    \item Few primitive types, numbers, strings, bools. No type coercion, very
        strong typing!
    \item Functions can be dynamically typed, allow type variables.
    \item Same pattern matching as before, takes first of matches. Can use
        guards \lstinline{| expr = func} to guard expr to only execute when
        func.
    \item Same destructuring syntax as Clojure for the most part, except
        destructure list using \lstinline{h:t} rather than \lstinline{h|t}.
    \item Lists are homogeneous here!
    \item Compose functions using \lstinline{f1.f2}, concat lists using
        \lstinline{:} operator.
    \item List comprehension like Erlang, e.g.
        \lstinline{[x*2|x<-[1,2,3]]}.
    \item Define anonymous functions with \lstinline{\x -> x}.
    \item Higher order functions map, filter, fold etc.\ many many!
    \item Cool concepts:
        \begin{itemize}
            \item Currying
            \item Lazy evaluation, infinite lists, \lstinline{take}
        \end{itemize}
    \item Haskell type system:
        \begin{itemize}
            \item Define types \lstinline{data Boolean = True | False}.
            \item To know how to display, you can \lstinline{deriving (Show)}.
            \item Polymorphic functions by typing a function w/ type variables.
            \item \lstinline{data} definitions can also be polymorphic.
            \item \lstinline{class} defines function typings for people that
                derive it and can have implementations for these functions for
                derivees.
        \end{itemize}
    \item Monads
        \begin{itemize}
            \item Monads are meant to simulate program state.
            \item Consist of a type constructor to hold a value, a
                \lstinline{return} function that wraps a function into a monad
                and a \lstinline{>>=} to unwrap a function in a monad.
            \item \lstinline{monad >>= return = monad}.
            \item Example monad \lstinline{Position}:
\begin{lstlisting}
module Main where
    data Position t = Position t deriving (Show)

    move (Position d) = Position (d + 2)
    rtn x = x
    x >>== f = f x
\end{lstlisting}
                then the advantage of using a monad for this is that instead of
                doing \lstinline{(move (move (move position)))} where the
                innermost one actually moves first, we can write them in
                executed order instead!
            \item The binding reads ``for a given monad \lstinline{x}, bind the
                application of function \lstinline{f} to apply \lstinline{f x},
                so an identity monad.
            \item \lstinline{do} + \lstinline{return} makes function
                declarations look imperative but actually uses monads.
        \end{itemize}
\end{itemize}

\end{document}

