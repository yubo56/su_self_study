    \documentclass[12pt]{report}
    \usepackage{fancyhdr, amsmath, amsthm, amssymb, mathtools, lastpage,
    hyperref, enumerate, graphicx, setspace, wasysym, upgreek, listings,
    times}
    \usepackage[margin=1in]{geometry}
    \newcommand{\scinot}[2]{#1\times10^{#2}}
    \newcommand{\bra}[1]{\left<#1\right|}
    \newcommand{\ket}[1]{\left|#1\right>}
    \newcommand{\dotp}[2]{\left<#1\,\middle|\,#2\right>}
    \newcommand{\rd}[2]{\frac{\mathrm{d}#1}{\mathrm{d}#2}}
    \newcommand{\pd}[2]{\frac{\partial#1}{\partial#2}}
    \newcommand{\rtd}[2]{\frac{\mathrm{d}^2#1}{\mathrm{d}#2^2}}
    \newcommand{\ptd}[2]{\frac{\partial^2 #1}{\partial#2^2}}
    \newcommand{\norm}[1]{\left|\left|#1\right|\right|}
    \newcommand{\abs}[1]{\left|#1\right|}
    \newcommand{\pvec}[1]{\vec{#1}^{\,\prime}}
    \newcommand{\svec}[1]{\vec{#1}\;\!}
    \newcommand{\tensor}[1]{\overleftrightarrow{#1}}
    \let\Re\undefined
    \let\Im\undefined
    \newcommand{\ang}[0]{\text{\AA}}
    \newcommand{\mum}[0]{\upmu \mathrm{m}}
    \DeclareMathOperator{\Re}{Re}
    \DeclareMathOperator{\Im}{Im}
    \DeclareMathOperator{\Log}{Log}
    \DeclareMathOperator{\Arg}{Arg}
    \DeclareMathOperator{\Tr}{Tr}
    \DeclareMathOperator{\E}{E}
    \DeclareMathOperator{\Var}{Var}
    \DeclareMathOperator*{\argmin}{argmin}
    \DeclareMathOperator*{\argmax}{argmax}
    \DeclareMathOperator{\sgn}{sgn}
    \DeclareMathOperator{\diag}{diag}
    \newcommand{\expvalue}[1]{\left<#1\right>}
    \usepackage[labelfont=bf, font=scriptsize]{caption}\usepackage{tikz}
    \usepackage[font=scriptsize]{subcaption}
    \everymath{\displaystyle}
    \lstset{basicstyle=\ttfamily\footnotesize,frame=single,numbers=left}

\tikzstyle{circ} = [draw, circle, fill=white, node distance=3cm, minimum
height=2em]

\begin{document}

\onehalfspacing

\pagestyle{fancy}
\rhead{Yubo Su}
\cfoot{\thepage/\pageref{LastPage}}

\tableofcontents

\chapter{Flat Spacetime}

\section{Spacetime Physics}

\begin{itemize}
    \item Space tells matter how to move, matter tells space how to curve.
        Physics is simple when local, and when we eliminate force at a distance
        life is good.

    \item \emph{Events} are coordinate-independent points in spacetime that are
        defined by ``what happens there'' or more concretely by an intersection
        of worldlines.

    \item Spacetime is locally Lorentzian, a.k.a.\ flat or non-accelerating.
        Gravitation/curvature is defined as the acceleration of the separation
        between two nearby geodesics.

        We measure curvature by the following: characterize $\xi$ the separation
        between two originally parallel geodesics, then propagate each forward
        by distance $s$. They are not necessarily parallel anymore (e.g.\ two
        great circles on a sphere), and the separation obeys the following EOM
        \begin{align}
            \rtd{\xi}{s} + R\xi &= 0
        \end{align}
        where $R$ is the \emph{Gaussian Curvature} of the surface.

        Generalizing to multiple dimensions, the separation $\mathbf{\xi}$ is a
        vector, and we describe $\frac{\mathrm{D}^2\mathbf{\xi}}{\mathrm{d}^2s}$
        with a capital $\mathrm{D}$ since the coordinates of the derivative
        $\rtd{\mathbf{\xi}}{s}$ is subject to the whims of the coordinate lines,
        which should not affect the separation between these two geodesics that
        live beyond coordinates. The curvature is instead described by the
        Reiman curvature tensor, which takes as arguments the 4-velocity
        $\mathbf{u} = \rd{x^\alpha}{\tau}$ and yields
        \begin{align}
            \frac{\mathrm{D}^2\xi^\alpha}{\mathrm{d}^2\tau^2} +
                {R^{\alpha}}_{\beta\gamma\delta}
                \rd{x^\beta}{\tau}\xi^\gamma \rd{x^\delta}{\tau} &= 0
        \end{align}
        where $\mathbf{R}$ is the 4-component Reimann curvature tensor. We call
        this the \emph{equation of geodesic deviation}.

        These functions are sister functions of the Lorentz force equation in
        electromagnetism
        \begin{align*}
            \rtd{x^\alpha}{\tau} - \frac{e}{m}{F^\alpha}_\beta \rd{x^\alpha}{\tau}
                &= 0.
        \end{align*}

    \item $\mathbf{R}$ is a fickle object, subject to perturbation e.g.\ by
        gravitational waves. A certain piece of the tensor is generated only
        by the local mass distribution though, $\mathbf{G}$ the \emph{Einstein
        curvature tensor}, incidently proportional to the \emph{stress-energy
        tensor} as $\mathbf{G} = 8\pi \mathbf{T}$. Its interpretation is a local
        average curvature.
\end{itemize}

\subsection{Practice Problems}

\begin{description}
    \item[Exercise 1.1] \emph{Show that the Gaussian curvature of a cylinder
        $R=0$.}

        Choose cylindrical coordinates $(a, \theta, z)$ for the surface of the
        cylinder of radius $a$. We assert temporarily that all geodesics can be
        parameterized as $g(s|\omega, \dot{z}) = (a, \omega s, \dot{z}s)$, i.e.\
        correspond to uniform translation along and rotation about the axis of
        the cylinder. Then, using the equation of geodesic deviation with
        Gaussian Curvatures
        \begin{align*}
            \rtd{\xi}{s} + R\xi &= 0
        \end{align*}
        for $\xi = g_1 - g_2$ the separation, then since $g_1, g_2$ are linear
        functions of $s$, their second derivatives with respect to $s$ are zero
        and thus $\rtd{\xi}{s} = 0$. Observe that this happens regardless of the
        value of $\xi$, thus $R=0$.

        We now justify our parameterization $g(s|\omega, \dot{z})$. Two points
        separated infinitisemally $(a, \theta, z)$, $(a, \theta + \delta\theta, z
        + \delta z)$ are separated by distance
        \begin{align*}
            \sqrt{\delta z^2 + a^2 \delta \theta^2} =
                \mathrm{d}\theta\sqrt{\left(\rd{z}{\theta}\right)^2 + a^2}.
        \end{align*}

        The distance between two points $(\theta_1, z_1), (\theta_2, z_2)$ is
        then given by
        \begin{align*}
            D[z(\theta)] &= \int\limits_{z_1}^{z_2}
                \sqrt{\left(\rd{z}{\theta}\right)^2 + a^2}\;\mathrm{d}\theta
        \end{align*}
        where $z(\theta)$ is any trajectory that runs between the desired
        endpoints. This is then a variational calculus problem, and thus the
        $z(\theta)$ that minimizes the path must satisfy the Euler-Lagrange
        Equation
        \begin{align*}
            \rd{}{\theta}\pd{I(z', z)}{z'} - \pd{I(z', z)}{z} &= 0.
        \end{align*}

        We note then that
        $I(z', z) = \sqrt{\left(\rd{z}{\theta}\right)^2 + a^2}$ is independent
        of $z$ and so we instantly find that $I(z',z) = C$ a constant. This
        furthermore implies that $\rd{z}{\theta}$ is a constant and so that $z
        \propto \theta$. Call the constant of proportionality $z \propto \alpha
        \theta$.

        Armed with this, we find that the arclength of the geodesic between the
        desired endpoints is
        \begin{align}
            D[z(\theta)] &= \int\limits_{z_1}^{z_2}
                \sqrt{\alpha^2 + a^2}\;\mathrm{d}\theta\\
                &= \sqrt{\alpha^2 + a^2}\Delta z.
        \end{align}

        Recalling that $s$ parameterizes the length of our geodesic, we find
        that $s, z, \theta$ must all be proportional. Call the ratios
        $\frac{\theta}{s} = \omega, \frac{z}{s} = \dot{z}$, justifying our
        parameterization.

    \item[Exercise 1.1b] \emph{Alternatively, employ $R =
        \frac{1}{\rho_1\rho_2}$ where $\rho_1, \rho_2$ are the principal radii
        of curvature at the point in question in the enveloping Euclidean 3D
        space.}

        We note that one of the radii in a cylinder is $a$ while the other is
        infinite, thus $R=0$.

    \item[Exercise 1.3] \emph{Show that given $\omega$, the rotational frequency
        of a planet about a fixed central mass $M$, we can not individually
        determine $r$ the radius of orbit or $M$. Instead, derive a relation
        between $\omega$ and $\rho$ the \emph{Kepler Density} of the mass, the
        density if $M$ were spread over the sphere of radius $r$.}

        We know that a central acceleration $\ddot{r} = r\omega^2 =
        \frac{GM}{c^2r^2}$ under Newton's Law of Gravitation. Thus, we have
        \begin{align*}
            \omega^2 &= \frac{GM}{c^2r^3} = \frac{4\pi}{3}\frac{G}{c^2}\rho
        \end{align*}
        where since there are no other constraints on the motion we are done.
\end{description}

\section{Special Relativity and 1-forms}

SR/GR aspire to describe all laws as relationships between geometric objects,
i.e.\ events are points, [tangent] vectors connect points, and the metric
(defining lengths of vectors) is also geometric. First some definitions

\begin{description}
    \item[Vector] The vector joining $A,B$ as $\rd{}{\lambda}P(\lambda)$ where
        $P(\lambda)$ is the straight line joining $A,B$.

    \item[Metric tensor] $G(\mathbf{u}, \mathbf{v}) = \mathbf{u} \cdot
        \mathbf{v}$.

    \item[Differential forms] Consider the propagation of a plane wave with
        wavevector $\vec{k}$. There are surfaces of codimension $1$ that have
        equal phase. Define $\mathbf{\tilde{k}}$ such that a vector $\vec{v}$
        incurs phase difference $\expvalue{\mathbf{\tilde{k}}, \vec{v}}$.

        This is an example of a 1-form, a differential form that takes a single
        vector and returns a scalar, often also denoted by $\mathbf{\sigma}$
        boldface greek letters. We call $\expvalue{\mathbf{\sigma}, \vec{v}}$
        the contraction of $\mathbf{\sigma}$ with $\mathbf{\vec{v}}$.

        Note that there is a one-to-one correspondence $\mathbf{\tilde{p}}$ and
        $\mathbf{p}$, i.e.\ $\expvalue{\mathbf{\tilde{p}}, \mathbf{u}}$ is the
        same as $\mathbf{p} \cdot \mathbf{u}$, and so we sometimes omit the
        tilde.

        More generally, a 1-form is a local approximation the same way a vector
        is a local derivative, i.e.\ if we want the phase of a point $P$ close
        to $P_0$, we have
        \begin{align}
            \phi(P) = \phi(P_0) + \expvalue{\mathbf{\tilde{k}}, P - P_0} + \dots
        \end{align}

    \item[Gradient as a 1-form] The gradient is a 1-form, not a vector, since it
        is the first-order approximation
        \begin{align}
            f(P) = f(P_0) + \expvalue{\mathbf{d}f, P - P_0} +\dots
        \end{align}

    \item[Coordinates] Choose an orthonormal basis $\mathbf{e}_\alpha$ for
        vectors, then there is also a basis $\mathbf{w}^\alpha =
        \mathbf{d}x^\alpha$ for 1-forms, such that
        \begin{align}
            \expvalue{\mathbf{w}^\alpha, \mathbf{e}_\beta} &=
                \delta^\alpha_\beta\\
            \expvalue{v^\alpha \mathbf{w}^\alpha, \sigma_\beta \mathbf{e}_\beta}
                &= \expvalue{\mathbf{\sigma}, \mathbf{v}} = \sigma_\alpha
                v^\alpha
        \end{align}

    \item[4-velocity] To work w/ dynamical variables, we need to start mapping
        quantities to geometric objects somewhere. Start with the 4-velocity,
        which in specific Lorentz reference frame has components
        \begin{align}
            u^0 &= \rd{t}{\tau} &= \frac{1}{\sqrt{1 - \abs{\vec{v}^2}}} \nonumber\\
            u^j &= \rd{x^j}{\tau} &= \frac{v^j}{\sqrt{1 - \abs{\vec{v}}^2}}
        \end{align}

    \item[Lorentz Transformations] Two types, rotations and boosts, generate a
        tensor ${\Lambda^\alpha}_\beta$. Rotations look like
        \begin{align}
            \Lambda &= \begin{bmatrix}
            1 & 0 & 0 & 0\\
            0 & \cos\theta & \sin\theta & 0\\
            0 & -\sin\theta & \cos\theta & 0\\
            0 & 0 & 0 & 1
            \end{bmatrix}
        \end{align}
        while boosts look like
        \begin{align}
            \Lambda &= \begin{bmatrix}
                \cosh \alpha & 0 & 0 & \sinh\alpha\\
                0 & 1 & 0 & 0\\
                0 & 0 & 1 & 0\\
                \sinh\alpha & 0 & 0 & \cosh\alpha
            \end{bmatrix}
        \end{align}
        where velocity $\beta = \tanh \alpha$.

        Then both components and basis vectors are simply contracted w/ this
        tensor. Lorentz transformations obey $\Lambda^T \eta \Lambda = \eta$
        transpose.
\end{description}

\subsection{Exercises}

\begin{description}
    \item[2.2--2.4] \emph{Derive the following formulae:}
        \begin{align*}
            u_\alpha &= \eta_{\alpha\beta}u^\beta\\
            u^\alpha &= \eta^{\alpha\beta}u_\beta\\
            \mathbf{u} \cdot \mathbf{v} &= u^\alpha v^\beta \eta_{\alpha\beta}
                = u^\alpha v_\alpha
                = u_\alpha v_\beta \eta^{\alpha\beta}
        \end{align*}

        This is trivial,
        \begin{align}
            \mathbf{u} \cdot \mathbf{v} &= u_\alpha v_\beta \eta^{\alpha\beta}
                \nonumber\\
            &= \expvalue{\mathbf{\tilde{u}}, \mathbf{v}} = u^\alpha v_\beta
                \nonumber\\
            &= \expvalue{\mathbf{u}, \mathbf{\tilde{v}}} = u^\alpha v^\beta
                \nonumber\\
            u_\alpha \eta^{\alpha\beta} &= u^\alpha\\
            v_\beta \eta^{\alpha\beta} &= v^\beta
        \end{align}

    \item[2.5] \emph{Verify the following coordinate-free relations a particle
        of mass $m$ and 4-momentum $\mathbf{p}$ measured by an observer with
        4-velocity $\mathbf{u}$:}
        \begin{align*}
            \mathbf{u}^2 &= -1\\
            E &= -\mathbf{p} \cdot \mathbf{u}\\
            m^2 &= -\mathbf{p}^2\\
            \abs{\vec{p}} &= \sqrt{(\mathbf{p} \cdot \mathbf{u})^2 +
                (\mathbf{p} \cdot \mathbf{p})}\\
            \abs{\vec{v}} &= \frac{\abs{\vec{p}}}{E}\\
            \mathbf{v} &= \frac{\mathbf{p + (\mathbf{p} \cdot
                \mathbf{u})\mathbf{u}}}{-\mathbf{p} \cdot \mathbf{u}}
        \end{align*}

        We must go into an arbitrary Lorentz frame for this:
        \begin{align}
            \mathbf{u}^2 &= -\frac{1}{1 - v^2} + \sum\limits_{i}^{}
                \frac{v_i^2}{1 - v^2} \nonumber\\
            &= -1
        \end{align}

        We know that $E^2 = m^2 + \abs{\vec{p}}^2$, where $\vec{p} = \gamma m
        \vec{v}$, and so
        \begin{align}
            -\mathbf{p} \cdot \mathbf{u} &= -m\mathbf{v} \cdot \mathbf{u}
                \nonumber\\
            &= m\frac{1}{\sqrt{1 - v^2}} \nonumber\\
            m^2 + \abs{\vec{p}}^2 &= m^2\frac{1}{1-v^2} \nonumber\\
            E^2 &= (\mathbf{p} \cdot \mathbf{u})^2
        \end{align}
        and the sign is chosen by defining $E$ as positive.

        But then, knowing the definition $\mathbf{p} = m\mathbf{u}$, this yields
        $m^2 = -\mathbf{p}^2$.

        Again, $E^2 = m^2 + \abs{\vec{p}}^2$, so since
        $E^2 = (\mathbf{p} \cdot \mathbf{u})^2$ and $m^2 = -\mathbf{p} \cdot
        \mathbf{p}$, we find
        $\abs{\vec{p}}^2 = (\mathbf{p} \cdot \mathbf{u})^2 + \mathbf{p} \cdot
        \mathbf{p}$.

        Earlier, we used that $\vec{p} = \gamma m \vec{v}$, and moreover we've
        seen that $E = m\gamma$ so we have $E =
        \frac{\abs{\vec{p}}}{\abs{\vec{v}}} = m\gamma$.

        Not sure how to do the last one \frownie.

    \item[2.6] \emph{Ascribe a function $T(Q)$ to describe the temperature at
        event $Q$. To an observer traveling with 4-velocity $\mathbf{u}$, show
        that he measures temperature change $\rd{T}{\tau} =
        \expvalue{\mathbf{d}T, \mathbf{u}}$. Why is this reasonable?}

        If the observer travels from event $P$ to event $P +
        \mathbf{u}\Delta\tau$, then the temperature at the new event can be
        computed as an expansion about the initial point, which to first order
        is $\Delta T = \expvalue{\mathbf{d}T, \mathbf{u}\Delta \tau}$. Divide
        both sides by $\Delta \tau$ to obtain the desired result. Note that we
        want $\tau$ the proper time in the above relation because we are
        expanding in events about the rest frame of the temperature function
        $T$.

    \item[2.7] \emph{Show the below Lorentz transformation corresponds to motion
        with velocity $\beta \vec{n}$ and satisfies $\Lambda^T\eta \Lambda =
        \eta$.}
        \begin{align}
            {\Lambda^0}_0 &= \gamma \equiv \frac{1}{\sqrt{1 - \beta^2}}\\
            {\Lambda^0}_j &= {\Lambda^j}_0 = -\beta\gamma n^j\\
            {\Lambda^j}_k &= {\Lambda^k}_j = (\gamma - 1)n^jn^k + \delta^{jk}
        \end{align}

        We know that the above matrix is a Lorentz transformation if it
        satisfies $\Lambda^T \eta \Lambda = \eta$. Most explicitly, recall that
        ${\Lambda^\mu}_{\overline{\nu}}$ is the LT from unprimed vectors to
        primed vectors $v_\mu {\Lambda^\mu}_{\overline{\nu}}$. The transpose is
        swapping the order of the arguments ${\Lambda^\mu}_{\overline{\nu}}
        = {(\Lambda^T)_{\overline{\nu}}}^\mu$. Thus, what we aspire to show is
        \begin{align}
            {(\Lambda^T)_{\bar{\nu}}}^\mu \eta_{\mu\rho}
                {\Lambda^\rho}_{\bar{\sigma}} &=
                \eta_{\bar{\nu}\bar{\sigma}}\\
            \eta_{\mu\rho} {\Lambda^\mu}_{\bar{\nu}}
                {\Lambda^\rho}_{\bar{\sigma}} &= \eta_{\bar{\nu}\bar{\sigma}}
        \end{align}

        Let's examine the following few cases
        \begin{itemize}
            \item $\bar{\nu} = \bar{\sigma} = 0$ ---
                \begin{align*}
                    \eta_{\mu\rho}{\Lambda^\mu}_0{\Lambda^\rho}_0 &=
                        (-\gamma^2 + \beta^2\gamma^2n_jn^j) \\
                        &= -1
                \end{align*}
            \item $\bar{\nu} = \bar{\sigma} \neq 0$ ---
                \begin{align*}
                    \eta_{\mu\rho}{\Lambda^\mu}_i{\Lambda^\rho}_i &=
                        -\beta^2\gamma^2(n_i)^2 + \left[
                            (\gamma-1)^2(n_i)^2n^jn_j + 2(\gamma - 1)(n_i)^2 + 1
                        \right]\\
                    &= (n_i)^2\left[ \gamma^2 - 2\gamma + 1 + 2\gamma - 2 -
                        \beta^2\gamma^2  + 1\right] + 1\\
                    &= 1
                \end{align*}
            \item $\bar{\nu} \neq \bar{\sigma}, \bar{\nu} = 0$ ---
                \begin{align*}
                    \eta_{\mu\rho}{\Lambda^\mu}_0{\Lambda^\rho}_i &=
                        \beta\gamma^2 n^i - \beta\gamma(\gamma-1)n^in^jn_j -
                        \beta \gamma n^j\\
                        &= \beta\gamma(\gamma n^i - (\gamma-1)n^in^jn_j - n^j)\\
                        &= 0
                \end{align*}
            \item $\bar{\nu} \neq \bar{\sigma}$, none are $0$ ---
                \begin{align*}
                    \eta_{\mu\rho}{\Lambda^\mu}_i{\Lambda^\rho}_j &=
                        -\beta^2\gamma^2n^in^j + (\gamma - 1)^2n^in^jn^kn_k +
                        2(\gamma - 1)n^in^j\\
                        &= n^in^j\left[ -\beta^2\gamma^2 + \gamma^2 - 2\gamma +
                            1 + 2\gamma - 2 \right]\\
                        &= 0
                \end{align*}
        \end{itemize}

        This verifies our claim that this tensor acts as a Lorentz
        transformation. To verify that it is a boost of $\beta\hat{n}$, we could
        simply rotate such that $\hat{n}$ lies along a coordinate axis. This
        would be very easy if we had a simple form for the rotation matrix, but
        I don't have one handy, so let's be more creative. Consider starting in
        a reference frame moving with some $4$-velocity $\mathbf{v}$. The
        components in this comoving frame of the $4$-velocity are simply $(-1,
        \vec{0})$. If we boost with our provided Lorentz transformation, we
        obtain
        \begin{align}
            \mathbf{v} &= \left( -\gamma, -\beta\gamma \hat{n} \right)
        \end{align}
        which, comparing with our earlier provided formula, corresponds to a
        $3$-velocity of $-\beta \hat{n}$. Thus, the boost takes us into a frame
        that is moving with $+\beta \hat{n}$ relative to the initial reference
        frame.
\end{description}

\section{Tensors, Electromagnetic Field and Stress-Energy}

\begin{description}
    \item[Lorentz Force Law] The Lorentz force law both defines fields and
        predicts motions. It is written in geometric form
        \begin{align}
            \rd{p^\alpha}{\tau} &= e{F^\alpha}_\beta u^\beta
        \end{align}
        where $\mathbf{F}$ is the Faraday tensor satisfying
        \begin{align}
            \rd{\mathbf{p}}{\tau} &= e\mathbf{F}(\mathbf{u})\\
            {F^\alpha}_\beta &= \begin{bmatrix}
            0 & E_x & E_y & E_z\\
            E_x & 0 & B_z & -B_y\\
            E_y & -B_z & 0 & B_x\\
            E_z & B_y & -B_x & 0
            \end{bmatrix}
        \end{align}

    \item[Tensors] A tensor with rank $\binom{n}{m}$ takes $n$ 1-forms and $m$
        vectors to a scalar. Components of a tensor e.g.\
        ${S^{\alpha\beta}}_\gamma$ can be computed by inserting the basis objects
        ${S^{\alpha\beta}}_\gamma = {S^{\alpha\beta}}_\gamma \sigma_\alpha
        \rho_\beta v^\gamma$. These transform with Lorentz transformation too.

        We can always raise/lower indicies w/ the metric, so we only discuss the
        total rank of a tensor usually. In any case, a raised index is
        contravariant and a lowered index is called covariant.

    \item[Tensor Operations] In geometric notation, the gradient of a tensor
        $\mathbf{S}(\mathbf{u}, \mathbf{v}, \mathbf{w})$ is
        \begin{align}
            \mathbf{\nabla}\mathbf{S} &= \partial_\xi \left(
            S_{\alpha\beta\gamma} u^\alpha v^\beta w^\gamma \xi^\delta \right)
            \nonumber\\
            &= S_{\alpha\beta\gamma, \delta}u^\alpha v^\beta w^\gamma\xi^\delta
        \end{align}

        Contracting a tensor is to sum over one raised and one lowered index,
        e.g. $M_{\mu\nu} = {{R_{\alpha \mu}}^\alpha}_{\nu}$.

        Divergence (on the first index of $\mathbf{S}$) is defined
        \begin{align}
            \mathbf{\nabla} \cdot \mathbf{S} &= \mathbf{\nabla} \mathbf{S}
                (\mathbf{w}^\alpha, \mathbf{u}, \mathbf{v}, \mathbf{e}_\alpha)
            {S^{\alpha}}_{\beta\gamma,\alpha}
        \end{align}

        Transpose is interchanging the order of arguments.

        Wedge product is antisymmetrized tensor product
        \begin{align}
            \mathbf{u} \wedge \mathbf{v} &= \mathbf{u} \otimes \mathbf{v} -
                \mathbf{v} \otimes \mathbf{u}
        \end{align}

    \item[Maxwell's Equations] These become
        \begin{align}
            F_{\alpha\beta,\gamma} + F_{\beta\gamma, \alpha} +
            F_{\gamma\alpha,\beta} &= 0 &
            {F^{\alpha\beta}}_{,\beta} &= 4\pi J^\alpha
        \end{align}

    \item[Stress-Energy Tensor] $\mathbf{T}$ is a symmetric, second rank tensor
        whose components $T_{jk}$ is the $j$-component of force acting across a
        unit surface area $\mathbf{e}_k$. Note that ${\mathbf{T}^\alpha}_\beta
        u^\beta = -\rd{p^\alpha}{V}$ 4-momentum density, but also
        $T_{\alpha\beta} u^\beta$ gives the component of the 4-momentum density
        along the $\alpha$ direction.

        It is conserved $\mathbf{\nabla} \cdot \mathbf{T} = 0$ (note that thanks
        to symmetry we can take the divergence on either index).

        One example of $\mathbf{T}$ is the following: An observer wants to know
        at event $P_0$ how much 4-momentum a volume corresponding to 1-form
        $\mathbf{\Sigma}$ contains. This is expressed $\mathbf{p} =
        \mathbf{T}(\_, \mathbf{\Sigma})$, but the $1$-volume of the box is
        simply $\mathbf{\Sigma} = -V \mathbf{u}$, so $\mathbf{p} = V
        \mathbf{T}(\_, \mathbf{u})$.

        Alternatively, in a perfect fluid/ideal gas, we know that the
        space-space components of the tensor are $T_{ij} = p\delta_{ij}$, i.e.\
        they are equal on the diagonal and vanish off it, so long as we are in
        the rest frame of the fluid where all velocities are isotropically
        distributed. The off diagonal terms vanish since we will never measure
        net $x$ momenta when we are moving only in the $y$ direction if the
        momenta are isotropic, and the on-diagonal terms must all be equal for
        isotropy as well. Moreover, the $T_{0j}$ components are the densities of
        the $j$ components of momentum, while $T_{00}$ is the mass-energy
        density. Thus, $T_{00} = \rho, T_{ij} = p\delta_{ij}$ and all other
        terms are zero. Note that $\rho$ is the rest-plus-kinetic energy.

\end{description}

\subsection{Exercises}

\begin{description}
    \item[3.4] \emph{Formalize the tensor product $\mathbf{T} = \mathbf{u}
        \otimes \mathbf{v}$ e.g.\ if $\mathbf{u}, \mathbf{v}$ are first-rank
        tensors, then $T^{\alpha\beta} = u^\alpha v^\beta$.}

        Let $\mathbf{T} = \bigotimes_i \mathbf{u}_i$ where $\left\{ \mathbf{u}_i
        \right\}$ is a collection of tensors of rank $n_i$. Then $\mathbf{T}$ is
        a tensor of rank $N = \sum\limits_{i}^{}n_i$ whose indicies are simply
        the concatenation of the $\mathbf{u}_i$ indicies. Thus, in the above
        $\mathbf{T} = \mathbf{u} \otimes \mathbf{v}$, then the first index of
        $\mathbf{T}$ is the index of $\mathbf{u}$ and the second $\mathbf{v}$.

    \item[3.10] \emph{Show that any second rank tensor $\mathbf{T}$ can be
        uniquely decomposed into symmetric and antisymmetric components
        $T^{\mu\nu} = A^{\mu\nu} + S^{\mu\nu}$.}

        For any tensor $T^{\mu\nu}$, consider its transpose $T^{\nu\mu}$. Define
        \begin{align*}
            S^{\mu\nu} &= \frac{T^{\mu\nu} + T^{\nu\mu}}{2} = T^{(\mu\nu)} &
            A^{\mu\nu} &= \frac{T^{\mu\nu} - T^{\nu\mu}}{2} = T^{[\mu\nu]}
        \end{align*}
        and the inverse operation $T^{\mu\nu} = A^{\mu\nu} + S^{\mu\nu}$.
        Uniqueness is manifest because combining the equations for index
        $\mu\nu$ and index $\nu\mu$ results in two equations for two unknowns
        (either $A^{\mu\nu}, S^{\mu\nu}$ or $T^{\mu\nu}, T^{\nu\mu}$ depending
        on which direction we're going) which exhibits a single unique solution.
        If each component exhibits a unique solution then the entire system of
        equations generated by the matrix equation is unique.

    \item[3.11] \emph{Show that $A_{\mu\nu}S^{\mu\nu} = 0$, if $A,S$ are
        antisymmetric and symmetric respectively. This in conjunction with the
        result from the next problem shows that we only need the
        symmetric/antisymmetric portions of a tensor when multiplying against an
        antisymmetric/symmetric tensor.}

        For each fixed $\mu, \nu$, we see that $(A_{\mu\nu})(S_{\mu\nu}) =
        - (A_{\nu\mu})(S_{\nu\mu})$. Since for each $(\mu, \nu)$ pair we also
        sum over $(\nu, \mu)$ the sum must vanish.

    \item[3.12] \emph{Show that for an arbitrary tensor $V$, $V_{(\mu\nu\dots)}$
        is symmetric, $V_{[\mu\nu\dots]}$ is antisymmetric, and that only second
        rank tensor is fully described by its symmetric/antisymmetric parts.}

        Let $V$ be rank $N$, then we recall that $V_{(\mu\nu\dots)}$ is
        shorthand for
        \begin{align*}
            V_{(\mu\nu\dots)} &= \frac{1}{N!}
                \sum\limits_{\alpha\beta\dots \in\text{Permutations}}^{}
                V_{\alpha\beta\dots}
        \end{align*}
        and so when we index $V$ with a one permutation of indicies as opposed
        to another, we don't change the list of all permutations and thus
        $V_{(\mu\nu\dots)}$ is invariant under exchange of any indicies, thus
        rendering it symmetric.

        Recall moreover that
        \begin{align*}
            V_{[\mu\nu\dots]} &= \frac{1}{N!}
                \sum\limits_{\alpha\beta\dots \in\text{Permutations}}^{}
                V_{\alpha\beta\dots}\varepsilon_{\alpha\beta\dots}
        \end{align*}

        We note then that if we interchange two indicies on the left hand side,
        each of the $\varepsilon$ contribute a single sign change ($\varepsilon$
        is fully antisymmetric in all indicies) and so the left hand side must
        too be antisymmetric.

        As for full decomposition, we note that any $N$-rank tensor generally
        has $d^N$ degrees of freedom, assuming there are $d$ allowed values per
        index. To compute DOF for symmetric tensor, we have $N$ indistinct
        indicies to deposit into $d$ distinct values, yielding
        $\binom{d + N - 1}{N}$ allowed values. For antisymmetric, we have
        $d$ allowed values of which we choose $N$, so $\binom{d}{N}$ allowed
        values.

        The numerator of these expressions is bound by $(2k)^N$, since
        $N \leq d$, while we are dividing by $N!$. Thus, the ratio of the sum of
        the decomposed DOF to the full DOF is $\frac{2^N}{N!}$ which vanishes.
        In fact, even for $N=3$ we have $\binom{d+2}{3} + \binom{d}{3} < d^3$,
        so we know that for $N \geq 3$ we can't have equal numbers of DOF\@.

    \item[3.14] \emph{Define the \emph{dual} of a tensor $J$ to be obtained by
        contraction against the Levi-Civita symbol, e.g.\
        ${\star J}_{\alpha\beta\gamma} =
        {J^\mu}\varepsilon_{\mu\alpha\beta\gamma}$ in four dimensions. For an
        arbitrary $m$-rank tensor, the definition goes $(\star J) = \frac{1}{N!}
        J \varepsilon$. Show that $\star \star J = (-1)^N J$ for antisymmetric
        $J$.}

        We leverage the following identity:
        $\varepsilon^{\left\{ \alpha_i \right\}}
        \varepsilon_{\left\{ \beta_i \right\}} =
        \det\left( \delta_{\alpha_i\beta_j} \right)
        $ where the argument of the determinant is the $(i, j)$th index of a
        matrix and $\delta$ is the Kronecker delta.

        Then, $\star \star J$ where $J$ is of rank $m$ in an $N$ dimensional
        space is simply
        \begin{align*}
            (\star \star J)^{\alpha_i} = \frac{1}{N!}J^{\beta_i}
                \varepsilon_{\beta_i\dots \gamma_j}
                \varepsilon^{\gamma_j\dots \alpha_i}
        \end{align*}

        We cyclically permute the indicies on the second $\varepsilon$. This
        incurs $N$ swaps and so a net sign of $(-1)^N$. There are $\frac{N!}{m!
        }$ choices of these last $N-m$ indicies to sum over.

        Of these last indicies, we note that $(\star \star J)^{\alpha_i} =
        J^{\beta_i} \varepsilon_{\beta_i}\varepsilon^{\alpha_i}$. For each
        ordering of $\beta_i$, they must all be distinct obviously if they are
        to contribute to the sum, and if they are an even[odd] permutation of
        the $\alpha_i$ then $J^{\beta_i} = \pm J^{\alpha_i}$, but the
        $\varepsilon$ are same[different] signs too, so the two sides have the
        same sign regardless of permutation. That's $m!$ orderings, so for each
        set of indicies $\alpha_i$, we have $\frac{1}{N!}N!(-1)^NJ^{\alpha_i}$
        on the right hand side, and the claim is proven.

        We leveraged a simple lemma here, the identity initially stated, but
        it's easy to show that the expression has the desired properties. The
        two $\varepsilon$ symbols must be antisymmetric under exchange of two
        indicies. If two indicies are exchanged, then either two rows or two
        columns of the right hand side are exchanged, and by the property of
        determinants a net sign flip is also incurred. Moreover, in the base
        case $\left\{ \alpha_i \right\} \equiv \left\{ \beta_i \right\}$
        element-by-element, then both sides are clearly unity. Thus, the right
        hand side satisfies the desired properties.

\end{description}

\section{Accelerated Observers}

\begin{description}
    \item[Overview] SR accomodates accelerated observers via local Lorentz
        frames. It's impossible to tell whether you're accelerating or whether
        you're in gravity, so it's possible to compute all gravity as if just
        accelerating. The mathematical result is GR\@.

    \item[Hyperbolic Motion] If we require $\mathbf{a} = \rd{\mathbf{u}}{t}$ to
        satisfy $0 = \mathbf{a} \cdot \mathbf{u}$, then we can show that this
        produces a trajectory satisfying $x^2 - t^2 = g^{-2}$, hyperbolic
        motion (note that $a^\mu a_\mu = g^2$).

    \item[Rotations in 4D] In 3D, rotation was defined about an angular velocity
        vector $\omega_i$. In 4D, we instead rotate about a plane
        $\Omega^{\mu\nu}$, such that
        \begin{align}
            \rd{v^\mu}{\tau} &= -\Omega^{\mu\nu}v_\nu
        \end{align}

        Compare to $\rd{v_i}{t} = \epsilon_{ijk} \omega_j v_k$.

    \item[Coordinates in Accelerated Frames] Choose basis vectors
        $\mathbf{e}_\alpha$ such that $\mathbf{e}_0 = \mathbf{u}$ the 4-velocity
        of the frame, and $\mathbf{e}_1$ the acceleration. Thus, $\mathbf{e}_2,
        \mathbf{e}_3$ must be unaffected by LT in the 1-direction. We lastly
        require no rotation outside of the above two criteria.

        We can verify that the choice $\mathbf{\Omega} = \mathbf{a} \wedge
        \mathbf{u}, \Omega^{\mu\nu} = a^\mu u^\nu - a^\nu u^\mu$ is the rotation
        that is enforced by our former criteria: any vector orthogonal to
        $\mathbf{a}, \mathbf{u}$ doesn't get rotated $\Omega^{\mu\nu}w_\nu = 0$.

        A vector that undergoes the given Lorentz transformation specified by
        $\mathbf{\Omega}$ (i.e.\ $\rd{v^\mu}{\tau} = \Omega^{\mu\nu}v_\nu$) is
        said to undergo \emph{Fermi-Walker transport} along the wordline of the
        observer.

        These coordinates approximate a Lorentz coordinate system out to
        $g^{-1}$.

\end{description}

\subsection{Exercises}

\begin{description}
    \item[Self] Derive hyperbolic motion.

    \item[6.1] Compute the proper time for an observer to travel distance $L$ at
        acceleration $g$ for half, then $-g$ for the other half.

    \item[6.6] Show that in the LLT coordinate system $\xi^\alpha$, the proper
        time measured at coordinate $\xi^1$ is given $\rd{\tau}{\xi^0} =
        1 + g\xi^1$.

    \item[6.8] Consider an observer that allows his LLT to rotate, not
        Fermi-Walker transport. Specifically, we rotate as
        \begin{align}
            \rd{\mathbf{e}_\alpha}{\tau} &= -\mathbf{\Omega} \cdot
            \mathbf{e}_\alpha
        \end{align}
        where $\mathbf{\Omega} = \mathbf{a} \wedge \mathbf{u} +
        u_\alpha \omega_\beta \mathbf{\varepsilon}^{\alpha \beta \mu \nu}$, the
        second component being the spatial rotation, where $\mathbf{\omega}$ is
        perpendicular to 4-velocity $\mathbf{u}$. Show that
        \begin{itemize}
            \item $\mathbf{e}_0 = \mathbf{u}$ is permitted by the above.

            \item The second term, call it $\mathbf{\Omega}_{SR}$, obeys
                \begin{align*}
                    \mathbf{\Omega}_{SR} \cdot \mathbf{u} &= 0 &
                    \mathbf{\Omega}_{SR} \cdot \mathbf{\omega} &= 0
                \end{align*}

            \item Compare the basis vectors transported via the above rules with
                those obtained via Fermi-Walker transport. Show that the
                space-like components are rotated by $\mathbf{\omega}$. Hint:
                pick a moment where the tetrads coincide and show that the
                change in their difference is rotating like $\vec{\omega} \cdot
                \vec{e}_{j}$.
        \end{itemize}
\end{description}

\end{document}

