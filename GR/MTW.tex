    \documentclass[12pt]{report}
    \usepackage{fancyhdr, amsmath, amsthm, amssymb, mathtools, lastpage,
    hyperref, enumerate, graphicx, setspace, wasysym, upgreek, listings, times}
    \usepackage{geometry}
    \newcommand{\scinot}[2]{#1\times10^{#2}}
    \newcommand{\bra}[1]{\left<#1\right|}
    \newcommand{\ket}[1]{\left|#1\right>}
    \newcommand{\dotp}[2]{\left<#1\,\middle|\,#2\right>}
    \newcommand{\rd}[2]{\frac{\mathrm{d}#1}{\mathrm{d}#2}}
    \newcommand{\pd}[2]{\frac{\partial#1}{\partial#2}}
    \newcommand{\rtd}[2]{\frac{\mathrm{d}^2#1}{\mathrm{d}#2^2}}
    \newcommand{\ptd}[2]{\frac{\partial^2 #1}{\partial#2^2}}
    \newcommand{\norm}[1]{\left|\left|#1\right|\right|}
    \newcommand{\abs}[1]{\left|#1\right|}
    \newcommand{\pvec}[1]{\vec{#1}^{\,\prime}}
    \newcommand{\svec}[1]{\vec{#1}\;\!}
    \newcommand{\tensor}[1]{\overleftrightarrow{#1}}
    \let\Re\undefined
    \let\Im\undefined
    \newcommand{\ang}[0]{\text{\AA}}
    \newcommand{\mum}[0]{\upmu \mathrm{m}}
    \DeclareMathOperator{\Re}{Re}
    \DeclareMathOperator{\Im}{Im}
    \DeclareMathOperator{\Log}{Log}
    \DeclareMathOperator{\Arg}{Arg}
    \DeclareMathOperator{\Tr}{Tr}
    \DeclareMathOperator{\E}{E}
    \DeclareMathOperator{\Var}{Var}
    \DeclareMathOperator*{\argmin}{argmin}
    \DeclareMathOperator*{\argmax}{argmax}
    \DeclareMathOperator{\sgn}{sgn}
    \DeclareMathOperator{\diag}{diag}
    \newcommand{\expvalue}[1]{\left<#1\right>}
    \usepackage[labelfont=bf, font=scriptsize]{caption}\usepackage{tikz}
    \usepackage[font=scriptsize]{subcaption}
    \everymath{\displaystyle}
    \lstset{basicstyle=\ttfamily\footnotesize,frame=single,numbers=left}

\tikzstyle{circ} = [draw, circle, fill=white, node distance=3cm, minimum
height=2em]

\begin{document}

\onehalfspacing

\pagestyle{fancy}
\rhead{Yubo Su}
\cfoot{\thepage/\pageref{LastPage}}

\tableofcontents

\chapter{Flat Spacetime}

\section{Spacetime Physics}

\subsection{Notes}

\begin{itemize}
    \item Space tells matter how to move, matter tells space how to curve.
        Physics is simple when local, and when we eliminate force at a distance
        life is good.

    \item \emph{Events} are coordinate-independent points in spacetime that are
        defined by ``what happens there'' or more concretely by an intersection
        of worldlines.

    \item Spacetime is locally Lorentzian, a.k.a.\ flat or non-accelerating.
        Gravitation/curvature is defined as the acceleration of the separation
        between two nearby geodesics.

        We measure curvature by the following: characterize $\xi$ the separation
        between two originally parallel geodesics, then propagate each forward
        by distance $s$. They are not necessarily parallel anymore (e.g.\ two
        great circles on a sphere), and the separation obeys the following EOM
        \begin{align}
            \rtd{\xi}{s} + R\xi &= 0
        \end{align}
        where $R$ is the \emph{Gaussian Curvature} of the surface.

        Generalizing to multiple dimensions, the separation $\mathbf{\xi}$ is a
        vector, and we describe $\frac{\mathrm{D}^2\mathbf{\xi}}{\mathrm{d}^2s}$
        with a capital $\mathrm{D}$ since the coordinates of the derivative
        $\rtd{\mathbf{\xi}}{s}$ is subject to the whims of the coordinate lines,
        which should not affect the separation between these two geodesics that
        live beyond coordinates. The curvature is instead described by the
        Reiman curvature tensor, which takes as arguments the 4-velocity
        $\mathbf{u} = \rd{x^\alpha}{\tau}$ and yields
        \begin{align}
            \frac{\mathrm{D}^2\xi^\alpha}{\mathrm{d}^2\tau^2} +
                R^{\alpha}_{\beta\gamma\delta}
                \rd{x^\beta}{\tau}\xi^\gamma \rd{x^\delta}{\tau} &= 0
        \end{align}
        where $\mathbf{R}$ is the 4-component Reimann curvature tensor. We call
        this the \emph{equation of geodesic deviation}.

        These functions are sister functions of the Lorentz force equation in
        electromagnetism
        \begin{align*}
            \rtd{x^\alpha}{\tau} - \frac{e}{m}F^\alpha_\beta \rd{x^\alpha}{\tau}
                &= 0.
        \end{align*}

    \item $\mathbf{R}$ is a fickle object, subject to perturbation e.g.\ by
        gravitational waves. A certain piece of the tensor is generated only
        by the local mass distribution though, $\mathbf{G}$ the \emph{Einstein
        curvature tensor}, incidently proportional to the \emph{stress-energy
        tensor} as $\mathbf{G} = 8\pi \mathbf{T}$. Its interpretation is a local
        average curvature.
\end{itemize}

\subsection{Practice Problems}

\begin{description}
    \item[Exercise 1.1] \emph{Show that the Gaussian curvature of a cylinder
        $R=0$.}

        Choose cylindrical coordinates $(a, \theta, z)$ for the surface of the
        cylinder of radius $a$. We assert temporarily that all geodesics can be
        parameterized as $g(s|\omega, \dot{z}) = (a, \omega s, \dot{z}s)$, i.e.\
        correspond to uniform translation along and rotation about the axis of
        the cylinder. Then, using the equation of geodesic deviation with
        Gaussian Curvatures
        \begin{align*}
            \rtd{\xi}{s} + R\xi &= 0
        \end{align*}
        for $\xi = g_1 - g_2$ the separation, then since $g_1, g_2$ are linear
        functions of $s$, their second derivatives with respect to $s$ are zero
        and thus $\rtd{\xi}{s} = 0$. Observe that this happens regardless of the
        value of $\xi$, thus $R=0$.

        We now justify our parameterization $g(s|\omega, \dot{z})$. Two points
        separated infinitisemally $(a, \theta, z), (a, \theta + \delta\theta, z
        + \delta z)$ are separated by distance
        \begin{align*}
            \sqrt{\delta z^2 + a^2 \delta \theta^2} =
                \mathrm{d}\theta\sqrt{\left(\rd{z}{\theta}\right)^2 + a^2}.
        \end{align*}

        The distance between two points $(\theta_1, z_1), (\theta_2, z_2)$ is
        then given by
        \begin{align*}
            D[z(\theta)] &= \int\limits_{z_1}^{z_2}
                \sqrt{\left(\rd{z}{\theta}\right)^2 + a^2}\;\mathrm{d}\theta
        \end{align*}
        where $z(\theta)$ is any trajectory that runs between the desired
        endpoints. This is then a variational calculus problem, and thus the
        $z(\theta)$ that minimizes the path must satisfy the Euler-Lagrange
        Equation
        \begin{align*}
            \rd{}{\theta}\pd{I(z', z)}{z'} - \pd{I(z', z)}{z} &= 0.
        \end{align*}

        We note then that
        $I(z', z) = \sqrt{\left(\rd{z}{\theta}\right)^2 + a^2}$ is independent
        of $z$ and so we instantly find that $I(z',z) = C$ a constant. This
        furthermore implies that $\rd{z}{\theta}$ is a constant and so that $z
        \propto \theta$. Call the constant of proportionality $z \propto \alpha
        \theta$.

        Armed with this, we find that the arclength of the geodesic between the
        desired endpoints is
        \begin{align}
            D[z(\theta)] &= \int\limits_{z_1}^{z_2}
                \sqrt{\alpha^2 + a^2}\;\mathrm{d}\theta\\
                &= \sqrt{\alpha^2 + a^2}\Delta z.
        \end{align}

        Recalling that $s$ parameterizes the length of our geodesic, we find
        that $s, z, \theta$ must all be proportional. Call the ratios
        $\frac{\theta}{s} = \omega, \frac{z}{s} = \dot{z}$, justifying our
        parameterization.

    \item[Exercise 1.1b] \emph{Alternatively, employ $R =
        \frac{1}{\rho_1\rho_2}$ where $\rho_1, \rho_2$ are the principal radii
        of curvature at the point in question in the enveloping Euclidean 3D
        space.}

        We note that one of the radii in a cylinder is $a$ while the other is
        infinite, thus $R=0$.

    \item[Exercise 1.3] \emph{Show that given $\omega$, the rotational frequency
        of a planet about a fixed central mass $M$, we can not individually
        determine $r$ the radius of orbit or $M$. Instead, derive a relation
        between $\omega$ and $\rho$ the \emph{Kepler Density} of the mass, the
        density if $M$ were spread over the sphere of radius $r$.}

        We know that a central acceleration $\ddot{r} = r\omega^2 =
        \frac{GM}{c^2r^2}$ under Newton's Law of Gravitation. Thus, we have
        \begin{align*}
            \omega^2 &= \frac{GM}{c^2r^3} = \frac{4\pi}{3}\frac{G}{c^2}\rho
        \end{align*}
        where since there are no other constraints on the motion we are done.
\end{description}

\end{document}

