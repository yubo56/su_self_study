    \documentclass[10pt]{article}
    \usepackage{fancyhdr, amsmath, amsthm, amssymb, mathtools, lastpage,
    hyperref, enumerate, graphicx, setspace, wasysym, upgreek, listings}
    \usepackage[margin=0.5in, top=0.8in,bottom=0.8in]{geometry}
    \newcommand{\scinot}[2]{#1\times10^{#2}}
    \newcommand{\bra}[1]{\left<#1\right|}
    \newcommand{\ket}[1]{\left|#1\right>}
    \newcommand{\dotp}[2]{\left<#1\,\middle|\,#2\right>}
    \newcommand{\rd}[2]{\frac{\mathrm{d}#1}{\mathrm{d}#2}}
    \newcommand{\pd}[2]{\frac{\partial#1}{\partial#2}}
    \newcommand{\rtd}[2]{\frac{\mathrm{d}^2#1}{\mathrm{d}#2^2}}
    \newcommand{\ptd}[2]{\frac{\partial^2 #1}{\partial#2^2}}
    \newcommand{\norm}[1]{\left|\left|#1\right|\right|}
    \newcommand{\abs}[1]{\left|#1\right|}
    \newcommand{\pvec}[1]{\vec{#1}^{\,\prime}}
    \newcommand{\tensor}[1]{\overleftrightarrow{#1}}
    \let\Re\undefined
    \let\Im\undefined
    \newcommand{\ang}[0]{\text{\AA}}
    \newcommand{\mum}[0]{\upmu \mathrm{m}}
    \DeclareMathOperator{\Re}{Re}
    \DeclareMathOperator{\Im}{Im}
    \DeclareMathOperator{\Log}{Log}
    \DeclareMathOperator{\Arg}{Arg}
    \DeclareMathOperator{\Tr}{Tr}
    \DeclareMathOperator{\E}{E}
    \DeclareMathOperator{\Var}{Var}
    \DeclareMathOperator*{\argmin}{argmin}
    \DeclareMathOperator*{\argmax}{argmax}
    \DeclareMathOperator{\sgn}{sgn}
    \newcommand{\expvalue}[1]{\left<#1\right>}
    \usepackage[labelfont=bf, font=scriptsize]{caption}\usepackage{tikz}
    \usepackage[font=scriptsize]{subcaption}
    \everymath{\displaystyle}
    \lstset{basicstyle=\ttfamily\footnotesize,frame=single,numbers=left}

\tikzstyle{circ} = [draw, circle, fill=white, node distance=3cm, minimum
height=2em]

\begin{document}

\pagestyle{fancy}
\rhead{Yubo Su --- Independent Study: Ph161}
\cfoot{\thepage/\pageref{LastPage}}

\tableofcontents
\clearpage

\section{Day 1 --- Lorenz Model}

The Lorenz Model arises as an approximation to atmospheric flow, starting with
the description of the simplest mode of a convection zone and dropping terms
that represent higher harmonics. This gives the following system of equations
\begin{align}
    \dot{X} &= -\sigma(X - Y) \nonumber\\
    \dot{Y} &= rX - Y - XZ \nonumber\\
    \dot{Z} &= b(XY - Z)
\end{align}
with parameters canonically set at $\sigma = 10, b = 8/3, r =27 > 1$ (note that
$\sigma$ is 0.7 for an ideal gas, 1--4 for water and $>10$ for oils). Note that
the equations are dissipative, i.e.\ volumes in phase space shrink in dynamics.
Note also that they are \emph{autonomous} in that there is no explicit time
dependence on the right hand side, a convenient property.

It turns out that the solutions to this system are chaotic, or in Lorenz's
original terms ``aperiodic.'' We do many demonstrations that are no longer
accessible, including \emph{strange attractors, Poincar\'e section} and
\emph{1D maps}.

The last is going to be covered in a later lecture, but in case I'm interested
in simulating the former two, I look them up here (the website has discussions
of these with the applets absent):
\begin{itemize}
    \item Poincar\'e section examines the intersection of an orbit with a
        particular plane in phase space, e.g.\ $Z=31$ is a common one. So we
        just simulate and examine every time that $Z=31$ is crossed, exactly
        what $(X,Y)$ are.
    \item Strange attractors are subsets of phase space with fractional
        dimension (fractals) that orbits at long times lie in.
\end{itemize}

\section{Day 2 --- Pendula}

\subsection{Ideal Pendulum}

The ideal pendulum is given by EOM $\rtd{\theta}{t} + \frac{g\sin\theta}{l} =
0$. We can plot phase space $(\theta, \omega = \dot{\theta})$ for solution
trajectories. There are a few features of interest
\begin{itemize}
    \item Fixed points --- Two types, elliptic (stable but not asymptotically
        stable) and hyperbolic (unstable)
    \item Limit Cycles --- Solution trajectories.
    \item Homoclinic orbits --- Trajectories that connect the same hyperbolic
        fixed point. Heteroclinic orbits connect two different hyperbolic fixed
        points.
\end{itemize}

We can write down the Hamiltonian for this system $H = \frac{J^2}{2I} +
Mgl(1-\cos\theta)$. The importance of being able to do is that phase space
volume is preserved. What do we mean by this? Well, consider Hamilton's
canonical equations
\begin{align}
    \dot{\theta} = \pd{H}{J} &= \frac{J}{I} \nonumber\\
    \dot{J} = -\pd{H}{\theta} &= -Mgl\sin\theta
\end{align}
and if we define some phase space velocity $\vec{V} = \left( \dot{\theta},
\dot{J} \right)$ we can verify that $\nabla \cdot \vec{V} = 0$. In other words,
no attractors in phase space can exist for a Hamiltonian system.

\subsection{Dissipative Pendulum}

New EOM is $\rtd{\theta}{t} + \eta \rd{\theta}{t} + \frac{g}{l}\sin\theta = 0$.
This creates a fixed point at $(0,0)$, asymptotically stable (``linearly
stable'' is his terminology), and $(\pi,0)$ is suddenly linearly unstable, in
that we will exponentially deviate from the point (before, we still
periodically returned).

\subsection{Driving + damping}

Let's drive the pendulum and rescale variables to $\rtd{\theta}{t} + \gamma
\rd{\theta}{t} + \sin\theta = g\cos(\omega_D t)$. Under small angle
approximation (small amplitude of solution and driving), we know how to solve
this. What about with larger driving amplitudes though?

To handle this, we wish to solve numerically, which is much easier with an
autonomous system than with an explicit time dependence. Thus, we introduce
$\dot{\theta}_D = \omega_D$ and suddenly we can write
\begin{align}
    \dot{\theta} &= \omega \nonumber\\
    \dot{\omega} &= -\gamma \omega - \sin\theta + g\cos(\theta_D) \nonumber\\
    \dot{\theta}_D &= \omega_D
\end{align}

This is a handy trick to get an autonomous system at the expense of an
additional equation in the system of ODEs. We might also suspect that the
behavior of the system changes when we get an extra dimension of phase space.

Well, let's start by observing that $\vec{\nabla} \cdot \vec{v} = -\gamma$ so
phase space volumes contract. This might naively lead us to conclude that any
volume of initial conditions must contract to a point, but we mustn't be hasty;
the Lorenz model also has contracting phase space volumes. Instead, we know
that we might approach a limit cycle in a way such that while the overall phase
space contracts, the space might ``stretch'' along some dimensions and contract
in others such that chaotic dynamics are observed! The demonstrations
illustrating this are broken, but we will definitely simulate this.

\section{Day 3 --- Nonlinear Oscillators}

\subsection{Van der Pol Oscillator}

Consider the following model of a harmonic oscillator to include a non-linear
damping. Damping is negative at small amplitudes to model local instability,
but is positive for larger amplitudes; we thus expect the system to have some
limit cycles where the average damping vanishes. The EOM is
\begin{align}
    \ddot{x} - \gamma(1-x^2)\dot{x} + x &= g\cos(\omega_D t)
\end{align}

\subsubsection{Small $\gamma$, secular perturbation}

Let's first look at oscillations with no driving, $g=0$. We return to the
second order EOM for this. The natural starting point is to also examine for
small $\gamma$ and try Lindstedt-Poincar\'e perturbation theory (also called
secular perturbation theory) on
\begin{align}
    \ddot{x} + x &= \gamma(1-x^2)\dot{x}
\end{align}

We begin with the $\gamma=0$ solution, which is just $x(t) = ae^{it + \phi}$. Then, by secular perturbation theory\footnote{We learned as \emph{Lindstedt-Poincar\'e} in 106, but this seems to differ slightly; instead of setting up this slowly varying timescale as below, we tend to set $s = \omega t$ and expand $\omega = 1 + \epsilon \omega_1 +\dots$.}, when $\gamma \neq 0$, we can substitute $x(t) = A(t)\cos(t) + \gamma x_1 +\dots$ and also expect $A(t)$ to be some slow-varying function such that $\frac{\mathrm{d}^nA(t)}{\mathrm{d}t^n} \sim \gamma^n$ (this is introducing a second slow time scale, characteristic of secular perturbation theory apparently) and collect $O(\gamma^1)$ terms in the ODE to obtain
\begin{align}
    \rtd{}{t}\left[ A\cos(t) + \gamma x_1 \right] + A\cos(t) + \gamma x_1 &= \gamma(1 - A^2\cos^2(t))\rd{}{t}(A\cos(t)) \nonumber\\
    - 2\sin(t)\rd{A}{t} + \gamma\rtd{x_1}{t} + \gamma x_1 &= -\gamma A\sin(t) + \gamma A^3\underbrace{\cos^2(t)\sin(t)}_{\frac{\sin(t)}{4} + \frac{\sin(3t)}{4}} \nonumber\\
    \gamma\left(\rtd{x_1}{t} + x_1\right) &= -\gamma A\sin(t) + \gamma A^3\frac{\sin(t)}{4} + 2\sin(t)\rd{A}{t} + O(\sin(3t))
\end{align}
where we don't care about the $\sin(3t)$ term since it doesn't drive $x_1$ on resonance. On the other hand, we require the remaining term vanish, which implies that
\begin{align}
    2\rd{A}{t} = \gamma A - \frac{\gamma A^3}{4} = \gamma \frac{A(4 - A^2)}{4}
\end{align}
off of which we can read that any initial $A > 0$ tends to grow until $A = 2$, asymptoting at $x(t) = 2\cos(t)$.

\subsubsection{Large $\gamma$}

For large $\gamma$, let's stick to driving $g=0$ and go to a slightly unconventional choice of phase space variables, namely
\begin{align}
    \dot{y} &= \ddot{x} - \gamma(1-x^2)\dot{x} = -x\\
    \dot{x} &= y + \gamma\left( x - \frac{x^3}{3} \right)
\end{align}

We then rescle $y \to \gamma Y, t \to \gamma T, x \to X$ and introduce $\eta = \gamma^{-2}$ and obtain
\begin{align}
    \eta \rd{X}{T} &= Y + \left( X - \frac{X^3}{3} \right)\\
    \rd{Y}{T} &= -X
\end{align}
and for large $\gamma$ we now have small $\eta$. We first find that $\eta = 0$ is inconsistent; the first equation reduces to $Y = -\left( X - \frac{X^3}{3} \right)$, predicting a minimum of $Y = -\frac{2}{3}$ while the latter suggests $Y \to 0$ strictly decreasing-ly, contradictory. However, we can still analyze qualitatively what happens for small $\eta$, an orbit that comprises two pieces
\begin{itemize}
    \item $Y + \left( X - \frac{X^3}{3} \right)$ is small, so $Y \simeq \frac{X^3}{3} - X$ and $Y, X$ change on timescales $O(1)$.
    \item $Y + \left( X - \frac{X^3}{3} \right) \gg \eta\rd{X}{T}$, then $X$ changes very rapidly over timescale $O(\eta^{-1})$, and since $\rd{Y}{T} \sim O(1)$ we can treat $Y$ to be constant.
\end{itemize}

This breaks down to the orbit following $Y = \frac{X^3}{3} - X$ until reaching an extremum, at which point $\rd{X}{T} = 0$ and $Y + \left( X - \frac{X^3}{3} \right) \gg \eta \rd{X}{T}$, and suddenly we evolve at constant $Y$ until we hit another point on the $Y = \frac{X^3}{3} - X$ curve. The picture is then of following a cubic on $X < -1, X > 1$ and jumping from $\left(-1, \frac{2}{3}\right) \to \left( 2, \frac{2}{3} \right)$ and from $\left( 1, -\frac{2}{3} \right)\to \left( -2, -\frac{2}{3} \right)$ (the constant $Y$ portions of the curve).

This analysis allows us to compute $\dot{x} = v = \gamma\left[ Y - \left( -X + \frac{X^3}{3} \right) \right]$ as the difference between $\frac{X^3}{3}$ the cubic and $Y$ which tracks the cubic for $X < -1, X > 1$ and is constant during the jumps. We can simulate via Mathematica, and verify the shape of the trajectory.

This shows that there are free oscillations at any given $\gamma$.

\subsection{Driven Oscillations: Frequency Locking}

In general, when one drives an oscillator, one can either observe oscillations at both the free and driving frequency (in which case the power spectrum would show signatures at both frequencies), or we would see only a single frequency (and maybe harmonics). The latter case is called \emph{frequency locking}, when the driving frequency locks the free oscillation to some fixed frequency. The former case, when the driving and free oscillations are in an irrational ratio, is considered \emph{quasiperiodic}, because the sum of the two frequencies never repeats.

Another way to define frequency locking is that while some parameters of the problem (e.g.\ drive amplitude, dissipation) are varied, all peaks in the power spectrum must remain fixed for some non-zero parameter space. If the frequency is not locked, at least some peaks should vary continuously as parameters are varied continuously. It turns out that frequency locking occurs for the vDP oscillator for high drive amplitude and small frequency differences, which is generally true.

The basic idea behind analyzing frequency locking (beyond the scope of this class) is to take equation of motion
\begin{align}
    \ddot{x} + x &= \gamma(1-x^2)\dot{x} + g\cos(\omega_D t)
\end{align}
and substitute $\omega_D = 1 + \gamma \Delta$ and $g = 2\gamma F$ again in the $\gamma \ll 1$ limit, so weak driving near resonance. Performing the same secular perturbation theory (I concede, it is easier to use $x(t) = A(t)e^{it} + c.c. + \epsilon x_1$ here than cosines) and write driving term $\epsilon F e^{i \Delta \epsilon t}e^{it}$, we obtain that killing the secular term requires
\begin{align}
    \rd{A}{t} &= \frac{\gamma}{2}\left( A - \abs{A}^2 \right) - \gamma\frac{i}{2}Fe^{i\Delta \epsilon t}
\end{align}
then we can make the substitution $\tilde{A} = Ae^{-i\Delta \epsilon t}$ which yields
\begin{align}
    \rd{\tilde{A}}{t} + i\gamma \Delta \tilde{A} &= \frac{\gamma}{2}\left( \tilde{A}(1-\abs{\tilde{A}}^2) - iF \right)
\end{align}
upon which whether locking happens boils down to the nature of algebraic solutions to $\tilde{A}$ above. A stable \emph{fixed point} yields a locked solution (harmonics nearby tend to disappear) while stable \emph{limit cycles} produce unlocked solutions (two frequencies), since the Poincar\'e-Bendixson theorem tells us that no other asymptotic long term dynamics are possible. Unstable solutions are uninteresting; if both stable fixed and limit cycles exist then both locked and unlocked dynamics can manifest. The remainder of this is omitted.

It turns out that the vDP oscillator does not exhibit chaos.

\subsection{Duffing Oscillator}

The Duffing Oscillator, given by potential
\begin{align}
    V(x) &= \pm \frac{1}{2}x^2 + \frac{1}{4}x^4 + \frac{1}{4}
\end{align}
and EOM
\begin{align}
    \ddot{x} + \gamma \dot{x} \pm x + x^3 &= g\cos(\omega_D t)
\end{align}
does exhibit chaos for both signs apparently!

\section{Day 4 --- One Dimensional Maps}

Let's consider an arbitrary system
\begin{equation}
    \dot{U} = f(U|r)\label{4.eom}
\end{equation}
with $r$ some control parameters and $U$ a vector of phase space coordinates. We have studied solely such systems that are
\begin{description}
    \item[autonomous] no time appears on RHS
    \item[deterministic] no stochasticity in EOM
    \item[dissipative] volumes in phase space shrink, giving attractors
\end{description}

The structure of the phase space is of smooth \emph{vector fields} in $\mathbb{R}^N$, and the solution $U(t)$ is called a \emph{flow}. Casting in this language gives us strong categorizations of what is/isn't possible such as the \emph{Poincar\'e-Bendixson Theorem}.

\subsection{Maps}

In a map, we study an analogous system to\eqref{4.eom} but with a discrete timestep $U_{n+1} = F(U_n|r)$. The effect of the evolution on volumes in phase space is then given by the Jacobian
\begin{align}
    J = \abs{\ptd{F^{(i)}}{U^{(j)}}}
\end{align}

If $J = 1$ then the map is called \emph{volume preserving}, otherwise \emph{dissipative}. A few ways to map flows to maps:
\begin{itemize}
    \item Integrate the flow for $n\tau$ with $\tau$ some fixed time interval.
    \item $N-1$ dimensional Poincar\'e section.
\end{itemize}

Some notes about flows:
\begin{itemize}
    \item Conventionally we scale maps to map $[0,1]$ onto itself.
    \item Successive iterations of the map are notated $F^n(x_0)$.
    \item Defining $x_f: F(x_f) = x_f$ fixed point, we can also ask the whether the fixed point is stable. The usual technique is to linearize the map about $x = x_f + \delta x$ and compute the Taylor expansion $\delta x_{n+1} = F'(x_f) \delta x_n$. We thus observe stability if $F'(x_f) < 1$ and instability for the opposite.
\end{itemize}

\subsubsection{Bifurcations}

Suppose we have a map $F(x|a)$ with some parameter $a$. Then, as $a$ is varied, it is possible that the long time solution of the map changes. These changes are called bifurcations. We depict bifurcations by plotting long time stable $F^k(x|a)$ as a function of $a$. For instance, a period 2 orbit that is exhibited for some value of $a$ is plotted as two points $(a, x_1), (a, x_2)$ ($F^2(x_1) = F(x_2) = x_1$). Regions of $a$ for which a fixed point orbit exists show up as a single curve, and regions with chaotic dynamics show up as a continuum of points.

To be able to better describe these chaotic regions, we look to a statistical description. The \emph{invariant measure} is the probability density of $x$ values that an orbit visits. We begin by defining the \emph{measure}.
\begin{align}
    \rho(x,x_0) \;\mathrm{d}x = \lim_{N \to \infty} \frac{1}{N} \times n(x,x_0)
\end{align}
where $\rho(x,x_0)$ is the density of points with initial condition $x_0$ and $n(x,x_0)$ is the number of times the orbit visits $[x-\mathrm{d}x, x+\mathrm{d}x]$ out of $N$ iterations.

It turns out that this measure $\rho(x,x_0)$ doesn't actually depend on $x_0$ for almost all choices of $x_0$, so we define $\rho(x)$ the invariant measure of our attractor. Sometimes the invariant measure can be constructed directly from its definition
\begin{align}
    \rho_n(y) = \rho_{n+1}(y) = \int \mathrm{d}x\; \delta \left[ y - F(x) \right]\rho_n(x)
\end{align}

\subsubsection{Lyapunov Exponents}

We can better quantify ``sensitive depndence on initial conditions'' by Lyapunov Exponents. We demand that for some pair of initial conditions $x_0, x_0 + \epsilon$ for the distance to grow as $\abs{\delta x_n} = \epsilon e^{n\lambda(x_0)}$. $\lambda(x_0)$ is the Lyapunov Exponent
\begin{align}
    \lambda(x_0) = \lim_{n \to \infty}\lim_{\epsilon \to 0}\log\abs{\frac{F^n(x_0 + \epsilon) - F^n(x_)}{\epsilon}} = \lim_{n \to \infty}\frac{1}{n}\log \abs{\rd{F^n(x_0)}{x_0}}
\end{align}

For systems with ergodic invariant measure, $\lambda$ is independent of $x_0$, so we will call $\lambda$ \emph{the Lyapunov Exponent of the map}. Note that the derivative is of the $n$-th iteration, and it turns out that
\begin{align}
    \lambda = \lim_{n \to \infty}\frac{1}{n}\log\abs{\rd{F^n(x_0)}{x_0}} = \lim_{n \to \infty}\frac{1}{n}\log \abs{F'(x_{n-1})F'(x_{n - 2})\dots F'(x_1) F'(x_0)} = \expvalue{\log\abs{F'}}
\end{align}

A positive $\lambda$ means that closely spaced initial conditions diverge exponentially, which is a signature of chaos and often used as a definition thereof.

\section{Day 5 --- Two Dimensional Maps}

One-dimensional maps are a restricted view of most dynamical systems, which cannot simply be iterated in reverse to find the initial condition (many non-invertible systems). On the other hand, integrating most dynamical equations in reverse can yield the preimage of a trajectory, which leads us to \emph{two}-dimensional maps as a faithful representation of a smooth flow in 3D phase space. Here, we will investigate four examples of 2D maps.

\subsection{Henon Map}

Consider the Henon map, which iterates
\begin{align}
    x_{n+1} &= y_n + 1 - ax_n^2 & y_{n+1} = bx_n
\end{align}

The Jacobian is $J =
\begin{vmatrix}
    -2ax_n & 1\\
    b & 0
\end{vmatrix} = |b|$, so for $b=1$ the map is area preserving and for $b < 1$ the map is dissipative, where areas in phase space contract. This is where the chaos lives. To get some intuition about what the strongly dissipitive limit looks like, we can substitute $x_{n+1} = bx_{n-1} + 1 - ax_n^2$ which for $b \to 0$ is simply $x_{n+1} \approx 1 - ax_n^2$ the quadratic map we studied in the previous chapter (that I ignored).

Canonically chosen values are $a = 1.4, b = 0.3$.

\subsection{Baker's Map}

The Baker's map is given by the mapping
\begin{align}
    x_{n+1} &=
    \begin{cases}
        \lambda_a x_n & y_n < \alpha\\
        (1 - \lambda_b) + \lambda_b x_n & y_n > \alpha
    \end{cases}
    y_{n+1} &=
    \begin{cases}
        y_n/\alpha & y_n < \alpha\\
        (y_n - \alpha) / \beta & y_n > \alpha
    \end{cases}
\end{align}
with $\beta = 1-\alpha, \lambda_a + \lambda_b \leq 1$, so points in 2D phase
space on either side of the $y=\alpha$ boundary are mapped to $x < \lambda_a, x
> 1-\lambda_b$ respectively. Then when $\lambda_a + \lambda_b < 1$, the total
phase space area decreases and the map is \emph{dissipative}. Qualitatively, we
see that the number of vertical striations increases with increasing iterations,
specifically with the $n$th iteration we have $\binom{n}{m}$ stries of width
$\lambda_a^m\lambda_b^{n-m}$.

Note that for the Bakers' map, the attractor is \emph{hyperbolic} in that,
crudely, at each point phase space both an expanding and contracting direction
can be defined such that the directions vary continuously across the map and are
bounded to be nonzero. In the case of the Bakers' map, the map is expanding in
the $y$ direction and contracting in the $x$ across the entire map and so is
hyperbolic. Not many maps/flows are hyperbolic, but many useful properties can
be proven in such cases.

\subsection{Other 2D maps}

Other interesting maps are the Duffing and the Kaplan-Yorke, also simulated in
our tests. We are unable to simulate the Baker's Map because its chaos relies on
how regions of phase space map and our simulation module is not presently
powerful enough to handle this.

\end{document}

